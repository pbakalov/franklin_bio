%! program = pdflatex

\documentclass[12pt]{book} 
\usepackage{cmap} % otherwise cyrillic is not searchable
\usepackage[utf8x]{inputenc}
\usepackage[bulgarian, english]{babel}
\usepackage[parfill]{parskip} % vertical space between paragraphs 
\usepackage[unicode]{hyperref} 
\usepackage[a4paper,margin=0.9in]{geometry}
\usepackage{graphicx} 
\usepackage{titlesec} % allows us to modify chapter headers

\titleformat{\chapter}[display] % remove 'Chapter N' from chapter headers
  {\normalfont\bfseries}{}{0pt}{\Large}

\title{Автобиография}
\author{Бенджамин Франклин}
\date{} % Delete this line to display the current date

%%% BEGIN DOCUMENT
\begin{document}
%\selectlanguage{english}
\selectlanguage{bulgarian}

\maketitle
\pagestyle{empty} % removes headers and footers from following pages (except TOC, apparently)

\large
Превод от английски (засега чернова): \href{http://tropcho.wordpress.com}{Тропчо (tropcho.wordpress.com)} (2013 - 2015). Преводът се разпространява под лиценз CC-by-SA 3.0. Английският текст на автобиографията може да бъде намерен \href{http://www.gutenberg.org/ebooks/20203}{примерно на страницата на проекта Гутенберг}.

\tableofcontents
\thispagestyle{empty} % removes headers and footers from this page (no page number)

\chapter{Част първа}
\pagestyle{plain} % removes headers & keeps page numbers
\setcounter{page}{1}
\textit{\textbf{Туифърд, при Епископа на Свети Асаф, 1771}}

Мили сине,

Събирането на всякакви разкази за предците ми винаги ми е доставяло удоволствие. Може би си спомняш за проучванията, които направих сред останалите ми роднини, когато беше с мен в Англия, както и пътешествието, което предприех с тази цел. Понеже предполагам, че на теб би ти било също толкова приятно да научиш за обстоятелствата на моя живот, с много от които все още не си запознат, и очаквайки насладата от една седмица свободно време прекарана в настоящото ми провинциално убежище, сядам да ги опиша за теб. За това имам и други причини. Тъй като се издигнах от бедността и безизвестността, в които бях роден и отгледан, до богато състояние и известна репутация в света, и тъй като преминах през живота си досега със значителен дял щастие, наследниците ми може да пожелаят да научат кои са ръководните средства, с които съм си служил, и които с Божията благословия постигнаха такъв успех, защото някои от тях могат да се окажат подходящи и в техните обстоятелства, и следоветелно удачни за подражание. 

Като се замисля, това щастие понякога ме е карало да казвам, че ако ми бъде предложено да избирам, не бих имал възражения да повторя същия живот отначало, като само бих поискал преимуществото, което авторите имат при второ издание, да поправя някои грешки от първото. Освен да поправя грешките, така бих могъл и да променя някой зловещи произшествия и събития, които се случиха в него с по-благоприятни. Но дори това право да ми бъде отказано, пак бих приел предложението. Тъй като такова повторение не може да се очаква, най-сходен с изживяването на един живот отново изглежда споменът за този живот, и записването на този спомен, за да стане той възможно най-траен. 

Тук също ще задоволя толкова естествената склонност на старите да говорят за себе си и собствените си минали дела; и ще я задоволя без да дотягам на околните, които от уважение биха се чувствали задължени да ме изслушат, защото това може да бъде четено по желание. И на последно място (даже мога и да си го призная, тъй като, ако отрека, никой няма да ми повярва), може би до голяма степен ще удовлетворя суетата си. Наистина почти никога не съм чувал или виждал уводните думи “Без суета мога да кажа” и т.н., без те да бъдат непосредствено последвани от нещо суетно. Повечето хора не харесват суетата у другите, какъвто и дял от нея да притежават; но тъй като съм убеден, че тя често носи добро на приносителя си и на хората в неговото обкръжение, аз й отдавам подобаващо място, където и да я срещна; и затова в много случаи не би било напълно абсурдно човек да благодари на Бога за суетата си, наред с другите утешения в живота.  

И като говоря за благодарност към Бога, желая с пълно смирение да призная, че дължа споменатото щастие на изминалия ми живот на Неговото милостиво провидение, което ме доведе до средствата, които използвах, и ги увенча с успех. Вярата ми в това ме кара да се надявам, макар че не бива да предполагам, че същата добрина ще продължи да ме съпътства чрез продължаването на това щастие, или като ми даде сили да понеса съдбовен обрат, какъвто може да ме сполети, както се е случвало с други: изражението на бъдещето ми бидейки известно само Нему, в чиято власт е да ни благослови дори чрез несгодите ни. 

Щом попаднаха в ръцете ми, бележките на един от чичовците ми (който изпитвал същото любопитство към събирането на семейни анекдоти) ми набавиха няколко подробности свързани с нашите предци. От тези бележки научих, че семейството бе живяло в същото село, Ектън, Нортемптъншайър, в продължение на триста години, но колко повече от това, той не знаеше (може би от времето когато името Франклин, което преди е било име на обществена прослойка, е било прието от тях като фамилно име, когато и други в цялото кралство приели фамилни имена), върху парцел от около тридесет акра, издържайки се чрез ковашкия занаят, който се предавал в семейството до негово време, като най-големият син винаги бил обучаван на този занаят; обичай, който той и баща ми спазиха по отношение на първородните си синове. Когато прегледах архивите на Ектън, открих записи за техните раждания, сватби и погребения едва от 1555 година, тъй като преди това в тази енория не бил поддържан архив. От този архив научих, че съм най-младият син на най-младия син от пет поколения назад. Дядо ми Томас, който се родил през 1598, живял в Ектън докато старостта не го принудила да остави занаята си, след което отишъл да живее със сина си Джон, багрилар в Банбери, Оксфордшайър, при когото баща ми служил като чирак. Дядо ми починал и бил погребан там. Видяхме надгробния му камък през 1758. Най-големият му син Томас живял в къщата в Ектън и я оставил със земята на единственото си дете – дъщеря – която заедно със съпруга си – един Фишър от Уелингборо – я продала на г-н Истед, сега господар на имота там. До зряла възраст доживяли четири от дядовите ми синове, а именно: Томас, Джон, Бенджамин и Иосия. За тях ще ти разкажа каквото мога на такова разстояние от бележките си, и, ако последните не се загубят докато ме няма, в в тях ще намериш много други подробности. 

Томас бил обучен от баща си за ковач; но, понеже бил надарен и насърчаван в ученето (като всичките ми братя) от господин Палмър \footnote{Esquire Palmer}, по онова време първенец в тази енория, добил квалификация за писар; добил положение в окръга; бил основен двигател на всички обществени предприятия за общината или за града Нортемптън, както и за собственото му село, поради което му се носела славата, и се ползвал с вниманието на и бил подкрепян от тогавашния Лорд Халифакс. Умрял на 6 януари (стар стил) 1702, точно четири години преди аз да се родя. Спомням си, че разказът, който научихме за неговия живот и характер от някои хора в Ектън, те впечатли като нещо особено, поради приликата си с това, което знаеше за моя. 

- Ако беше умрял на същия ден - каза ти - можеше да предположиш прераждане.

Ако не се лъжа, Джон бил обучен за багрилар на вълна. Бенджамин бил обучен за багрилар на коприна и чиракувал в Лондон. Той беше умен мъж. Помня го добре, защото когато бях момче, дойде при баща ми в Бостън и живя няколко години с нас в къщата. Доживя до дълбока старост. Внукът му, Самюел Франклин, сега живее в Бостън. Остави след себе си два сборника ръкописи в quarto формат със собствената си поезия, съдържащи тук-таме малки бележки адресирани до приятелите и роднините му, от които следната, изпратена до мен, е пример\footnote{ тук в полето следват думите (в скоби) „добави тук“, но стиховете ги няма. Г-н Спаркс ни казва (\textit{Life of Franklin}, с. 6), че тези сборници са запазени и в момента са в ръцете на г-жа Emmons от Бостън, правнучка на създателя им.}. Беше измислил и собствена система за бързопис, която ми и предаде, но тъй като никога не я използвам, вече съм я забравил. Кръстен съм на този чичо поради особената обич между него и баща ми. Той беше много религиозен, редовен слушател на проповедите на най-добрите проповедници, които стенографираше със своята система, и имаше много сборници с тях. Падаше си и бая политик; може би прекалено за неговото положение. Наскоро в Лондон ми попадна една колекция, която той беше събрал, на основните памфлети свързани с обществени дела от 1641 до 1717г; от номерацията личи, че много от книжките липсват, но все пак остават осем във фолио формат, и двадесет и четири в \textit{quarto} и \textit{octavo}. Един антикварен книжар, който ме познава, понеже понякога пазарувам от него, се натъкнал на тях и, ми ги донесе. Изглежда, че чичо ми ги беше оставил там когато заминавал за Америка, което бе станало петдесет години по-рано. В полетата има много негови бележки.

Това наше невзрачно семейство рано се включило в Реформацията и постоянствали като протестанти през управлението на Кралица Мария\footnote{Мария I Тюдор, Кървавата Мери 1553-1558, бел. пр.}, по време на което понякога било в опасност поради ревността си срещу папщината. Имали английска Библия, и за да я скрият на сигурно, я закрепяли с каиши отворена от долната страна на столче. Когато пра-пра-дядо ми четял на семейството си, той обръщал столчето върху коленете си и така отгръщал страниците изпод каишите. Едно от децата стояло на вратата, за да предупреди, ако види съдията\footnote{apparitor} да идва, който бил служител на духовния съд. В такъв случай столчето било поставяно на краката си, a Библията оставала скрита под него, както преди. Тази история знам от чичо ми Бенджамин. Цялото семейство продължило да се придържа към Англиканската църква до около края на управлението на Чарлз Втори, когато някои от свещенослужителите, които били изхвърлени от църквата за разколничество, почнали да организират тайни събрания; Бенджамин и Иосия се присъединили към тях и продължили с тях цял живот; останалата част от семейството останала с Епископалната църква \footnote{Англиканската църква, бел. пр.}. 

Баща ми Иосия се оженил млад и пренесъл жена си с три деца в Ню Инглънд около 1862г. Забраната със закон на разколните събрания и честите им смущения накарали някои влиятелни познати на баща ми да се преместят в тази страна, и те го убедили да ги придружи там, където очаквали да практикуват вярата си по-свободно. От същата жена имал още четири деца, а от втората – още десет, общо – седемнадесет; от които аз си спомням тринадесет да седат едновременно с него на масата, които всички станаха мъже и жени и се задомиха; аз бях най-младият син, имах две по-малки сестри, и съм роден в  Бостън, Нова Англия. Майка ми, втората жена, беше Абия Фолгър, дъщеря на Питър Фолгър\footnote{Folger}, един от първите заселници на Нова Англия, когото Котън Мейдър\footnote{Cotton Mather} споменава с почит в църковната история на този край, наречена \textit{Magnalia Christi Americana}, като “благочестив, образован англичанин”, ако си спомням думите правилно. Чувал съм, че писал различни малки произведения отвреме на време, но само едно от тях, което видях преди много години, било отпечатано. Беше написано през 1675г. в импровизирана\footnote{home-spun} стихотворна форма от онова време и край, и беше адресирано до онези заети с управлението там по онова време. Беше в подкрепа на свободата на съзнанието от името на баптисти, куейкъри и други сектанти, които бяха преследвани, и приписваше на тези гонения Индианските войни и други бедствия, които бяха сполетяли страната, като Божии съд за наказание за такова ужасно престъпление, и призоваваше към отмяна на тези сурови закони. Цялото ми се стори написано с добра доза благоприлично простодушие и мъжествено свободолюбие. Спомням си последните шест стиха, макар че не помня първите два от строфата; но техния смисъл беше, че тъй като критиката му е плод на добра воля, би искал да се знае, че той е авторът.



	“Защото да съм клеветник,
	аз мразя от сърце;
	от град Шербърн\footnote{Sherburne}, където сега живея,
	помествам името си тук;
	без лоши чувства, твой истински приятел
	е Питър Фолджър\footnote{Folgier}

Всичките ми по-големи братя бяха дадени за чираци в различни занаяти. Мен ме пратиха в класическо училище\footnote{\textit{ grammar school } според речника на Самюел Джонсън от 1755 това е училище в което се учи граматиката на класическите езици} на осем години, като баща ми имаше намерение да ме посвети на църквата като десятък от синовете си. Това, че рано се научих да чета (което трябва да е било много рано, защото не си спомням времето, когато не можех да чета), а и мненията на всичките му приятели, че със сигурност от мен ще излезе добър учен, го насърчаваха в това негово намерение. Чичо ми Бенджамин също го одобряваше и предложи да ми даде всичките си сборници със стенографирани проповеди, предполагам като материал, с който да се упражнявам, ако реша да уча системата му. Въпреки това се задържах в училище не повече от година, при все че през това време постепенно се бях издигнал от средата на тазгодишния клас до негов първенец, и след това бях преместен в този над него, за да премина в по-горния в края на годината. Но междувременно, с оглед на високата цена на университетското образование, която баща ми не можеше лесно да си позволи с голямото си семейство, както и на малкия доход, който мнозина добре образовани успяваха по-късно да си осигурят – причини, които баща ми изложи пред приятелите си в мое присъствие – баща ми промени първоначалното си намерение и ме премести от класическото училище в училище за писане и смятане, ръководено от един известен тогава мъж, г-н Джордж Браунуел, много успешен в професията си като цяло, и при това с меки, насърчителни методи. При него доста скоро се научих на краснопис, но се провалих в смятането и не никак не напреднах в него. На десет ме прибраха вкъщи, за да помагам на баща ми в работата му, която беше да лее лоени свещи и да вари сапун, работа, за която той не беше обучен, а беше захванал при пристигането си в Нова Англия, когато открил, че отмиращият му занаят няма да издържа семейството му, при ниското търсене. Както подобава, аз бях зает с рязане на фитили за свещите, пълене на отливката за топене и отливките за лети свещи, наглеждане на магазина, ходене по задачи и т.н.
 
Не харесвах занаята, а морето силно ме влечеше, но баща ми се обяви против това; но живеейки близо до водата често бях в или около нея, научих се рано да плувам добре и да управлявам лодки; и когато бях в лодка или кану с други момчета, обикновено ми позволяваха да водя, особено в трудни случаи; и при други случаи обикновено бях водач сред момчетата, и понякога ги въвеждах в неприятности, за което ще разкажа след малко, тъй като случката показва рано изразен дух на общественик, макар и в този случай неправилно насочен.

Имаше едно соленоводно блато, което граничеше с вира на воденицата, и на чийто край стояхме при прилив и ловяхме лещанки. От много тъпкане го бяхме превърнали в рядка кал. Предложението ми беше да построим там кей, на който да можем да стоим, и показах на другарите си една голяма купчина камъни предназначени за нова къща близо до блатото, които щяха да ни свършат много добра работа. Както подобава, вечерта след като си бяха отишли работниците събрах няколко от приятелите ми за игра и работейки усърдно като мравки, понякога по двама-трима на камък, отнесохме всички камъни и построихме малкия си кей. На другата сутрин работниците се изненадали, че им липсват камъните, които били открити в нашия кей. Разпитали за извършителите; бяхме разкрити и се оплакаха от нас; някои от нас бяха поправени от бащите си; и въпреки че изтъкнах полезността на работата, баща ми ме убеди, че нищо не е полезно, което не е честно. 

Мисля, че може да ти иска да научиш нещо за него и за характера му. Имаше превъзходно телосложение, среден на ръст, но добре сложен, и много силен; беше находчив, умееше да рисува хубаво, имаше малко музикални умения и ясен, приятен глас, така че беше изключително приятно да го слушаш как свири псалми на цигулката си и пее, както понякога правеше вечер след като приключеше с работата за деня. Имаше и дарба на механик и понякога беше много сръчен в употребата на инструменти на други занаятчии; но истинското му превъзходство беше в дълбокото му разбиране на и надеждна преценка по въпроси които изискват прозорливост и мъдрост, както в личните така и в обществените дела. Наистина той никога не бе се занимавал с последните, тъй като многобройното му семейство, което трябваше да образова, и притесненото му положение го държаха близо до занаята му; но си спомням добре, че често бе посещаван от видни хора, които се допитваха до него по градските дела или делата на църквата, към която принадлежеше, и показваха доста уважение към преценката и съветите му: в случай на трудности често беше търсен за съвет и от частни лица във връзка с техните дела, а и често беше избиран за арбитър между спорещи страни.

Винаги, когато имаше възможност, обичаше да кани на масата у дома за разговор някой разумен приятел или съсед, и винаги се стараеше да подхване някоя хитроумна или полезна тема за разговор, която би могла да подобри умовете на децата му. По този начин той насочваше вниманието ни към това, което е добро, справедливо и благоразумно в живота; и на храната на масата – дали е добре или зле приготвена, според сезона или не, добра или лоша на вкус, за предпочитане пред или по-лоша от това или онова подобно ястие – се обръщаше много малко или почти никакво внимание, така че бях възпитан в такова пълно невнимание към тези неща, че въобще не ми прави впечатление каква храна слагат пред мен, и толкова не я забелязвам, че и до ден днешен ми е трудно да кажа какво съм вечерял, ако ме попитат няколко часа след вечеря. Това е било много удобно за мен при пътувания, при които понякога спътниците ми са били много нещастни поради липсата на подходящо удовлетворение за по-добре обучените им и затова по-деликатни вкусове и апетити. 

Майка ми също беше много здрава: откърми всичките си десет деца. Не си спомням майка ми и баща ми някога да са боледували освен от болестите, от които умряха – той на 89, тя на 85 години. Погребани са заедно в Бостън, където преди няколко години поставих мраморна плоча със следния надпис на гроба им: 

ИОСИЯ ФРАНКЛИН \\
и \\
жена му – АБИЯ –  \\
лежат погребани тук. \\
Живяха в любов венчани \\
петдесет и пет години.  \\
Издържаха голямо семейство в уют, \\
отгледаха тринадесет деца  \\
и седем внука  \\
почтено. \\
От този пример, читателю, \\
бъди насърчен да си усърден в призванието си, \\
и не се съмнявай в Провидението. \\
Той беше благочестив и разумен мъж; \\
тя – разумна и добродетелна жена. \\
Най-младият им син, \\
от синовна почит към паметта им, \\
поставя този камък. \\
И.Ф. роден 1655, починал 1744, на 89г. \\
А.Ф. родена 1667, починала 1752, на 85г.  \\

По несвързаните си отклонения познавам, че съм остарял. Едно време пишех по методично. Но човек не се облича за среща с приятел така, както се облича за обществен бал. Може би е просто немарливост.

Да се върна на думата си: така продължих зает в бащиния си занаят две години, тоест докато навърших дванадесет години; а тъй като брат ми Джон, който беше обучен за този занаят, беше вече напуснал баща ми, беше се оженил и отворил дюкян в Роуд айлънд, по всичко личеше, че на мен ми е предопределено да заемо неговото място и да стана свещолеяр. Но тъй като продължавах да не харесвам занаята, баща ми се опасяваше, че ако не намери някой, който по ми допада, ще избягам и ще стана моряк, както синът му Иосия беше направил за голямо негово съжаление. Затова понякога ме взимаше с него на разходка, за да видя как работят дърводелци, зидари, стругари, пещари, и т.н., за да установи какви са склонностите ми, и да се опита да ги фиксира върху някой сухопътен занаят. Оттогава винаги ми е доставяло удоволствие да гледам как добри работници използват инструментите си. Също ми е било полезно, понеже от това научих достатъчно, че да мога да се справям с някои малки работи сам у дома, когато не е имало възможност да се извика човек, и да конструирам малки машини за експериментите ми, докато намерението да направя експеримент е свежо и топло в ума ми. Баща ми най-накрая избра ножарския занаят и бях изпратен при сина на чичо ми Бенджамин, Самюел, който беше обучен в този занаят, и който беше по това време установен в Бостън, да прекарам известно време при условие, че му се харесам. Но неговото очакване да му нося и приход не допаднаха на баща ми и отново бях прибран вкъщи.

От дете обичах да чета и всичките дребни пари, които попадаха в ръцете ми отиваха за книги. Понеже „Напредъкът на поклонника“\footnote{Pilgrim's Progress, на български преведена като „Пътешественикът от този свят до онзи“} ми хареса, първата ми колекция беше от творби на Джон Бъниън\footnote{John Bunyan} в отделни малки томчета. По-късно ги продадох, за да мога да купя „Историческите колекции“ \footnote{Historical Collections} на Р. Бъртън\footnote{R. Burton}. Това бяха малки книжки, и евтини, общо 40 или 50. Малката библиотека на баща ми се състоеше предимно от книги за богословски спорове, повечето от които прочетох, и оттогава често съм съжалявал, че по време, когато имах такава жажда за знание, по добри книги не ми попаднаха, тъй като по това време беше вече решено, че няма да бъда духовник. „Успоредните животописи“ на Плутарх бяха там, от които четох много, и все още смятам това време за много добре употребено. Имаше и една книга на Дефо наречена „Есе за проекти“\footnote{Essay on Projects}, и една на Др. Мейдър, наречена „Опити за правене на добро“\footnote{Essays to do good}, която ми даде начин на мислене, който оказа влияние върху някои от основните бъдещи събития от живота ми.

Книжовните ми наклонности в крайна сметка убедиха баща ми да ме направи печатар, макар вече да имаше един син (Джеймс) с тази професия. През 1717 брат ми Джеймс се върна от Англия с преса и букви, за да започне собствен бизнес в Бостън. Много повече го харесвах от този на баща ми, но все още копнеех по морето. За да предотврати очаквания ефект от тази моя наклонност, баща ми бързаше да ме даде чирак при брат ми. Удържах известно време, но накрая бях убеден и подписах договорът за чиракуване; тогава бях едва на дванадесет години. Щях да служа до двадесет и първата си година, но през последната година щях да получавам заплата като наемен работник. За кратко време научих много за тази работа и се превърнах в полезен помощник на брат си. Сега имах достъп до по-добри книги. Познанствата ми с чираците на продавачи на книги ми позволяваха да вземам на заем малки книжки, които внимавах да върна скоро и чисти. Често седях в стаята си и прекарвах в четене по-голямата част от нощта, ако книгата взета на заем вечерта трябваше да бъде върната рано сутринта, да не би някой да я потърси и забележи, че липсва.

И след известно време един хитроумен търговец, г-н Матю Адамс, който имаше прилична колекция книги, и който често посещаваше нашата печатница, ме забеляза и ме покани в библиотеката си, и много любезно ми заемаше книгите, които избирах да чета. По това време се увлякох по поезията и написах някои къси стихотворения. Брат ми, мислейки, че може да изкара нещо от това, ме насърчаваше, и ми поръчваше от време на време да композирам балади. Една се казваше „Трагедията при фара“ и съдържаше историята на удавянето на Капитан Уъртилейк с двете му дъщери. Другата беше моряшка песен за залавянето на пирата Тийч (или Блекбиърд). Бяха окаяни произведения, в стила на Гръб-стрийт\footnote{Grub street - улица в Лондон прочута с големия си брой бедни писатели. Към осемнадесети век вече е нарицателно за долнопробна литература.} баладите, и щом бъдеха отпечатани, той ме изпращаше да ги продавам из града. Първата се продаваше чудесно, тъй като събитието се беше случило наскоро и беше вдигнало голяма шумотевица. Това поласка суетността ми, но баща ми ме обезкуражи, като се подигра на творбите ми и ми каза, че стихописците обикновено са просяци. Така че ми се размина да стана поет, най-вероятно лош. Но тъй като писането в проза ми е било много полезно през живота, и беше основно средство за издигането ми, ще ти разкажа как и в каква ситуация натрупах малкото умения, които имам в тази област. 

Имаше още един младеж в града, който обичаше да чете. Казваше се Джон Колинс и бяхме близки приятели. Понякога влизахме в спорове, много обичахме дебатирането, и имахме голямо желание да се оборим един друг. Тази полемична нагласа, между другото, има склонността да се превръща в лош навик и прави хората изключително неприятна компания поради противоречията, които са нужни за прилагането ѝ на практика. И по този начин, освен че вкисва и разваля разговора, води до отвращение и – може би – вражди, където би могло да има приятелство. Прихванах я докато четях бащините си книги за религиозни спорове. Оттогава съм забелязал, че разумните хора рядко имат такава нагласа, освен адвокати, хора, които са били в университет и всякакви хора отраснали в Единбург.
 
По един или друг начин, веднъж закачихме въпроса дали е правилно жените да бъдат образовани, и за техните способности да се учат. Той смяташе, че не е правилно, и че по природа не им се отдава. Аз бях на противоположното мнение, може би донякъде за спора. Той беше по природа по-красноречив, имаше изобилие от думи под ръка, и ми се струваше, че понякога ме притиска повече с изразителността си, отколкото със силата на аргументите си. Тъй като се разделихме без да сме приключили разговора и нямаше да се видим известно време, аз седнах и поставих аргументите си на хартия, преписах ги, и му ги изпратих. Три или четири писма бяха изпратени от всяка страна, когато баща ми случайно намери книжата ми и ги прочете. Без да влиза в спора, той използва случая да ми говори за начина ми на писане. Отбеляза, че въпреки че превъзхождам антагониста си в правопис и пунктуация (което дължах на печатницата), изоставах в елегантността на израза си, в метод и в яснота, в които неща той ме убеди с няколко примера. Видях, че забележката му бе справедлива  и оттогава почнах да отделям повече внимание на начина си на писане, и реших да опитам да го подобря.

По това време попаднах на странен брой на \textit{the Spectator}. Беше третият. Преди това не бях виждал нито един брой. Купих го и го прочетох много пъти и бях много доволен от него. Бях на мнение, че начина на писане е отличен и реших – ако мога – да го имитирам. С тази мисъл наум, взех някои от статиите, загатнах накратко смисъла на всяко изречение, оставих ги настрана за няколко дни и след това, без да гледам в книгата, се опитах да пресъздам статийте, като изразя смисъла на всяко изречение обширно и изцяло както в оригинала, с каквито подходящи думи ми попаднат. След това сравних моя \textit{Spectator} с оригинала, открих някои от грешките си и ги поправих. Но открих, че нямам богат речник, нито умението да си припомням и използвам думи, което – си помислих – нямаше да е проблем, ако бях продължил да създавам стихове, тъй като постоянната нужда от думи с едно и също значение, но с различна дължина – за да отговарят на ритъма, или с различно звучене – за римата, щеше да ме принуди постоянно да търся разнообразие, а и щеше да ме кара да запаметя това разнообразие, и да го овладея. Затова взех някои разкази и ги обърнах в стихове, а след време, когато бях позабравил до голяма степен прозата, ги преведох обратно. Понякога разбърквах колекцията си от загатвания и след няколко седмици се опитвах да ги подредя по правилния начин преди да започна да пиша пълните изречения и да пресъздавам статиите. Целта на това беше да науча метод за организиране на мислите. Чрез сравнение на работата ми с оригинала открих много свои грешки и ги поправих, но понякога с удоволствие си въобразявах, че в някои дреболии съм имал достатъчно късмет да подобря начина на изразяване и това ме насърчи да мисля, че може би с времето ще се науча да пиша на английски поносимо, което беше голяма моя амбиция. Времето за тези упражнения и за четене беше през нощта, след работа, или преди да започна сутрин, или в неделя, когато нареждах нещата, така че да съм сам в печатницата, като се опитвах да избягвам, доколкото мога, обичайното присъствие на обществено поклонение, което баща ми изискваше от мен, докато се грижеше за мен, и което все още считах за задължение, въпреки че нямах възможност – както ми се струваше – да отделя време, за да го практикувам.

Когато бях на около 16 години, са натъкнах на една книга написана от някой си Трайън\footnote{Tryon}, която препоръчваше вегетарианска диета. Реших да премина на нея. Понеже брат ми беше неженен, не поддържаше дом, а се хранеше с чираците си при чуждо семейство. Отказът ми да ям месо създаде неудобство, и често бях хокан за странността си. Запознах се с начина, по който Трайън приготвя някои от ястията си, като варене на картофи или ориз, качамак\footnote{по-точно \textit{hasty pudding}}, и няколко други, и предложих на брат ми да си готвя сам, ако ми дава седмично половината от парите, които плаща за храната ми. Той незабавно се съгласи и скоро открих, че мога да спестявам половината от това, което ми плащаше. Това бяха допълнителни средства за закупуване на книги. Но имаше и още едно предимство в промяната. Брат ми и останалите излизаха от печатницата за да се хранят, а аз оставах сам, и след като бързо изяждах лекия си обяд, който често бе не повече от бисквита или филия хляб, шепа стафиди или плодов сладкиш от сладкаря, и чаша вода, можех да използвам останалото време преди тяхното завръщане за учене, и по този начин напредвах още по-бързо, поради по-бистрия ум и по-бързата мисъл, които обикновено съпътстват умереността в яденето и пиенето.

И по това време стана така че, след като по някакъв случай бях засрамен от неумението си да смятам, което на два пъти не бях успял да са науча да правя докато бях в училище, взех Аритметиката на Кокър и минах през цялата книга сам и с голяма лекота. Също прочетох книгите на Селър и Шерми по навигация и се запознах с малкото геометрия, която съдържат, но никога на стигнах далеч в тази накуа. И по горе-долу това време прочетох  „Есе за човешкия разум“ на Лок\footnote{\textit{An Essay Concerning Human Understanding} на John Locke} и „Изкуството на мисленето“ на господата от Пор Роял.\footnote{\textit{La logique, ou l'art de penser} написана от Antoine Arnauld и Pierre Nicole.} 

Докато се опитвах да подобря езика си попаднах на английска граматика (мисля, че беше тази на Гринуд), в края на която имаше два малки откъса илюстриращи двете изкуства реторика и логика, като вторият завършваше с пример за спор по Сократовия метод. Малко след това придобих „Спомени за Сократ“\footnote{\textit{Memorabilia}} на Ксенофонт , където се намираха много примери на същия метод. Методът ме плени, възприех го, спрях с директните си противоречия и положителната аргументация, и се превъплътих в смирен, задаващ въпроси и съмняващ се човек. И тъй като по това време от четене на Шафтсбъри и Колинс наистина бях почнал да се съмнявам в някои точки от релегиозната ни доктрина, този метод ми се стори най-безопасен за мен самия и много притеснителен за тези, срещу които го използвах. Затова ми доставяше голямо удоволствие, използвах го постоянно, и станах експерт и много умел в това да подвеждам дори хора превъзхождащи ме по знания да направят признания, чиито последствия не можеха да предвидят, и които ги оплитаха в трудности, от които не можеха да се измъкнат, и по този начин печелех победи, които нито аз, нито каузата ми заслужавахме. Продължих с този метод няколко години, но постепенно го оставих, като запазих само навика да се изразявам със скромна неувереност и никога да не използвам думи като със сигурност, без съмнение, или други, които създават впечатление за утвърдително мнение, когато предлагам неща, които могат да бъдат оспорени, а вместо тях да казвам, струва ми се, или на мнение съм, че нещо е така и така; намирам че, или мисля, че е така и така, поради тази и тази причина; или въобразявам си че е така, или, така е, ако не се лъжа. Този навик, вярвам, ми е бил много полезен, при случаи, когато е трябвало да налагам мненията си и да убеждавам хора в мерки, които съм подкрепял по едно или друго време. И тъй като основните цели, към които човек се стреми в разговор, са да информира или да бъде информиран, да достави удоволствие или да убеди, ми се ще разумните и добронамерени хора да не намаляват способността си за добри дела като възприемат утвърдителен, безпрекословен маниер, който рядко успява да не отврати, има склнонността да предизвиква съпротива, и напълно се проваля в постигането на всички цели на разговора, т.е. даването и получаването на информация или удоволствие. Защото, ако искаш да информираш, утвърдителен или догматичен начин на изразяване може да предизвика противоречия и да възпрепятства искреното внимание. Ако искаш да бъдеш информиран и подобряван от чуждото знание, и все пак се изразяваш като човек здраво придъжащ се към настоящото си мнение, скромните и разумни хора, които не обичат спорове, най-вероятно ще те оставят да живееш необезпокояван в заблудата си. И по този начин рядко ще можеш да доставиш удоволствие на слушателите си или да убедиш тези, чиято подкрепа ти е нужна. Поуп мъдро отбелязва

	“Учи хората, като че не го правиш,
	и им предлагай нещата, които те не знаят, като че са забравени“

и още препоръчва: 

	„Да говориш, макар и сигурен, привидно с неувереност“ 

И би могъл да съчетае с този стих един, който е съчетал с друг, според мен, по-несполучливо,

	„Защото липсата на скромност е липса на разум“

Ако попиташ, защо по-несполучливо? трябва да повторя стиховете,

	“Нескромните слова нямат оправдание,
	защото липсата на скромност е липса на разум“

Но не е ли липсата на разум (когато човек има нещастието да му липсва) някакво оправдание за липсата му на скромност? И не биха ли били тези стихове по-добре сложени така?

	“Нескромните слова намират само едно оправдание,
	че липсата на скромност е липса на разум“

Това, обаче, трябва да предам на по-разумните за преценка.

През 1720 или 1721 брат ми започна да печата вестник. Беше вторият вестник, който се появи в Америка, и се наричаше \textit{New England Courant}.  Единственият преди него беше \textit{Boston News Letter}. Спомням си как някои от приятелите му се опитваха да го разубедят от предприемане на начинанието, защото им се струваше, че няма да успее, тъй като един вестник беше – според тях – достатъчен за Америка. В момента (1771) има не по-малко от двадесет и пет. Той обаче продължи с начинанието и след като приключи работата по съставяне на колите и отпечатването, бях изпратен да разнасям вестника на клиентите по улиците. 

Той имаше някои хитроумни приятели, които се забавляваха да пишат някои кратки писания за вестника му, което спечели доврието на читателите и направи вестника по-търсен, и тези господа често ни посещаваха. Разговорите им и разказите им за одобрението, с което творбите им биваха посрещани, ме накараха да желая и аз да си опитам перото сред тях. Но понеже бях още момче, и подозирах, че брат ми ще откаже да публикува нещо мое във вестника си, ако знае, че е мое, реших да си преправя почерка и, след като написах анонимна статия, я пъхнах под вратата на печатницата през нощта. Тя беше намерена на сутринта и споделена с пишещите му приятели на тяхната редовна сбирка. Те я прочетоха и обсъдиха в мое присъствие, и имах голямото удоволствие да установя, че заслужава тяхното одобрение, и че в опитите си да познаят кой е авторът споменаха само мъже с име сред нас за образованост и хитроумие. Сега предполагам, че ми излезе късмета с тези съдници, и че може би не са били толкова добри, колкото ми се струваха.

Все пак, насърчен от това, написах и предадох по същия начин до печатната преса още статии, които също така бяха одобрени. Пазих тайната си докато малкото ми усет за такива изпълнения не се поизчерпа, и я открих, когато познатите на брат ми започнаха да ме смятат за нещо повече, по начин, от който той не беше доволен, тъй като – може би справедливо – си мислеше, че има склонност да поощрява суетата ми. И може би това стана една от причините за различията, които започнаха да се появяват между нас, приблизително по това време.  Въпреки че ми беше брат, той се смяташе за мой майстор, а мен за негов чирак, и – както подобава – очакваше от мен същата работа, каквато очакваше от другите, а пък аз мислех, че твърде много ме унизява с някои от нещата, които изискваше от мен, понеже очаквах от него, като от брат, по-добро отношение. Споровете ни често бяха представяни пред баща ми, и ми се струва, че или като цяло бях прав, или пък по-добре пледирах, защото присъдата обикновено беше в моя полза. Но брат ми беше избухлив и често ме беше бил, което аз приех изключително тежко. И смятайки чиракуването за голямо тегло, постоянно се надявах да намеря възможност да го съкратя, и такава се появи след време по неочакван начинfootnote{Струва ми се, че грубото му и тиранично отношение към мен, може да е послужило като средство за насаждане в мен на това отвращение от безразборната власт, което ме съпровожда цял живот.}.

Някакво писание в нашия вестник на политическа тема, която вече съм забравил, засегна Събранието. Той беше привикан, осъден и затворен за месец, предполагам заради говорителя, защото брат ми не пожела да разкрие самоличността на автора. Мен също ме привикаха и разпитаха, но въпреки че отговорите ми не ги задоволиха, те сметнаха за достатъчно да ме порицаят и ме освободиха, може би смятайки, че като чирак съм задължен да пазя тайните на майстора си.

Докато изтече братовата ми присъда – която доста ме огорчи и ядоса, въпреки нашите различия – ръководих вестника и се осмелих да понатрия чрез него носовете на някои от нашите управници, което брати ми прие много добре, докато други започнаха да гледат на мен в лоша светлина, като млад гений, който има склонност към клевети и сатира. Освобождаването на брат ми бе съпътствано от заповед на Събранието (доста странна), „Джеймс Франклин да спре да издава вестника наречен \textit{New England Courant}.“

В печатницата ни се проведе съвещание с приятелите му какви действия да предприеме в този случай. Някои предложиха да се заобиколи заповедта чрез смяна на името на вестника, но виждайки недостатъци в това решение, брат ми в крайна сметка се спря на решението да го печата в бъдеще от името на Бенджамин Франклин, като по-добро. И за да заобиколи забраната на Събранието, която все още би могла да го засегне, ако печата чрез чирака си, според схемата ни чирашкият ми договор ми беше върнат с пълно освобождаване на гърба, за да бъде показван при нужда, но за да си осигури труда ми, трябваше да подпиша нов договор за остатъка от времето ми, който да остане скрит. Планът ни беше доста крехък, но пък го изпълнихме незабавно, и съответно вестникът продължи под мое име за няколко месеца. 

След време между мен и брат ми се породиха нови несъгласия и аз реших да наложа свободата си, като предположих, че той няма да рискува извади наяве новите договори. Не беше честно от моя страна да се възползвам от това предимство, и по тази причина считам това за една от първите грешки в живота ми. Но несправедливостта беше от малко значение за мен, когато бях под влияние на горчивината и яда от ударите, които поривите му често го караха да стоварва върху мен, макар и иначе той да не бе лош човек: може би аз се държах твърде непочтително и предизвикателно.

Когато откри, че искам да го напусна, се погрижи да предотврати наемането ми на работа във всички други печатници в града като ги обиколи и говори с всички майстори, които съответно отказаха да ми дадат работа. След това реших да отида в Ню Йорк, като най-близкото място с печатар. И бях доста склонен да напусна Бостън след като осъзнах, че вече съм се опротивил на властта, и – съдейки по безразборните действия на Събранието в случая с брат ми – че мога, ако остана, скоро да си докарам неприятности; и освен това, че недискретните ми спорове на тема религия започваха да карат добрите хора да ме сочат с ужас като неверник или атеист. Взех решение, но понеже баща ми този път взе страната на брат ми, съзнавах, че ако опитам да си тръгна открито, ще бъдат използвани средства, за да ме спрат. Затова приятелят ми Колинс се зае с извършването на някои малки приготовления за мен. Осигури съгласието на капитана на нюйоркски платноход да ме прекара, като ме представи за млад негов познат, от когото лошо момиче е забременяло, и който сега ще бъде принуден от приятелите ѝ да се ожени за нея, и затова не може да се появи или дойде открито. И така продадох някои от книгите си, за да събера малко пари, качих се скришно на борда, и тъй като имахме добър вятър, три дни по-късно се озовах в Ню Йорк, на 300 мили от дома, момче на едва 17 без каквато и да е препоръка към или знание за който и да е в града, и с много малко пари в джоба. 

Дотогава увлечението ми по морето беше преминало, иначе бих могъл да го задоволя. Но имайки занаят и смятайки се за доста добър работник, предложих услугите си на печатаря в града, стария Г-н Уилям Брадфърд, който бил първият печатар в Пенсилвания, но се преместил насам при свадата около Джордж Кийт. Той не можа да ме наеме, тъй като нямаше много работа, а имаше достатъчно помощници. Но казва той: "Синът ми във Филаделфия скоро загуби основния си помощник – Акила Роуз – поради смърт. Ако отидеш натам, вярвам, че може да те наеме."
Филаделфия беше сто мили по-нататък. Аз обаче се качих в лодка към Амбой, като оставих сандъка и нещата си да ме последват по море.

Докато прекосявахме залива ни застигна внезапна буря, която разпра гнилите ни платна, не ни позволи да влезем в устието на реката\footnote{the Kill, виж английската уики.} и ни притисна към Лонг Айлънд. По пътя един от пътниците, пиян Холандец, падна през борда. Докато потъваше, се пресегнах през водата към косматото му теме и го издърпах, така че отново го качихме. Топването му го отрезви малко и той заспа, като преди това извади от джоба си книга, която искаше да му изсуша. Оказа се моят стар любим автор, „Напредъкът на поклонника“ на Бъниън, на холандски, отпечатана с малки букви на добра хартия, с медни гравюри, премяна, по-добра от всички, които бях виждал да носи на родния си език. Оттогава научих, че е преведена на повечето европейски езици, и предполагам, че е по-четена от която и да е друга книга, освен може би Библията. Доколкото знам, честнинят Джон е първият човек, който смесва разказ с диалог, начин на писане много увлекателен за читателя, който – в най-интересните части – се намира като че ли в компанията и присъства на разговора. Дефо, в своите „Крузо“, „Мол Фландърс“, „Религиозно ухажване“\footnote{\textit{Religious Courtship}}, “Семеен инструктор“ \footnote{\textit{Family Instructor}} и други, успешно го имитира. Ричардсън прави същото в своята „Памела“.

Когато наближихме острова, установихме, че няма как да слезем, поради големия прибой на скалистия бряг. Затова пуснахме котва и се залюляхме към брега. Хора слязоха до водата и ни викнаха, както и ние на тях. Но вятърът беше толкова силен и прибоят толкова шумен, че не можехме да чуем достатъчно, за да се разберем. На брега имаше канута, и ние направихме знаци, и им викнахме, че трябва да ни вземат. Но те или не ни разбраха, или помислиха, че е невъзможно, така че си тръгнаха, и с настъпването на нощта ние нямахме друг изход освен да чакаме вятърът да спадне. И междувременно лодкарят и аз решихме да спим, ако можем, и затова се натъпкахме в трюма при холандеца, който още беше мокър, а пръските, които прехвърляха борда на лодката протичаха при нас, така че скоро бяхме почти толкова мокри, колкото той. По този начин лежахме цяла нощ, с много малко почивка. Но, с падането на вятъра на другия ден, опитахме да стигнем до Амбой преди нощта, след като бяхме прекарали тридесет часа на вода, без продоволствия и каквато и да е напитка освен една бутилка долнопробен ром, и при положение, че водите, в които плувахме бяха солени. 

Вечерта се намерих със силна треска и си легнах. Но понеже бях чел някъде, че обилното пиене на студена вода помага при треска, последвах предписанието, потих се обилно през по-голямата част от нощта, треската ме остави, и на сутринта, след като преминах с ферибота, продължих пътуването си пеш, имайки още петдесет мили до Бърлингтън, където – беше ми казано – щях да намеря лодки, с които да премина остатъка от пътя до Филаделфия. 

Цял ден валя много силно. Бях напълно подгизнал, а до обед и доста уморен. Затова се спрях в една бедна страноприемица, където останах цяла нощ, и сега започваше да ми се иска никога да не бях напускал дома. Толкова окаяно изглеждах, че по въпросите, които ми задаваха, установих, че подозират, че съм избягал слуга, и има опасност да бъда задържан по това подозрение. Въпреки това на другия ден продължих, и вечерта стигнах до една страноприемница – стопанисвана от някои си Д-р Браун – на  осем или десет мили от Бърлингтън. Той ме заговори докато се подкрепях и щом разбра, че съм чел малко, стана много общителен и дружелюбен. Приятелството ни продължи докато беше жив. По-рано ще да е бил странстващ доктор, защото нямаше град в Англия или държава в Европа, които не можеше да опише подробно. Беше учил малко, беше хитроумен, но голям неверник, и няколко години по-късно нечестиво захвана да пародира Библията в долнопробни стихове, както Котън бе направил с Вергилий. По този начин той представи много от фактите в нелепа светлина, и ако работата му беше публикувана можеше да нарани някои слаби умове, но това никога не се случи. 

Спах в неговата къща тази нощ, а на другата сутрин стигнах до Бърлингтън, но с голямо огорчение трябваше да науча че всички редовни лодки са тръгнали малко преди да дойда и не се очаква да тръгват други преди вторник, а денят беше събота. Затова се върнах при една стара жена в града, която ми беше продала джинджифилов сладкиш за из път, и я помолих за съвет. Тя ми предложи да квартирувам в къщата й докато се намери възможност да премина по вода, и – уморен от пътуване на крак – аз приех поканата. Когато разбра, че съм печатар, ми предложи да се установя в града и да отворя печатница, не знаейки какъв капитал е нужен за това. Беше много гостоприемна, и за вечеря ми даде волска буза с голяма доза доброжелание, като прие само една кана с бира в замяна, и аз си мислех че съм уреден до вторник. Докато се разхождах вечерта покрай реката, обаче, пристигна лодка с няколко човека, която – установих – пътува за Филаделфия. Взеха ме на борда и, тъй като нямаше вятър, гребахме по целия път. И когато към полунощ още не бяхме видели града, някои от компанията бяха сигурни, че трябва да сме го подминали, и отказаха да гребат повече. Останалите не знаеха къде сме. Затова се насочихме към брега, влязохме в един приток и слязохме на брега до една стара ограда, с чийто пръти си накладохме огън, тъй като беше октомври и нощта беше студена, и останахме там до съмване. Тогава някой от компанията установи, че мястото е Купър Крийк, малко нагоре от Филаделфия, която видяхме щом излязохме от притока и пристигнахме в осем или девет часа в Неделя сутринта, и слязохме на кея на Търговската улица.

Описах пътуването си с по-големи подробноси и ще направя същото с първото ми влизане в този град, за да можеш да сравниш наум това невероятно начало с името, което оттогава съм изградил там. Бях в работно облекло, тъй като най-добрите ми дрехи тепърва трябваше да пристигнат по море. Бях мръсен от пътуването. Джобовете ми бяха пълни с ризи и чорапи и не познавах жива душа, нито къде да потърся квартира. Бях уморен от път, гребане и липса на почивка, бях и много гладен, а всичките ми парични запаси възлизаха на холандски долар и около един шилинг на дребно. Последния го дадох на хората от лодката за пътуването. Те първо отказаха, понеже бях гребал, но аз настоях да го вземат. Човек понякога може би е по-щедър когато има малко пари, отколкото когато има много, може би от страх, да не помислят, че има малко.

След това тръгнах нагоре по улицата оглеждайки се наоколо, докато близо до пазара не срещнах момче с хляб. Много пъти вече бях се хранил с хляб, и след като го попитах откъде го е взел, веднага отидох до хлебаря, към който ме насочи – на Втора улица  – и поисках бисквита от тези, които имахме в Бостън. Но, изглежда, във Филаделфия не правеха такива. Тогава поисках самун за три пенса, и ми бе казано, че няма такъв. Затова – без да мисля за или да съзнавам разликата в стойността на парите и по-ниските цени, и без да знам имената на хлябовете му – му казах да ми даде каквото и да е за три пенса. Съответно той ми даде три големи пухкави рола. Учидих се на куличеството, но ги взех, и понеже нямах джобове тръгнах с по едно руло под всяка мишница и ядейки третото. Така продължих нагоре по Търговската улица до Четвърта улица, и минах пред вратата на Г-н Рийд, бащата на бъдещата ми жена, когато тя, стоейки на вратата, ме видя и си помисли – съвсем правилно – че изглеждам най-непохватно и нелепо. След това завих и продължих надолу по \textit{Chestnut Street} и част от \textit{Walnut Street}, ядейки рулото през цялото време, и завършвайки обиколката пак се намерих на кея на Търговската улица, близо до лодката с която пристигнах, и при която отидох за глътка речна вода. И тъй като първото руло ме беше заситило дадох другите две на една жена, която се беше спуснала по реката с нашата лодка и чакаше да продължи пътя, и детето ѝ.

И така освежен отново тръгнах нагоре по улицата, която по това време вече беше пълна с чисто облечени хора, които всички отиваха в една посока. Аз се присъединих към тях и по този начин бях отведен в голямата сграда за срещи на квакерите близо до пазара. Седнах сред тях, и след като се пооглеждах наоколо известно верме и не чух някой да говори, понеже бях много сънен от работа и липса на почивка предишаната нощ, заспах, и спах до края на събранието, когато някой бе така добър да ме събуди. Така че това беше първата къща, в която влязох или спах във Филаделфия.  

Докато се връщах надолу към реката и се вглеждах в лицата на хората срещнах млад квакер чието изражение ми хареса, и го заговорих и попитах къде може да отседне странник. Бяхме близо до знака на „Тримата моряци“. „Тук,“ казва той, „е място, което приема странници, но няма добро име. Ако дойдеш с мен ще ти покажа по-добро.“ Заведе ме до \textit{the Crooked Billet}\footnote{буквално: Кривата военна квартира} на \textit{Water Street}. Тук вечерях. И, докато ядях, няколко хитри въпроса ми бяха зададени, тъй като младостта и видът ми явно можеха да събудят подозрения, че съм беглец.

След вечеря сънливостта ми се върна и след като ме заведоха до леглото ми легнах без да се събличам и спах до шест вечерта, бях извикан за вечеря, легнах си пак много рано, и спах крепко до следващата сутрин. След това се спретнах доколкото можах и отидох до печатаря Андрю Брадфърд. В печатницата заварих стария му баща, когото бях видял в Ню Йорк, и който беше стигнал във Филаделфия преди мен, пътувайки на кон. Той ме представи на сина си, който ме прие учтиво, даде ми закуска, но ми каза, че в момента не му трябва помощник, тъй като скоро си е намерил; но че в града има друг печатар, който започнал наскоро, някакъв Каймер, който може би ще иска да ме наеме; ако ли не, бих бил добре дошъл да остана при него,  той ще ми дава малко работа от време навреме докато не се намери повече заетост.

Старият джентълмен каза, че ще отиде с мен при новия печатар. Когато го намерихме, той казва: „Съседе, доведох да те види млад човек от твоя занаят. Може би ти липсва такъв.“ Той ми зададе няколко въпроса, даде ми компас\footnote{печатарски, за съставяне на текст} да види как работя и след това каза, че ще ме наеме скоро, въпреки че точно в този момент няма никаква работа за мен. И, взимайки старият Брадфърд, когото никога преди това не беше виждал, за човек от града добронамерен към него, захвана разговор за настоящите си занимания и проекти. А Брадфърд, без да открива, че е баща на другия печатар, като чу Каймер да казва, че очаква скоро да поеме по-голямата част от бизнеса в свои ръце, го поведе с умели въпроси и посявайки съмнения да обясни всичките си виждания, на какво разчита, и по какъв начин смята да действа. Стоейки отстрани веднага видях, че единият беше сръчен стар софист, а другият само новак. Брадфърд ме остави с Каймер, който много се изненада, когато му казах, с кого беше разговарял.

Установих, че печатницата на Каймер се състои от стара, разбита печатарска машина, и един малък, износен английски шрифт, който в онзи момент самият той използваше, композирайки Елегия за Акила Роуз, вече споменат, изобретателен млад човек, с превъзходен характер, много уважаван в града, писар на събранието и приличен поет. Каймер също правеше стихове, но много безучастно. Не може да се каже, че ги пишеше, тъй като имаше навика да ги излива от главата си направо в касетите. Тъй като нямаше копие, а само две кутии, а за елегията вероятно щяха да отидат всички букви, никой не можеше да му помогне. Опитах се да оправя печатарската му машина (която той още не бе използвал, и от която нищо не разбираше) и да я вкарам във вид готов за работа. И  след като му обещах да се върна и да отпечатам Елегията щом е готов, се върнах при Брадфърд, който ми даде малка задача за момента и там отседнах и се хранех. След няколко дни Каймер ме повика да отпечатам Елегията. Сега беше получил още две кутий и памфлет за препечатване, с който ми даде да се занимавам.

Намерих, че тези двама печатари са зле квалифицирани за работата си. Брадфърд не беше обучен и беше много неграмотен. А Каймер, макар и да бе малко учен, беше само съставител, и не разбираше нищо от печатарската работа. Беше един от Френските пророци и можеше да разиграва техните въодушевени изстъпления. По това време не изповядваше определена вяра, но по нещо от всичко според случая; знаеше много малко за света и – както разбрах по-късно – беше замесен с доста негодническо тесто. Не му харесваше, че съм отседнал при Брадфърд докато работя при него. Наистина той имаше къща, но без мебели, и затова не можеше да ме подслони. Но ми намери квартира при г-н Рийд, който беше споменат по-рано, и който беше собственикът на къщата му; и понеже сандъкът ми по това време вече беше пристигнал, направих доста по-достойно представяне в очите на г-ца Рийд, отколкото бях направил първия път, когато ме видя да ям рулото си на улицата. 

По това време започнах да се запознавам с младите хора в града, които обичаха да четат, и с които прекарвах вечерите си много приятно; и понеже с трудолюбието и пестеливостта си печелех пари, живеех доста добре, не мислех за Бостън доколкото можех, и не желаех никой там да знае къде пребивавам освен приятеля ми Колинс, който знаеше моята тайна и я запази след като му писах. В крайна сметка се случи произшествие, което ме върна там доста по-рано отколкото имах намерение. Имах зет  Робърт Холмс, капитан на лодка, която търгуваше между Бостън и Делауеър. Той чул за мен в Нюкасъл, четиридесет мили надолу от Филаделфия, и ми писа писмо, в което споменаваше тревогата породена у приятелите ми в Бостън от внезапното ми отпътуване, уверяваше ме в добронамереността им, и че всичко ще се нареди според желанията ми, ако се върна, за което той ме увещаваше горещо. Писах му в отговор, че съм му благодарен за съветите му, но изложих изцяло причините си да напусна Бостън, и по такъв начин, че да го убедя, че не съм постъпил толкова погрешно, колкото той беше разбрал.

Сър Уилям Кийт, губернатор на провинцията, бил по това време в Нюкасъл, и понеже капитан Холмс бил в неговата компания, когато получил писмото ми, говорил с него за мен и му показал писмото. Губернаторът го прочел и изглеждал изненадан, когато му казали на каква възраст съм. Казал, че изглежда съм млад мъж с обещаващи дарби, и затова трябва да бъда насърчен; печатарите във Филаделфия били ужасни; и ако съм започнел бизнес там, той не се съмнявал, че съм щял да успея; той щял да ми осигури всички обществени поръчки, и щял да ме подкрепи с всяка друга услуга, която му е по силите. Тези неща ми каза зет ми по-късно в Бостън, но по това време не знаех нищо за тях; когато един ден докато Каймер и аз работехме до прозореца, видяхме губернатора и един друг джентълмен (който се оказа полковник Френч от Нюкасъл), добре облечени, да се приближават през улицата право към нашата къща, и ги чухме на вратата.

Каймер веднага изтича долу мислейки, че идват при него; но губернаторът попита за мен, качи се, и със снизходителна учтивост, с която не бях свикнал, ми направи много комплименти, пожела да се запознае с мен, нежно ме порица, че не съм му се представил щом съм дошъл в града, и ме покани да отида с него в кръчмата, накъдето се беше запътил с полковник Френч, за да опита, както той каза, една превъзходна мадейра. Аз бях немалко изненадан, а Каймер гледаше като прасе отровено. Въпреки това отидох с губернатора и полковник Френч до една кръчма, на ъгъла на Трета улица, и докато пиехме мадейрата той предложи да започна свой бизнес, изложи пред мен шансовете за успех и заедно с полковник Френч ме увери, че техните застъпничество и влияние ще са на моя страна за осигуряване на обществени поръчки от двете правителства. Щом изразих съмненията си, че баща ми би ме подкрепил в такова начинание, сър Уилям каза, че ще ми даде писмо до него, в което ще изложи предимствата, и нямаше съмнение, че ще успее да го убеди. Затова бе решено, че ще се върна в Бостън с първия кораб и препоръка до баща ми. Междувременно намерението трябваше да остане в тайна и продължих да работя при Каймер както обикновено, а губернаторът изпращаше отвреме на време покана да вечерям с него, което аз смятах за голяма чест, и разговаряше с мен по най-приветливия, фамилиарен и приятеслки начин, който човек може да си представи.

Към края на април 1724 се намери малък кораб, който щеше да плава до Бостън. Казах на Каймер, че отивам да видя приятелите си. Губернаторът ми даде предостатъчно писмо, в което казваше много ласкави неща за мен на баща ми и силно препоръчваше плана да започна собствен бизнес във Филаделфия, като нещо, което със сигурност ще ме направи богат. Докато плавахме надолу по залива ударихме плитчина, което предизвика теч; морето беше бурно и трябваше да помпаме почти през цялото време, в което и аз взех участие. Въпреки това, след около две седмици благополучно пристигнахме в Бостън. Бях отсъствал седем месеца и приятелите ми не бяха получили никакви вести от мен; зет ми Хомлс още не се беше върнал и не беше писал нищо за мен. Неочакваното ми появяване изненада семейството; въпреки това всички се радваха да ме видят и ме посрещнаха добре освен брат ми. Отидох да го посетя в печатницата му. Издокаран от глава до пети с изтънчен нов костюм, часовник и подплата от почти пет лири стерлинги в сребро в джобовете, бях по-добре облечен от когато и да е било по времето, когато работех за него. Той не ме посрещна радушно, огледа ме отгоре до долу и се върна към работата си.

Работниците ме разпитаха къде съм бил, каква страна е, и дали ми е харесала. Много я хвалих, както и щастливия живот, който водих в нея, и изразих силното си намерение да се завърна; а щом един от тях попита кави пари имаме там, извадих една шепа сребро и я разстлах пред тях, което беше нещо като рядко срещан спектакъл за тях, тъй като бостънските пари бяха хартиени. След това им дадох възможност да видят часовника ми; и накрая (брат ми все още мрачен и навъсен) им дадох едно петаче\footnote{a piece of eight} да се почерпят, и си тръгнах. Това мое посещение го обиди изключително много. Когато по-късно майка ми говорила с него за сдобряване и за нейните желания да ни види в добри отношения и да можем да живеем като братя в бъдеще, той казал, че съм го обидил по такъв начин пред хората му, че не може никога да забрави нито да прости. В това обаче той се лъжеше.

Баща ми беше видимо изненадан от писмото на губернатора, но през следващите няколко дни ми каза малко за него, докато капитан Холмс се върна и той му го показа и го попита дали познава Кийт и що за човек е; като добави, че според него трябва да е не много разумен, щом смята, че е добра идея да помогне на момче, което има още три години до пълнолетие да започне свой бизнес. Холмс каза каквото можа в подкрепа на проекта, но баща ми ясно изрази, че го смята за непристоен, и в крайна сметка твърдо отказа. След това написа учтиво писмо до сър Уилям, в което му благодарше  за подкрепата, която така любезно ми беше предложил, но отказваше да ми помогне със започването на собствен бизнес, тъй като – според него – бях твърде млад, за да ми бъде поверено управлението на толкова важна работа, подготовката за която ще да е много скъпа. 

Моят приятел и спътник Колинс, който работеше като писар в пощата, хареса разказа  ми за новата ми страна и  реши също да отиде там. И докато чаках решението на баща ми, той тръгна преди мен по суша за Роуд Айлънд като остави книгите си, които бяха хубава колекция по математика и естествена философия, да пътуват заедно с моите и с мен до Ню Йорк, където той предложи да ме чака. 

Въпреки че баща ми не одобри предложението на сър Уилям, все пак беше доволен, че съм успял да създам такова благоприятно мнение у такъв изтъкнат човек там, където пребивавах, и че съм бил толкова усърден и внимателен, че да успея да се обзаведа така хубаво за толкова кратко време; и понеже не виждаше вероятност за постигане на съгласие между мен и брат ми, ми даде съгласието си да се върна във Филаделфия, посъветва ме да се държа почтително с хората там, да се опитвам да спечеля общото одобрение, и да избягвам подигравателните писания и клеветите, към които той смяташе, че имам твърде голяма склонност; каза ми, че със постоянно усърдие и разумна пестеливост, докато навърша двадесет и една години бих могъл да спестя достатъчно, за да започна собствен бизнес; и че ако събера почти толкова, колкото е нужно, той ще ми помогне с остатъка. Това е всичко, което можах да получа, освен някои малки подаръци в знак на неговата и майчината ми любов, когато отново поех към Ню Йорк този път с тяхното одобрение и благословия.

Понеже платноходът спря в Нюпорт, Роуд Айлънд, посетих брат ми Джон, които от няколко години беше женен и установен там. Той ме прие много любящо, понеже винаги ме е обичал. Понеже имаше приятел, някакъв Върнън, на когото дължеше пари в Пенсилвания, около тридесет и пет паунда валута, поиска да ги получа от негово име и да ги пазя докато не получа инструкции как да му ги пратя. Даде ми и съответното нареждане. Това по-късно ми създаде доста тревоги.

В Нюпорт взехме няколко пътника за Ню Йорк, сред които бяха две млади жени, спътнички, и сериозна, здравомислеща квакерка с осанка на матрона и нейните слуги. Бях ѝ засвидетелствал любезната си готовност да ѝ правя разни малки услуги, което предполагам породи у нея известна добронамереност към мен; затова, когато видя, че между мен и двете млади жени се създава ежедневно растяща близост, която изглежда бе насърчавана от тях, тя ме дръпна настрана и ми каза: „Млади човече, загрижена съм за теб, тъй като нямаш приятел със себе си и изглежда не познаваш света добре, нито примките, които дебнат младостта; можеш да разчиташ на думата ми, тези жени са много лоши; мога да го видя във всичките им действия; и ако не си нащрек, ще те вкарат в някоя опасност; не ги познаваш, и те съветвам от приятелска загриженост за твоето добро, да не се сближаваш с тях.“ Тъй като първоначално видимо не споделих нейното лошо мнение за тях, тя спомена някои неща, които беше видяла и чула, които аз не бях забелязал, и с това ме убеди, че е права. Благодарих ѝ за милия съвет и ѝ обещах да го последвам. Когато пристигнахме в Ню Йорк, те ми казаха къде живеят и ме поканих да ида да ги видя, но аз не го направих, и добре че сторих така; на следващия ден капитанът откри, че от каютата му липсват сребърна лъжица и някои други неща, и знаейки, че тези са проститутки, получи съдебно разпореждане за претъсване на жилището им, намери откраднатото, и предаде крадлите за наказание. Затова, въпреки че се разминахме с подводен камък, в който се одрахме по време на пътуването, бях на мнение, че това избавление бе по-важно за мен.

В Ню Йорк намерих приятеля си Колинс, който беше пристигнал малко преди мен. Бяхме близки от деца и бяхмо прочели същите книги заедно; но негово преимущество бе, че имаше повече време да чете и учи, както и чудесна дарба да учи математика, в която далеч ме превъзхождаше. Докато живях в Бостън повечето ми свободни за разговор часове бяха прекарани с него и той напредваше като трезв и усърден момък; бе много уважаван поради знанията си от няколко от духовниците и други джентълмени, и изглежда обещаваше да върви добре в живота. Но докато ме нямаше беше придобил навика да се напива с бренди; и научих от собствения му разказ, и от това, което чух от други, че откакто е пристигнал в Ню Йорк се е напивал всеки ден и се е държал много странно. Беше играл комар и беше загубил парите си, така че трябваше да платя за квартирата му и разходите му до и във Филаделфия, което се оказа изключително неудобно за мен.

Щом чул от капитана, че един от пътниците, млад човек, има много книги, тогавашният губернатор на Ню Йорк, Бърнет (син на епископ Бърнет), поискал капитанът да ме доведе да се срещна с него. Съответно аз изпълних желанието му, и щях да взема и Колинс с мен, но той не беше трезвен. Губернаторът се отнесе с мен много учтиво, показа ми библиотеката си, която беше голяма, и доста говорихме за книги и автори. Това беше вторият губернатор, който ми оказа честта да ми обърне внимание; което за бедно момче като мен беше много приятно.

Продължихме към Филаделфия. По пътя получих парите на Върнън, без които трудно щяхме да завършим пътуването си. Колинс искаше да започне в една счетоводна фирма, но дали разбраха, че си пийва по дъха му или по поведението му, въпреки че имаше някакви препоръки, не успя никъде да си намери работа и продължи да живее и да се храни в същата къща, в която живеех и аз, и за моя сметка. Понеже знаеше, че имам парите на Върнън, постоянно взимаше назаем от мен, като все обещаваше, че ще ми плати щом намери работа. Накрая зае такава част от парите, че почна да ме тревожи мисълта какво ще правя, ако ми ги поискат. 

Пиянството му продължи, заради което понякога се карахме; защото когато беше леко пиян беше много раздразнителен. Веднъж когато пътувахме към Делауеър в лодка с няколко други млади мъже, той отказа да гребе на свой ред. „Ще бъда отгребан до вкъщи“, казва той. „Ние няма да те отгребем“, казвам аз. „Ще трябва, или оставате цяла нощ в реката“, казва той, „както ви харесва.“ Другите казаха, „Нека да гребем, какво значение има?“ Но понеже умът ми беше раздразнен от другото му поведение, аз продължих да отказвам. Така че той се закле, че ще ме накара да греба или ще ме хвърли през борда; и ме приближи стъпвайки по седалките за гребците, а когато дойде до мен и ме удари, аз го подхванах с ръка под чатала и като се изправих го хвърлих с главата напред в реката. Знаех, че е добър плувец, затова  не се тревожех за него; но преди да успее да хване лодката, с няколко удара я издърпахме извън обхвата му; и винаги когато приближеше до лодката, ние го питахме дали ще гребе и с няколко удара я плъзвахме далеч от него. Той беше готов да пукне от яд и инатливо отказваше да обещае да гребе. Но щом накрая видяхме, че почва да се уморява, го вдигнахме и го прибрахме у дома мокър до кости тази вечер. Почти не разменихме любезна дума след това, а един карибски капитан, който беше натоварен със задачата да намери учител за синовете на джентълмен в Барбадос, случайно се срещна с него и се съгласи да го откара до там. Той ме напусна тогава и обеща да ми преведе първите пари, които ще получи, за да си уреди дълга; но никога повече не получих вест от него.

Използването на парите на Върнън беше една от първите големи грешки в живота ми; и този случай показа, че баща ми не беше сгрешил с предположението си, че съм твърде млад да управлявам важни дела. Сър Уилям, обаче, щом прочете писмото му, каза, че е твърде разумен. Имало голяма разлика между хората; и умът не винаги придружавал годините, нито пък младостта винаги била без него. „И тъй като той няма да те подкрепи в започването на бизнес“, казва той, „аз ще го направя. Дай ми списък с нещата, които трябва да се купят от Англия, и ще изпратя да ги купят. Ще ми се отплатиш, когато можеш; решен съм да уредя добър печатар тук, и съм сигурен, че ще сполучиш.“ Това бе казано с такова сърдечно изражение, че нямах ни най-малко съмнение, че той говори сериозно. Дотогава държах в тайна идеята да започна собствен бизнес във Филаделфия, и продължих да го правя. Ако се знаеше, че завися от губернатора, вероятно някой приятел, който го познаваше по-добре, щеше да ме посъветва да не разчитам на него, тъй като по-късно научих, че бил известен със склонността си да прави големи обещания, които нямал никакво намерение да спази. При все това, след като не аз го бях потърсил за това, как можех да предположа, че щедрите му предложения са неискрени? Вярвах, че е един от най-добрите хора на света.

Представих му инвентара на малка печатница, който по моите изчисления възлизаше на около сто паунда стерлинги. Той го хареса, но ме попита дали няма да е добре да отида лично в Англия да избера шрифтовете и да се уверя, че всичката стока е добра. „Така,“ казва той, „когато си там, ще можеш да се запознаеш с хора и да установиш връзки с продавачите на книги и канцеларски материали.“ Съгласих се, че това може да е предимство. „Тогава,“ казва той, „приготви се да заминеш с Анис;“ което беше ежегодния кораб и единствения по онова време, който обикновено пътуваше от Лондон до Филаделфия. Но до отплаването на Анис оставаха няколко месеца, затова продължих да работя с Каймер, да се тревожа за парите, които Колинс взе от мен, и в ежедневни опасения да не ме потърси Върнън, което обаче се случи едва след няколко години.

Вярвам, че пропуснах да спомена, че при първото ми пътуване от Бостън се случи да попаднем в безветрие до остров Блок, и нашите хора почнаха да ловят треска, и хванаха много. Дотогава се бях придържал към решението си да не ям храна с животински произход, и в този случай гледах – следвайки учителя си Трайън – на улавянето на всяка риба като на непредизвикано убийство, тъй като нито една от тях не ни беше сторила – нито някога би могла да ни стори – каквато и да е злина, която да оправдае клането. Всичко това звучеше много разумно. По-рано обаче бях голям любител на рибата, а когато вадеха тези риби горещи от тигана, миришеха възхититлено добре. Известно време запазих равновеси между принцип и склонност, докато не си спомних, че когато кормят рибата, съм виждал да вадят по-малки рибки от стомаха й; помислих тогава, „Ако вие се ядете една друга, не виждам защо ние да не можем да ви ядем.“ Така че вечерях много добре с треска и продължих да се храня като другите хора, като се връщах само отвреме навреме към зеленчуковата си диета. Много е удобно, че човек е разумно същество, защото това му позволява да измисли причина за всяко нещо, което му се иска да направи.

Каймер и аз живеехме в доста добри, близки отношения, и се разбирахме порядъчно добре, защото той не подозираше нищо за започването на бизнеса ми. Той пазеше голяма част от стария си ентусиазъм и обичаше да спори. По тази причина имахме много диспути. Обработвах го със Сократовия метод и толкова често го бях подлъгвал с въпроси привидно много отдалечени от всичко, което объждахме, но всъщност постепенно водещи до темата, и го бях вкарвал в трудности и противоречия, че накрая стана смехотворно предпазлив и едва ми отговаряше дори на най-обикновените въпроси без да попита първо, „Какво искаш да заключиш от това?“ Но това го накара да придобие такова добро мнение за способностите ми да опровергавам, че сериозно ми предложи да му стана съдружник в един негов проект за започване на нова секта. Той щеше да проповядва ученията, а аз щях да оборвам всички противници. Когато започна да ми обяснява ученията, подигнах възражения по няколко трудни въпроса, за да мога да се наложа и да прокарам някои от собствените си доктрини.

Каймер не си стрижеше брадата, защото някъде в Моисеевия закон пише „не разваляйте края на брадата си “. Също така, той пазеше Седмия ден, Съботата; и тези две учения бяха основни за него. На мен и двете не ми харесваха; но се съгласих да ги приема при условие, че той се съгласи с моето за неяденето на харана от животински произход. „Подозирам“, каза той, „че организмът ми няма да го понесе.“ Уверих го, че ще го понесе, и че ще се чувства по-добре. Той обикновено беше голям лакомник и си обещах малко да се позабавлявам почти уморявайки го от глад. Той се съгласи да опита, ако му правя компания. Аз го направих и продължихме така три месеца. Храната ни беше приготвяна редовно от една жена в квартала, на която бях дал списък с четиридесет различни ястия, които да ни готви по различно верме, в които нямаше нито риба, нито месо, нито птица, а тази приумица още повече ми допадна в онзи момент, защото излизаше евтино, като ни струваше не повече от осемнадесет пенса стерлинги на седмица.  Оттогава съм изкарал няколко Великденски поста съвсем стриктно, като минавах от обиконовената диета на тази и обратно внезапно без каквито и да е проблеми, затова мисля, че съветът да се сменя диетата постепенно не струва много. Аз си изкарвах приятно, но клетия Каймер страда много, измори се от проекта, закопня за египетските казани с месо, и поръча печено прасе. Покани мен и две приятелки да вечеряме с него; но понеже донесоха прасето на масата твърде рано, той не можа да се сдържи и го изяде цялото преди да дойдем. 

По това време бях ухажвал г-ца Рийд известно време. Много я уважавах и обичах и имах известни причини да вярвам, че тя изпитва същите чувства спрямо мен; но, тъй като щях да поемам на дълго пътешествие, и тъй като бяхме доста млади – едва навършили осемнадесет -  майка й намери за най-разумно в този момента да ни попречи да стигнем  твърде далеч, тъй като, ако щяхме да се женим, най-добре щеше да е това да стане след като се върна, когато би трябвало според моите очаквания да се установя в работата си. Освен това, тя може би смяташе моите очаквания не толкова оправдани, колкото на мен ми се струваха.

Основните ми познанства по това време бяха Чарлз Озбърн, Джоузеф Уотсън и Джеймс Ралф, всичките любители на четенето. Първите двама бяха чиновници при изтъкнат нотариус или адвокат по недвижими имоти в града, Чарлз Брогден; другият беше чиновник при един търговец. Уотсън беше набожен, разумен млад мъж, много честен; другите бяха по-отпуснати в религиозните си убеждения, особено Ралф, когото  – като Колинс –  бях притеснил, за което и двамата ме накараха да страдам. Озбърн беше разумен, прям, честен и нежен с приятелите си; но по литературни въпроси твърде склонен да критикува. Ралф беше изобретателен, изтънчен в обноските си и изключително красноречив; мисля, че никога не съм срещал човек, който говори по-добре. И двамата бяха големи почитатели на поезията и започваха да си опитват перото в разни малки произведения. Четиримата много пъти си правихме приятни неделни разходки в гората близо до река Скулкил\footnote{Schuylkill}, където четяхме един на друг и обсъждахме, каквото бяхме прочели.

Ралф имаше склонност към изучаването на поезия, не се съмняваше, че може да стане изтъкнат поет и да си изкарва хляба с нея, и твърдеше, че в началото на писателската си кариера най-добрите поети трябва да правят толкова грешки, колкото той правеше. Озбърн го разубеждаваше и го уверяваше, че няма гений за поезия, и го съветваше да не мисли за нищо повече от занаята, за който е бил обучен; и че в търговията, въпреки че няма капитал, би могъл с усърдие и точност да се издигне в кариерата, и с времето да събере едно-друго, с което да започне собствена търговия. Аз намирах за добре човек да се забавлява с поезия отвреме на време, колкото да си подобри езика, но не повече.

По време на една от нашите разходки се предложи всеки от нас да напише сам стихотворение за следващата ни среща, за да станем по-добри чрез взаимните си забележки, критики и поправки. Тъй като искахме да се съсредоточим върху езика и изразителността изключихме всички творчески съобръжения, като се съгласихме, че задачата ще е преработка на Псалом 18 (17 по православному), който описва слизането на Божество. Когато времето за срещата ни наближи Ралф ме потърси и ми каза, че е готов. Аз му казах, че съм бил зает и понеже не съм имал голямо желание не съм успял да напиша нищо. Той ми показа произведението си за оценка, и аз силно го одобрих, защото ми се стори, че има големи достойнства. „Сега“, казва той, „Озбърн никога няма да признае никакво достойнство в нещо мое и ще отправи 1000 критики просто от завист. На теб не ти завижда толкова; затова искам да вземеш стиховете ми и да ги представиш като свои; аз ще се престоря, че не съм имал време и няма да представя нищо. Тогава ще видим какво ще каже за тях.“ Съгласихме се, а аз моментално ги преписах, за да са написани с моя почерк. 

Събрахме се; произведението на Уотсън бе прочетено; имаше някои хубави неща в него, но много недостатъци. Това на Озбърн беше прочетено; беше много по-добре; Ралф го оцени справедливо; отеляза някои грешки, но похвали хубавите неща. Той  самият не беше написал нищо. Аз бях неохотен; изглежда имах желание да ме извинят; не бях имал достатъчно време за поправки, и т.н.; но никакво извинение не можеше да бъде прието; трябва да чета! Беше прочетено и повторено; Уотсън и Озбърн се отказаха от съревнованието, и заедно започнаха да го хвалят. Само Ралф отправи някои критики и предложи някои промени; но аз защитих текста си. Озбърн беше срещу Ралф и каза, че е не по-добър критик отколкото поет, така че той се отказа от спора. Докато се прибирали заедно, Озбърн още по-силно одобрил това, което мислеше, че е мое произведение; по-рано се въздържал, за да не помисля, че ме ласкае, според неговите думи. „Кой би помислил“, казал той, „че Франклин е способен на такова представяне; такава картина, такава сила, такъв плам! Даже е подобрил оригинала. В ежедневните си разговори изглежда, че няма богат речник; колебае се и прави грешки; и въпреки това, мили Боже! как пише!“ При следващото ни събиране Ралф разкри шегата, която му бяхме изиграли, и Озбърн беше леко осмян.

Тази работа утвърди решението на Ралф да стане поет. Направих всичко, което можах, за да го разубедя, но той продължи да драска стихове, докато Поуп не го излекува. Но се научи да пише проза доста добре. Повече за него по-нататък. Но понеже може би няма да има случай да спомена пак другите двама, само ще отбележа тук, че Уотсън умря в ръцете ми няколко години по-късно, и беше много оплакван, тъй като беше най-добрият от групата ни. Озбърн замина за Западните Индии, където стана известен адвокат и печелеше добри пари, но умря млад. Двамата с него направихме сериозна уговорка: който от двамата умре пръв, ако може, да направи приятелско посещение на другия, и да му каже как намира нещата в това друго състояние. Той така и не изпъни обещанието си.

Губернаторът, който изглежда харесваше компанията ми, често ме канеше у дома си, и това, че ще ми помогне да започна бизнес винаги се споменаваше като уговорено нещо. Щях да взема препоръчителни писма до негови приятели, в допълнение към акредитивно писмо\footnote{letter of credit}, което ще ми осигури парите за закупуване на печатарска преса, букви, хартия, и т.н. За тези писма трябваше да дойда в различни моменти, когато ще са готови, но в бъдеще време, което предстоеше да бъде определено. Нещата продължиха така до момента, в който корабът, чието отпътуване беше няколко пъти отлагано, щеше да отплава. Когато отидох да се сбогувам и да получа писмата, секретарят му, Д-р Бард, излезе и каза, че губернаторът е изключително зает с писане, но че ще бъде в Нюкасъл (Делауер) преди кораба, и че писмата ще ми бъдат предадени там.

Въпреки че беше женен и имаше едно дете, Ралф беше решил да ме придружи в това пътуване. Смяташе се, че възнамерява да установи контакти и да купи стоки, които да продава с комисионна; но по-късно открих, че поради някакво недеволство от роднините на жена му, той възнамеряваше да я остави на техни грижи и никога да не се върне. След като се сбогувах с приятелите си и обмених известни обещания с г-ца Рийд, отпътувах от Филаделфия с кораба, който акостира в Нюкасъл. Губернаторът беше там; но когато отидох в квартирата му, секретарят му ме посрещна с най-любезното съобщение на света от губернатора, че не може да се срещне с мен в момента, тъй като е зает с изключително важна работа, но ще ми изпрати писмата на борда, сърдечно ми пожела на но добър път и бързо завръщане и т.н. Върнах се на борда малко объркан, но все още без съмнения.

Г-н Андрю Хамилтън, известен адвокат от Филаделфия, пътуваше в същия кораб със сина  си, и те, заедно с г-н Денъм\footnote{Denham}, търговец-квакер, и господата Ониън\footnote{Onion} и Ръсел\footnote{Russel}, шефове на завод за желязо в Мериленд, бяха заели голямата каюта; по тази причина Ралф и аз бяхме принудени да вземем каюта в кормилното, и понеже никой на борда не ни познаваше, ни смятаха за обикновени хора. Но г-н Хамилтън и синът му (това беше Джеймс, който оттогава стана губернатор) се върнаха от Нюкасъл във Филаделфия, тъй като бащата беше привикан да пледира за един задържан кораб срещу голям хонорар; и тъкмо преди да отплаваме, полковник Френч се качи на борда и ми засвидетелства голямо уважение, след което повече ме забелязаха и с Ралф бяхме поканени от другите джентълмени да се преместим в каютата, в която сега имаше място. Съответно ние се преместихме.

Когато разбрах, че полковник Френч е донесъл писмата на губернатора, помолих капитана да ми даде писмата, които са за мен. Той каза, че всичките са в торба, и че по това време не може да ги извади; но преди да стигнем Англия ще имам възможност да ги извадя, затова бях удовлетворен за момента, и продължихме с пътуването. Имахме общителна компания в каютата и живеехме необичайно добре, тъй като имахме в допълнение всичките запаси на г-н Хамилтън, който беше заредил обилно. При това преминаване г-н Денъм прихвана дружба с мен, която продължи цял живот. Иначе пътуването не беше приятно, тъй като времето беше доста лошо.

Котаго стигнахме до Ламанша, капитанът спази обещанието си и ми позволи да разгледам чантата с писмата на губернатора. Не намерих нито едно с моето има – като да е оставено на моя грижа. Избрах шест или седем, за които прецених по почерка, че може да са обещаните писма, още повече, че едно от тях беше за Баскет, кралския печатар, а едно друго до някакъв книгоиздател. Пристигнахме в Лондон на 24-ти декември 1724 г. Отидох при книгоиздателя, който пръв ми беше напът, и предадох писмото като от губернатор Кийт. „Не познавам такъв човек,“ казва той; но щом го отвори, „О! Това е от Ридълсдън. Скоро установих, че е пълен мошеник и не искам да имам нищо общо с него, нито да получавам писма от него.“ И така, като остави писмото в ръката ми, се завъртя на пета и ме остави, за да обслужи клиент. Изненадан бях, че това не са писмата на губернатора; и след като си спомних и сравних обстоятелствата, започнах да се съмнявам в искреността му. Потърсих приятеля си Денъм и му разказах цялата работа. Той ми отоври очите за характера на Кийт; каза, че нямало ни най-малка вероятност да е написал писма за мен; че никой, който го познава, не разчита на него дори за най-малкото нещо; и се изсмя на идеята, че губернаторът ще ми даде акредитивно писмо, тъй като, както той се изрази, не притежава кредит, който да даде. Когато споделих, че съм разтревожен и не знам какво да правя, той ме посъветва да опитам да си намеря работа с моя занаят. „Сред печатарите тук,“ казва той, „ще станеш по-добър и когато се върнеш в Америка, ще имаш преимущество при започването на бизнес.“

Така се случи, че и двамата знаехме – като книгоиздателя – че Ридълсдън, адвокатът, е истински негодник. Почти беше съсипал бащата на г-ца Рийд като го беше убедил да гарантира за него. От писмото изглеждаше, че е в движение тайна схема в ущърб на Хамилтън (за който се предполагаше, че ще пътува с нас); и че Кийт участва в нея заедно с Ридълсдън. Денъм, който беше приятел на Хамилтън, беше на мнение, че трябва да го запознае с нея; затова, когато пристигна в Англия – което стана малко след това – донякъде от разочарование и недоброжелание към Кийт и Ридълсдън, донякъде от доброжелание към него, му услужих и му дадох писмото. Благодари ми сърдечно, тъй като информацията беше важна за него. И от това време ми стана приятел, което многократно ми беше от голяма полза по-късно.

Но какво да кажем за губернатор, който си прави такива жалки шеги и се подиграва така дебелашки с бедно, неуко момче! Той беше придобил този навик. Искаше да угоди на всички; и тъй като нямаше възможност да даде много, даваше очаквания. Иначе беше изобретателен и разумен човек, доста добре пишеше, и беше добър губернатор за хората, но не и за тези, които го бяха избрали – собствениците на колонията – чийто съвети понякога пренебрегваше. Някои от най-добрите ни закони бяха изработени от него и приети от неговата администрация. 

Ралф и аз бяхме неразделни другари. Наехме си квартира заедно в Литъл Бритън\footnote{квартал на Лондон} за по три шилинга и половина на седмица – колкото можехме да си позволим тогава. Той намери някакви роднини, но те бяха бедни и не можеха да му помогнат. Тогава ми разкри намерението си да остане в Лондон и никога да не се върне във Филаделфия. Не беше взел никакви пари със себе си, тъй като всичко, което беше успял да събере беше отишло за да покрие пътуването. Аз имах петнадесет пистоли и той заемаше от мен отвреме навреме, за да преживява, докато си намери работа. Първо опита да се хване в театъра, понеже вярваше, че го бива за актьор; но Уилкъс, при когото се цани, направо го посъветва да не мисли за тази работа, защото няма никакъв шанс да успее в нея. След това предложи на Робъртс, издател от улица Патерностер,\footnote{Paternoster Row, център на издателската дейност в Лондон до втората световна война.} да списва за него седмичен вестник като Спектейтър, при определени условия, които Робъртс не хареса. След това се опита да намери работа като писар, да преписва за книгоиздателите и адвокатите около Темпъл\footnote{Temple, район в централен Лондон, където са съсредоточени съдилища, адвокатски професионални асоциации и юридически учреждения.}, но не успя да намери свободна позиция. 

Аз веднага започнах работа в Палмърс, печатница известна по това време в Бартоломю Клоуз, \footnote{Bartholomew Close} и продължих там една година. Бях доста усърден, но изхарчвах голяма част от приходите си в ходене по театри и други места за забавление с Ралф. Заедно бяхме изяли всичките ми дублони и сега карахме ден за ден. Той изглежда забравяше жена си и детето си, а аз постепенно моите уговорки с г-ца Рийд, на която не написах повече от едно писмо, с което ѝ съобщих, че най-верояно няма да се върна скоро. Това е още една от големите грешки в живота ми, и бих я поправил, ако можеш да преживея това време отново. Всъщност поради разходите ни постоянно не ми стигаха пари да си платя за преминаването обратно.

В Палмърс бяхме заети с композирането на второто издание на „Религията на природата“ на Уоластън. Тъй като някои от разсъжденията му ми се сториха не добре обосновани, написах малко метафизично съчинение, в което отправях някои забележки към тях. Нерекох го „Трактат за Свободата и Нуждата, Удоволствието и Болката.“ Посветих го на приятеля си Ралф; отпечатах малък брой. Това накара г-н Палмър да ме смята за по-находчив млад мъж, отколкото мислеше дотогава, макар че сериозно ми възрази по отношение на принципите в памфлета ми, които му се струваха ужасяващи. Отпечатването на този памфлет беше друга грешка. Докато живеех в Литъл Бритън се запознах с един Уилкокс, продавач на книги, чиято книжарница беше до нас. Имаше огромна колекция от книги втора ръка. По това време нямаше обществени библиотеки; но с него се съгласихме, че при определени разумни условия, които сега не си спомням, мога да взимам, чета и връщам коя да е от книгите му. Това беше голямо предимство и аз се възползвах от него доколкото можех. 

Памфлетът ми попадна в ръцете на един Лайънс, хирург, автор на книга наречена „Непогрешимостта на човешкия разум“\footnote{The infallibility of human judgement}, и това стана причина да се запознаем. Той ми обръщаше много внимание, често ме канеше да разговаряме на тези теми, водеше ме в \textit{The Horns}, бирария със светла бира на улица ----- в Чийпсайд и ме запозна с Д-р Мандевил, автор на „Басня за пчелите“ \footnote{Fable of the Bees}, който имаше клуб там, чието сърце беше, бидейки най-шеговит и забавен събеседник. Лайънс също така ме представи на Д-р Пембъртън в кафенето Батсън, който обеща да ми даде възможност някой път да видя сър Айзък Нютон, което много исках; но това никога не стана.

Бях донесъл някои интересни неща, сред които основното беше една кесия от азбеста, която пречиства чрез огън. Сър Ханс Слоан чу за нея, дойде да се срещне с мен и ме покани в къщата си на площад Блумсбъри, където ми показа цялата си колекция от странни предмети и ме убеди да му позволя да добави и този към тях, за което ми плати хубаво. 

В нашата къща живееше млада жена – шапкарка – която, мисля, имаше дюкян в \textit{the Cloisters}. Тя беше добре възпитана, разумна и енергична и много приятен събеседник. Ралф ѝ четеш пиеси вечерно време, сближиха се, тя се премести в друга квартира, а той я последва. Живяха заедно известно време; но понеже той още нямаше работа, а нейните доходи не стигаха да ги издържат с нейното дете, той реши да напусне Лондон и да опита да намери работа в училище в провинцията, за което смяташе, че е добре подготвен, тъй като имаше отличен почерк и беше много добър в аритметиката и счетоводството. Но, понеже смяташе, че тази работа е под нивото му, и беше сигурен, че в бъдеще ще има по-добър късмет, когато няма да иска да се знае, че някога е работил такава скромна работа, си смени името, и ми оказа честта да вземе моето; скоро след това получих писмо от него, с което ме осведомяваше, че се е установил в малко село (в Бъркшайър, струва ми се, че беше, където преподаваше четене и писане на десет или дванадесет момчета, за шест пенса седмично на глава), и с което ми поръчваше да се грижа за г-жа Т----, и ме молеше да му пиша, като адресирам писмата си до г-н Франклин, учител, на това и това място. 

Той продължи да ми пише често, като ми изпращаше големи откъси от една епична поема, която пишеше по онова време, и ме молеше за поправки и забележки. Отвреме навреме му изпращах такива, но по-скоро се опитвах да го обезсърча в работата му. По онова време тъкмо беше излязла една от Сатирите на Юнг. Преписах и му изпратих една голяма част от нея, която ясно показваше колко е глупаво човек да преследва Музите с надеждата чрез тях да успее. Всичко беше напразно; листове от поемата продължиха да пристигат с всяка поща. Междувременно, г-жа Т----, която според неговия разказ, беше загубила приятели и бизнес, често изпадаше в нужда и ме търсеше да ѝ заема колкото можех да заделя, за да ѝ помогна. Комапнията ѝ започна да ми харесва и – понеже по онова време нямах религиозни задръжки – възползвайки се от важното си положение пред нея, опитах интимности (друга грешка), които тя отхвръли със справедливо негодувание, и го уведоми за поведението ми. Това доведе до разрив между нас; и когато се върна в Лондон, той ме уведоми, че смята, че съм го освободил от всички негови задължения към мен. Така разбрах, че не трябва да очаквам да ми върне парите, които му бях заел. Това обаче не бе от голямо значение, понеже той не беше в състояние; и загубата на това приятелство ме облекчи от бреме. По това време започнах да мисля да оставям малко пари настрана, и, в очакване на по-добра работа, напуснах Палмърс и се преместих в Уатс, близо до Линкълнс Ин Фийлдс, още по-голяма печатница. Тук останах до края на пребиваването си в Лондон.

Тъй като ми се струваше, че ми липсват физическите упражнения, с които бях свикнал в Америка, където работата на печатарската преса се комбинира с композирането, в тази печатница в началото започнах работа на пресата. Пиех само вода; другите работници, близо петдесет на брой, смучеха бира като смокове. От време на време носех по една голяма каса с букви във всяка ръка нагоре по стълбите, докато други носеха само една в двете ръце. Това и няколко други случая ги накараха да се чудят как водния Американец – както ме наричаха – е по силен от тях, които пиеха силна бира. Имахме момче от кръчмата, което винаги беше в печатницата, за да снабдява работниците. Колегата ми на пресата пиеше по една халба преди закуска, между закуската и обеда, на обед, следобед към шест, и още една след работа всеки ден. Намирах това за отвратителен обичай, но той смяташе, че трябва да пие силна бира, за да е силен на работа. Опитах да го убедя, че телесната сила, която може да получи от бирата е пропорционална на количеството хмелово брашно разтворено във водата, от която е направена; и че в хляб за едно пени има повече брашно; и че, съответно, ако го изяде с една пинта вода, ще му даде повече сила от една кварта бира. Въпреки това, той продължи да пие, и плащаше четири или пет шилинга от заплатата си всяка събота вечер за алкохола, който му размътваше ума; разход, от който аз бях освободен. И така тези окаяници винаги бяха пияни.

След няколко седмици Уатс поиска да се преместя в стаята за композиране и така оставих работниците на пресата; композиторите ми поискаха нови пет шилинга пари за пиене за 'добре дошъл'. Сметнах го за нахалство, тъй като вече бях платил на долния етаж; главният беше на същото мнение и ми забрани да платя. Удържах две или три седмици, през които съответно бях смятан за отлъченик, и колкото пъти излязох от стаята, толкова пъти си докарах малки пакости, като: разбъркване на буквите ми, транспониране на страниците ми, чупене на инструментите ми, и т.н., и т.н., които всичките бяха приписани на домашния таласъм, който, казват, преследва тези, които са приети не по обичайния ред, така че – въпреки защитата на главния – се убедих, че е глупаво да съм в лоши отношения с тези, с които живея, и се видях принуден да платя.

Сега бях на равна нога с тях и скоро придобих значително влияние. Предложих някои разумни промени в правилата на техния параклис\footnote{Печатниците винаги биват наричани параклиси от работниците, като произхода на тази традиция изглежда е фактът, че в началото печатането в Англия се практикува в древен параклис преустроен в печатница, и името е запазено от традицията. 'Bien venu'-то сред работниците отговаря на условията за прием сред механиците; така работник, който постъпва в печатница, обикновено плащал един или повече галони бира за доброто на параклиса; този обичай отмираше преди тридесет години; много правилно е напълно отхвърлен в Съединените щати. - Уилям Темпъл Франклин } и ги прокарах преодолявайки всичката съпротива. По мой пример много от тях се убедиха, че на цената на една халба бира – т.е. пени и половина - могат да се снабдяват като мен от една съседна къща с голяма купа топла овесена каша, поръсена с пипер и трохи и с малко масло, и оставиха пиянската закуска с бира, хляб и кашкавал. Това беше по-удобна и евтина закуска, и освен това оставяше главата по-бодра. Тези, които продължиха да смучат бира по цял ден, понеже не си плащаха в кръчмата често не можеха да взимат повече на вересия и взимаха от мен на заем с лихва, тъй като светлината им беше изгаснала, както те се изразяваха. В събота вечер изчиствах таблицата с дълговете и събирах каквото бях дал, като понякога ми се налагаше да плащам до тридесет шилинга седмично за тях. Тези неща, както и това, че ме смятаха за порядъчно добър \textit{riggite}, т.е. шеговит игрослов сатирист, поддържаше позицията ми в общността. Постоянното ми присъствие пък (никога не си взимах \textit{St. Monday}\footnote{по времето на Франклин не било необичайно занаятчиите да отсъстват от работа в понеделник}) ме издигна в очите на главния; а поради необичайната ми бързина при композирането ми възлагаха всички бързи поръчки, които като цяла бяха по-добре платени. Така че сега се справях доста добре.

Понеже квартирата ми в Литъл Бритън беше доста далеч, намерих друга на улица Дюк, срещу католическия параклис. Беше от задната страна на втория етаж на един италиански склад. Вдовица държеше къщата; имаше дъщеря, слугиня и един работник, който идваше да се занимава със склада, но спеше другаде. След като изпрати да разпита за характера ми в къщата, в която живеех дотогава, тя се съгласи да ме вземе на същата цена, 3 шилинга и 6 пенса седмично; по-евтино заради – както тя каза – защитата, която очаква от присъствието на мъж в къщата. Тя беше вдовица, стара жена; беше отгледана като протестантка, дъщеря на пастор, но по-късно съпругът ѝ я обърнал в Католицизма, чиято памет тя много тачеше; дълго бе живяла сред отбрани хора и знаеше хиляда анекдота за тях чак до времето на Чарлз II. Колената ѝ бяха окуцяли от подаграта и по тази причина рядко излизаше от стаята си, така че понякога ѝ липсваше компания; и тъй като нейната ми беше много забавна, прекарвах вечер с нея винаги, когато тя поискаше. Вечеряхме по половин аншоа на много тясна филия хляб с масло, и си разделяхме половин халба светла бира; това, което ни забавляваше, бяха нейните истории. Тъй като винаги се прибирах рано и създавах малко неприятности на семейството, тя не искаше да се раздели с мен; така че, когато споменах за една квартира по-близо до работата ми за два шилинга на седмица, което имаше значение сега, когато бях съсредоточен върху спестяването на пари, тя ми каза и да не помислям, защото ще ми отстъпи два шилинга седмично за в бъдеще; така останах при нея за шилинг и шест пенса до края на престоя си в Лондон.

В една таванска стайчка на къщата й, в най-голямо усамотение, живееше седемдесетгодишна госпожица, за която хазяйката ми разказа следната история. Била католичка, като млада била изпратена в чужбина и живяла в девически манастир, с намерението да стане монахиня; страната обаче не ѝ допаднала, и затова се върнала в Англия, където – понеже в Англия няма девически манастири – взела обет да живее като монахиня, доколкото е възможно при дадените условия. Съответно раздала всичкия си имот за благотворителни цели, като запазила само дванадесет паунда годишно, с които да се издържа, а от тях също давала голяма част за благотворителност, живеейки само на каша от овес и вода и палейки огън само за да си свари кашата. Беше живяла в таванската стаичка много години, благодарение на поредица от наематели католици, които ѝ бяха позволили да остане безплатно, тъй като смятали присъствието ѝ за благословия. Свещеник идваше да я изповяда всеки ден. „Питала съм я,“ казва моята хазяйка, „как при нейният начин на живот има такава нужда от изповедник.“ „О,“ каза ми тя, „не е възможно да се опази човек от суетни мисли.“ Веднъж ми беше позволено да я посетя. Беше щастлива и учтива, и беше приятно да се разговаря с нея. Стаята беше чиста, но в нея нямаше други мебели освен матрак, маса с разпятие и книга, стол, на който ми даде да седна, и една картина закачена на комина, изобразяваща св. Вероника с кърпичката ѝ с чудотворното изображение на кървящото лице на Христос на нея, което тя ми обясни много сериозно. Изглеждаше беледа, но никога не боледуваше; и го давам като още един пример за това с колко малко пари може да се поддържа човек жив и в добро здраве.

В печатницата на Уат се запознах с един Уигът\footnote{Wygate}, находчив млад човек, който благодарение на богатите си роднини беше по-добре образован от повечето печатари; знаеша латински порядъчно добре, говореше френски и обичаше да чете. При две разходки до реката научих него и един негов приятел да плуват, и те скоро станаха добри плувци. Те ме запознаха с няколко джентълмена от провинцията, които веднъж отидоха по вода до Челси, за да видят Колежа и любопитните предмети в кафето Дон Салтерос. На връщане оттам, по желание на компанията, чието любопитство Уигът беше подразнил, се съблякох и скочих в реката и плувах от близо до Челси до Блекфрайърс\footnote{Около 5 километра по посока на течението на Темза, бел. пр.}, като по пътя демонстрирах различни умения под и над водата, които изненадаха и доставиха удоволствие на тези, за които бяха непознати.

От дете това упражнение ми доставяше удоволствие. Бях изучил и тренирал всичките движения и позиции на Тевно\footnote{Thevenot} и бях добавил някои мои, като се стремях към това, което е изящно и лесно както и към полезното. Възползвах се от случая да демонстрирам всички тези неща на компанията и бях поласкан от тяхното възхищение; а Уигът, който имаше голямо желание да стане добър плувец, заради това, а и заради сходността на нещата, които изучавахме, още повече се привърза към мен. По-късно ми предложи да обиколим цяла Европа заедно издържайки се където отидем със занаята си. По едно време бях склонен да го направя; но когато го споменах на добрия ми приятел Денъм, с когото често се виждах, когато имах свободно време, той ме разубеди и ме посъветва да мисля само за връщането си в Пенсилвания, където и той скоро щеше да се върне.

Трябва да спомена една черта от характера на този добър човек. По-рано той се беше занимавал с бизнес в Бристъл, но задлъжнял към доста хора, нещата се усложнили и заминал за Америка. Там, чрез усърдие в търговията за няколко години събрал голямо богатство. След като се върнал в Англия с мен с кораба, поканил старите си кредитори на забава, на която им благодарил за благосклонността, с която се отнесли към него, и – когато те не очаквали нищо повече от гощавката – след като първото ястие било вдигнато от маста всеки намерил под чинията си банково нареждане за цялата останала дъжима сума с лихва. 

Сега той ми каза, че ще се върне във Филаделфия с много стока, за да отвори магазин там. Предложи да ме вземе като писар, да се грижа за счетоводството му – за което щеше да ме обучи – да преписвам писмата му, и да наглеждам магазина. Добави, че щом се запозная с търговията, ще ме повиши и ще ме изпрати с товар брашно, хляб и т.н. до Карибите и ще ми осигурява поръчки от други, които ще ми носят печалба; а ако се справя добре, ще ми позволи да се установя предоволно. Идеята ми хареса; защото вече се бях уморил от Лондон и си спомнях с удоволствие за приятните месеци, които бях прекарал в Пенсилвания, и ми се искаше отново да я видя; затова веднага се съгласих на петдесет паунда годишно, пенсилвански пари; действително това беше по-малко от тогавашния ми доход, но ми предлагаше по-добри перспективи.

Затова сега спрях с печатарската работа – завинаги, както си мислех – и ежедневно зает в новата си работа, обикалях с г-н Денъм сред търговците да купувам разни неща и да съблюдавам как ги опаковат, тичах по задачи, посещавах работници, които имаха да пращат нещо, и т.н.; и когато всичко беше на борда имах няколко свободни дни. В един от тези дни за голяма моя изненада бях повикан от един велик човек, когото познавах само по име, един Сър Уилям Уиндъм\footnote{William Wyndham} и отидох да го посетя. По един или друг начин беше научил, че съм плувал от Челси до Блекфрайърс, и че за няколко часа съм научил Уигът и един друг младеж да плуват. Той имаше двама сина, които скоро щяха да тръгнат на пътешествие; искаше първо да бъдат обучени да плуват, и предложи да ми плати предоволно за да ги науча да плуват. Те все още не бяха дошли в града, а не беше сигурно колко ще остана, затова ми беше невъзможно да се заема; но този случай ме накара да мисля, че ако остана в Англия да отворя школа по плуване, бих могъл да правя доста пари; и особено ме впечатли, че ако предложението ми беше отправено по-рано, едва ли щях да се върна в Америка токлова скоро. След много години ти и аз имахме по-важна работа с тези синове на Сър Уилям Уиндъм, който по-късно стана ърл на Егръмонт\footnote{Egremont}, за което ще разкажа където му е мястото. 

Така прекарах около осемнадесет месеца в Лондон; през по-голямата част от времето работих усилено в занаята си и харчех малко за себе си, като изключим гледането на пиеси и купуването на книги. Заради приятелят ми Ралф бях останал беден; дължеше ми около двадесет и седем паунда, които най-вероятно нямаше да получа; голяма сума за малките ми доходи! Въпреки това го обичах, защото имаше много хубави страни. По никакъв начин не бях подобрил финансовото си състояние; но се бях запознал с някои много интелигентни познати, и разговорите с тях ми бяха от голяма полза; и бях прочел доста неща. 

Отплавахме от Грейвсенд\footnote{Gravesend} на 23 юли 1726. За случките от пътуването тe препращам към дневника ми, където ще ги намериш разказани в подробности. Може би най-важната част от този дневник е планът\footnote{Дневникът е бил публикуван от Спаркс по копие направено в Рединг през 1787. Но не съдържа плана.}, който може да бъде намерен в него, и който изготвих докато плавахме, относно начинът по който ще направлявам живота си в бъдеще. Още по-забележителен го прави фактът, че бе съставен, когато бях толкова млад, и въпреки това бе следван вярно до доста дълбока старост.

Акостирахме във Филаделфия на 11-ти октомври и установих разни промени там. Кийт вече не беше губернатор и беше сменен от майор Гордън. Срещнах го докато се разхождаше по улицата като обикновен гражданин. Изглежда малко се засрами щом ме видя, но отмина без да каже нищо. И аз щях да се засрамя толкова при среща с г-ца Рийд, ако при получаването на писмото ми нейните приятели не бяха основателно спряли да се надяват, че ще се върна, и не я бяха убедили да се ожени за друг, един Роджърс, грънчар, докато отсъствах. С него обаче тя никога не успя да бъде щастлива и скоро се раздели с него, отказвайки да съжителства с него и да носи името му, като по това време започна да се говори, че имал друга жена. Той беше безполезен тип, макар да беше отличен работник, което беше и съблазнило приятелите й. Натрупа дългове, избяга през 1727 или 1728, отиде на Карибите и умря там. Каймер беше купил по-добра къща, магазин добре снабден с канцеларски принадлежности, много нови шрифтове, нови помощници, макар и не добри, и изглежда имаше доста работа.

Г-н Денъм взе един магазин на улица Уотър и там разопаковахме нашите стоки; аз се грижех за работата усърдно, учех счетоводство, и за кратко време станах експерт в продажбите. Живеехме и се хранехме заедно; той ме съветваше като баща и се отнасяше към мен със сърдечна грижа. Аз го уважавах и обичах и може би щяхме да продължим заедно много щастливи; но в началото на февруари 1726-7, когато тъкмо бях навършил двадесет и една, и двамата се разболяхме. Аз имах плеврит, който почти ме отнесе. Доста страдах, в съзнанието си се отказах, и бях доста разочарован когато видях, че се възстановявам, съжалявайки донякъде, че някой ден ще трябва отново да свърша всичката тази работа. Забравил съм вече от какво боледува той; държа го дълго време и накрая го отнесе. Остави ми малко наследство в устно завещание, като знак за неговата добрина към мен, и отново ме остави на широкия свят; защото изпълнителите му поеха грижата за магазина и ме освободиха.

Зет ми Холмс, който сега беше във Филаделфия, ме посъветва да се върна към занаята си; а и Каймер ме изкуши с предложение за ежегодно растяща заплата да дойда и да поема управлението на печатницата му, за да може по-добре да се грижи за магазина за канцеларски материали. Бях чул от жена му и нейните приятели в Лондон лоши неща за него и не ми се искаше да се замесвам с него повече. Опитах отново да намеря работа като търговски счетоводител; но тъй като не излезе нищо бързо, отново се споразумях с Каймер. В печатницата му заварих следните помощници: Хю Мередит \footnote{Hugh Meredith}, пенсилванец с уелски корени, тридесетгодишен, обучен за селска работа; честен, разумен, имаше доста правилни наблюдения, имаше читателска жилка, но пиеше. Стивън Потс\footnote{Stephen Potts}, млад селянин достигнал пълнолетие, също обучен за селска работа, с необичайна физика, с голям ум и чувство за хумор, но малко мързелив. С тях той бе се спазарил за изключително ниски седмични надници, с повишение от по един шилинг на всеки три мисеца, което те щяха да заслужат като подобряват работата си; и той ги бе привлякъл с очакването за високите заплати, които ще дойдат по-късно. Мередит щеше да работи на печатарската преса, а Потс щеше да се занимава с подвързване, за което по споразумение, той щеше да ги обучи, въпреки че не умееше нито едното, нито другото. Джон ---, див ирландец, без занаят, когото Каймер беше наел за четири години от капитана на един кораб; той също щеше да бъде печатар. Джордж Уеб\footnote{George Webb}, учил в Оксфорд, когото Каймер също беше наел за четири години, с намерението да го направи съставител, за когото повече след малко; и Дейвид Хари\footnote{David Harry}, селско момче, което беше взел за чирак. 

Скоро разбрах, че ме наема на заплата толкова по-висока от тази, която преди плащаше, с намерението да ме използва да оформя тези груби, евтини кадри; и щом ги обучех, той щеше да може да се справи без мен, тъй като те всичките бяха обвързани с него с договор. Въпреки това, с много охота се захванах с работата, сложих печатницата – която беше в голям безпорядък – в ред, и лека полека научих помощниците му да си гледат работата и да я вършат по-добре. 

Беше странно да се види човек учил в Оксфорд в положението на нает работник. Не беше на повече от осемнадесет години, и ми разказа за себе си следното; бил роден в Глостър, учил в гимназия там\footnote{grammar school. Според Cambridge English Dictionary online, исторически grammar school е училище за ученици на възраст 11-18, които са добри в ученето}, когато поставяли театрални постановки се бил отличил сред учениците с очевидно превъзходство в изпълнението на неговите роли, бил член на Клуба на остроумните\footnote{Witty Club} там, и писал поезия и проза, които били отпечатани в Глостърските вестници; оттам бил изпратен в Оксфорд; там продължил около година, но не бил доволен, тъй като най-много му се искало да види Лондон и да стане актьор. Накрая, след като получил тримесечната си издръжка от петнадесет гвинеи, вместо да си погаси дълговете, напуснал града, скрил тогата си в един прещипов храст\footnote{или храст жълтуги, furze bush}, и поел пеша към Лондон, където понеже нямал приятели да го посъветват, попаднал в лоша компания, скоро изхарчил всичките си гвинеи, не намерил начин да се запознае с актьорите, изпаднал в немотия,  заложил дрехите си, и нямал какво да яде. Както ходел един ден по улицата много гладен и без да знае какво да прави със себе си, някой му тикнал дипляна в ръцете, която предлагала незабавна гощавка и насърчение за тези, които биха се наели да служат в Америка. 

Отишъл направо, подписал договора, с който се цанил чирак, бил качен на кораб и пресякъл океана без да напише нито ред на приятелите си, за да им разкаже какво се е случило с него. Беше жив, остроумен, добродушен, и приятна компания, но мързелив, без склонност към размишление, и крайно неразумен.

Джон – ирландецът – скоро избяга; с останалите започнах да живея доста приятно поради тяхното растящо уважение към мен, след като откриха че Каймер беше неспособен да ги обучи, а от мен научаваха нещо всеки ден. В събота никога не работехме, тъй като това беше Съботата на Каймер, и затова имах два дни за четене. Познанствата ми с интелигентни хора в града се умножаваха. Самият Каймер се отнасяше с мен много учтиво и с привидна грижа, и сега само дългът ми към Върнън – който все още не можех да платя, тъй като до този момент бях само слаб спестовник – ме тревожеше. Той обаче учтиво не го изискваше.

В нашата печатница често липсваха букви, а в Америка нямаше букволеяр\footnote{letter-founder  (виж също type-founder –> букволеяр)}; при Джеймс в Лондон бях виждал как се леят букви, без обаче да обръщам много внимание на технологията; въпреки това, сега измислих форма, използвах буквите, които имахме за щампи, излях матрици от олово, и така по доста приемлив начин успях да запълня всички липси. Случи се също така да гравирам няколко неща; правих мастило; бях отговорник за склада и за всичко, и въобще бях момче за всичко.

Но колкото и полезен да бях, открих, че с всеки изминал ден, с напредването на останалите в работата, услугите ми ставаха по-маловажни; когато Каймер ми плати възнаграждението за второто тримесечие, ме усведоми, че му идва в повече, и че смята, че трябва да направя отстъпка. С времето ставаше все по-неучтив, държеше се като по-господарски, често се задяваше, придиряше за дреболии, и изглеждаше готов да избухне. Въпреки това продължих с голямо търпение, мислейки, че отчасти неговото тежко финансово състояние е причината. В крайна сметка една дреболия ни скара; голяма врява се беше вдигнала пред съда и аз подадох глава през прозореца, за да видя какво става. Каймер беше на улицата, погледна нагоре, видя ме, и ми викна сърдито с висок глас да си гледам работата, като добави някои упреци, които ме жегнаха още повече, защото ги каза пред други хора – всички съседи, които бяха по прозорците заради врявата видяха как се отнася с мен. Той веднага влезе в печатницата, продължи караницата, силни думи бяха казани и от двете страни, и той ми даде тримесечното предизвестие, за което се бяхме договорили, като сподели, че съжалява, че се е съгласил на такова дълго предизвестие. Аз му казах, че няма нужда да съжалява, защото ще напусна веднага; и така аз си взех шапката и си излязох, и помолих Мередит, който видях долу, да се погрижи за някои неща, които бях оставил и да ми ги донесе в квартирата.

Мередит дойде вечерта и тогава обсъдихме случката. Той се беше привързал много към мен и не желаеше да остане в печатницата без мен. Убеди ме да не се връщам в родния си край, за което започнах да мисля; напомни ми, че всичко, което Каймер притежава, е купено със заеми; че кредиторите му започват да се притесняват; че ръководи бизнеса си много неумело, често продава без печалба за пари на ръка, и често се доверява, без да си води отчет; и че поради тези причини трябва да падне, което ще отвори възможност, от която аз бих могъл да се възползвам. Аз възразих с липсата на пари. Тогава той ми каза, че баща му има добро мнение за мен, и че заради разговори, които са водили, той е сигурен, че ще даде пари, за да започнем бизнес, ако се съглася да си партнирам с него. „Договорът ми“, казва той, „с Каймер изтича през пролетта; дотогава можем да сме си доставили букви и преса от Лондон. Съзнавам, че не съм добър работник; ако искаш, твоите умения в работата ще вложим срещу капитала, който аз ще осигуря, и ще делим печалбите поравно.“

Предложението беше привлекателно и се съгласих; баща му беше в града и го одобри; още повече, като видя, че имам голямо влияние върху сина му, и бях успял да го убедя дълго време да се въздържа от пиянство, и се надяваше, че ще успея да го отуча съвсем от този злощастен навик, ако се обвържем така тясно. Дадох инвентар на бащата, който го предаде на търговец; нещата бяха поръчани, тайната щеше да бъде пазена докато не пристигнат, а междувременно щях да се хвана на работа – ако е възможно – в другата печатница. Но там не открих свободна позиция и така стоях без работа няколко дни, докато Каймер – изправен пред възможността да отпечата някакви хартиени пари в Ню Джърси, за които щяха да са нужни гравюри и различни букви, които само аз можех да изработя, и опасявайки се, че Брадфърд може да ме наеме и да му измъкне поръчката – не ми изпрати учтиво съобщение, че стари приятели не трябва да се разделят за няколко думи изречени във внезапен пристъп, и иска да се върна. Мередит ме убеди да приема, понеже това щеше да предостави повече възможности той да се учи в работата под ежедневното ми ръководство; затова се върнах и нещата вървяха по-гладко отколкото известно време преди това. Поръчката от Ню Джърси беше спечелена, измислих медна преса за нея, първата в страната; гравирах няколко орнамента и чекове за банкнотите. Отидохме заедно до Бърлингтън, където свърших цялата работа задоволително; и той получи достатъчно голямо количество пари от работата, за да се задържи на повърхността много по-дълго.

В Бърлингтън се запознах с много от главните хора в провинцията. Неколцина от тях бяха избрани от Събранието да сформират комитет, който да присъства при печатането и да се погрижи, да не бъдат отпечатани повече пари отколкото законът предписва. По тази причина те бяха постоянно с нас, и като цяло този, който ни наглеждаше водеше по един или двама приятели за компания. Приказката ми изглежда се ценеше повече от тази на Каймер, предполагам понеже умът ми се беше подобрил от четене много повече от неговия. Канеха ме по къщите си, представяха ме на приятелите си, и се отнасяха с мен много любезно; докато той – макар и шеф – беше малко пренебрегнат. Всъщност, той беше странна риба; не познаваше обикновения живот, обичаше грубо да се противопоставя на общоприети схващания\footnote{ receiv'd opinions}, нечистоплътен до крайна занемареност, въодушевен относно някой аспекти на религията и като цяло малко негодник. 

Прекарахме там близо три месеца; и в края им можех да причисля към новопридобитите си приятели съдия Алън, Самюъл Бъстил, секретар на провинцията, Айзък Пиърсън, Джоузеф Купър, и неколцина от Смитови, членове на Събранието, както и Айзък Декау, главния земемер. \footnote{Judge Allen, Samuel Bustill, Isaac Pearson, Joseph Cooper, the Smiths, Isaac Decow} Последният беше проницателен, мъдър стар човек, който ми разказа, че запознал да печели, когато бил млад, като карал с количка глина за тухларите, научил се да пише след като бил вече навършил пълнолетие, носил веригата за земемерите, които го научили на занаята си, и сега чрез трудолюбието си беше спечелил добър имот; и казва той, „Предвиждам, че ти скоро ще изкараш този човек от бизнеса му, и ще забогатееш от него във Филаделфия.“ По това време той не знаеше нищичко за моето намерение да започна бизнес там или където и да е. Тези приятели по-късно ми бяха от голяма полза, както и отвреме-навреме аз на тях. Всички те продължиха да ме тачат, докато бяха живи.

Преди да почна да разказвам за това как започнах бизнеса си може би е добре да ти разкажа за състоянието на ума ми по онова време по отношение на принципите и нравствените ми ценности, за да видиш доколо са оказали те влияние върху последващите събития от живота ми. Родителите ми отрано ми бяха дали религиозни напътствия и като дете благочестиво ме бяха отгледали като протестант. Но бях едва на петнадесет, когато, след като последователно се усъмних в няколко твърдения, които видях оспорени в различни книги, които бях прочел, започнах да се съмнявам в самото Откровение. Попаднаха ми няколко книги против деизма; за тях се говореше, че са същината на проповеди проповядвани като част от Лекциите на Бойл. \footnote{Boyle's Lectures, поредица от лекции основани от физика Робърт Бойл, който завещава определена сума за тях, чиято цел била да се докаже верността на християнските вярвания.} Така се случи, че те ми повлияха противно на очакваното; защото аргументите на деистите, които бяха цитирани, за да бъдат оборени, ми се сториха доста по-силни от оборващите; и с две думи скоро след това бях станал убеден деист. Моите аргументи поквариха и някои други, особено Колинс и Ралф; но тъй като и двамата по-късно ми сториха големи злини без каквито и да е угризения, и като си спомних за начина, по който Кийт (който също беше свободомислещ) се отнесе към мен, и за собственото ми поведение към Върнън и г-ца Рийд, което понякога ми създаваше големи неприятности, започнах да се подозирам, че тази доктрина, дори да е вярна, може би не е много полезна. Лондонският ми памфлет, за който бях взел като мото следните стихове на Драйдън 

	Каквото е, е правилно. Макар че полуслепия човек
	вижда само малка част от веригата, най-близкото звено:
	очите му не стигат до справедливия лъч,
	който уравновесява всичко горе.

и който от качествата на Бога, неговата безкрайна мъдрост, доброта и сила, заключваше, че нищо никога не се обърква в света, и че пороците и добродетелите са празни разграничения и не съществуват, вече не ми се виждаше толкова остроумно произведение, за каквото го смятах навремето; и почнах да се съмнявам, дали в разсъжденията ми не се е промъкнала някоя грешка, която е инфектирала всички последващи заключения, както често се случва в метафизичните размишления.

Постепенно се убедих, че истината, откровеността и честността в отношенията между хората са изключително важни за щастието в живота; и взех решения – които си записах, и които все още са в дневника ми – които да спазвам докато съм жив. Откровението, като такова, нямаше тежест пред мен; но бях на мнение, че макар и някои неща да не са лоши, защото са забранени от него или добри, защото са заповядани, все пак те вероятно са забранени от него, защото са лоши за нас или са ни заповядани, защото са добри за нас в естеството си, в края на крайщата. И това убеждение, благодарение на Провидението, или на някой ангел пазител, или на случайно благоприятно стечение обстоятелства и случки, или на всичките взети заедно, ме запази през опасния период на младостта и рискованите ситуации, в които понякога попадах сред непознати, далеч от очите и съветите на баща ми, чист от каквато и да е умишлена силно неморална или несправедлива постъпка, която човек би могъл да очаква поради моята нерелигиозност. Казвам умишена, защото в случаите, които споменах досега, има елемент на принуда, произлизащ от моята младост, неопитност и от чуждото мошеничество. Затова имах приемлив характер за начало в живота; ценях го както подобава и реших да го запазя.

Не мина много време след завръщането ни във Филаделфия, когато новите шрифтове от Лондон пристигнаха. Уредихме нещата с Каймер и го напуснахме с негово съгласие преди да чуе за това. Намерихме къща под наем до пазара и я наехме. За да свалим наема, който тогава беше едва двадесет и четири паунда годишно – а научих, че по-късно са я давали и за седемдесет – взехме Томас Годфри – стъклар – и семейството му, които щяха да плащат значителна част от наема на нас, а ние щяхме да се храним с тях. Едва бяхме разопаковали буквите и стъкмили пресата, когато Джордж Хаус\footnote{George House}– мой познат – доведе един човек от провинцията, когото беше срещнал на улицата да пита за печатар. Всичките ни пари в брой бяха изхарчени в различни консумативи, които трябваше да си набавим и петте шилинга на този селянин, първите плодове на нашия труд, и толкова навременни, ми доставиха повече удоволствие от всички крони, които съм спечелил оттогава; и благодарността, която изпитах към Хаус, може би често ме е карала да съм по склонен да помагам на млади начинаещи отколкото бих бил иначе. 

Във всяка страна има черногледци, които винаги предвещават края й. Един такъв живееше във Филаделфия по онова време; изтъкнат човек, възрастен, с мъдро изражение и много сериозен начин на говорене; казваше се Самюел Микъл. Този джентълмен, когото не познавах, спря един ден пред моята врата, и ме попита дали не съм младият човек, който наскоро е отворил печатница. След като получи положителен отговор, той каза, че съжалява за мен, защото това било скъпо начинание, и съм щял да си загубя парите; защото Филаделфия отивала надолу, хората вече били полу-разорени, или почти разорени; всички знаци, че нещата вървят добре, като увеличаващите се наеми и строителството на нови сгради, били фалшиви, според неговите сигурни сведения; защото те били – всъщност – сред нещата, които щели скоро да ни съсипят. И ми разказа в такива подробности за съществуващи несгоди и за такива, които скоро ще съществуват, че ме остави полу-меланхоличен. Ако го бях срещнал преди да започна с бизнеса си, сигрно никога не бих се захванал. Този човек продължи да живее в това разпадащо се място и да декламира в този дух, отказвайки в продължение на много години да купи къща там, защото всичко отивало към края си; в края на крайщата имах удоволствието да го видя да купи къща за пет пъти повече пари, отколкото би дал, ако бе я купил, когато започна да кряка с черните си предсказания.

Трябваше да спомена, че през есента на предишната година бях организирал повечето от находчивите ми познати в клуб за взаимно подобряване, който нарекохме Хунто; срещахме се в петък вечер. Правилата, които съставих, изискваха от всеки член, когато му дойде редът, да представи едно или повече разсъжения на тема нрави, политика, естествена философия, които да бъдат обсъдени от компанията; и веднъж на всеки три месеца да представи и прочете есе написано от него самия, на каквато тема му хареса. Обсъжданията ни щяха да са под напътствието на президент и да се провеждат в искрения дух на търсене на истината, без да се увличаме в спорове, или да се стремим към победа; и за да предотвратим разгорещяването, всички изрази на увереност в изразяваните мнения или на директно противоречие, след известно време бяха обявени за незаконни, и бяха позволявани само срещу малки парични глоби.

Първите членове бяха Джоузеф Брейнтнъл\footnote{Joseph Breintnal}, който преписваше актове за нотариусите, добродушен, дружелюбен човек на средна възраст, голям почитател на поезията, който четеше всичко, до което можеше да се добере, и пишеше някои поносими неща; много находчив, когато става дума за разни дреболии, и разумен събеседник.

Томас Годфри\footnote{Thomas Godfrey} – самоук математик, много добър в своята област\footnote{great in his way}, и по-късно изобретател на октанта (или: инструмента, който сега е известен като Хадлиев квадрант). Но той знаеше малко извън своята област\footnote{his way} и не беше приятна компания; защото – като повечето велики математици, които съм срещал – очакваше всеобхватна точност във всичко, което се казва, или прекарваше безкрайно количество време да отрича или разграничава дреболии, което правеше всякакъв разговор невъзожен. Той скоро ни напусна.

Никълъс Скъл\footnote{Nicholas Scull} – земемер, по късно главен земемер, който обичаше книгите и понякога пишеше стихове.

Уилям Парсънс\footnote{William Parsons}, който беше учил за обущар, но обичаше да чете, беше научил доста математика, която беше започнал да учи заради астрологията, на което по-късно се смееше. Той също стана главен земемер.

Уилям Могридж\footnote{William Maugridge} – дърводелец, прекрасен механик, и надежденен, разумен мъж.

Хю Мередит, Стивън Потс и Джордж Уеб, които вече описах.

Робърт Грейс – млад джентълмен с известно състояние, щедър, жив и остроумен; обичаше игрословиците и приятелите си.

И Уилям Колмън\footnote{William Coleman} – по това време чиновник при един търговец, на приблизително моя възраст, който притежаваше най-бистрата, и скъпа за мен глава, най-доброто сърце и най-взискателните нрави от почти всички, които съм срещал. По късно стана известен търговец и един от съдиите в провинцията ни. Приятелството ни продължи без прекъсване до смъртта му повече от четиридесет години; а клубът продължи почти толкова дълго и беше най-доброто училище за философия, нрави и политика, което съществуваше по онова време в провинцията; защото въпросите, които си поставяхме, и които биваха прочитани в седмицата преди тяхното обсъждане, ни караха да четем вниманелно по различните теми, за да можем да говорим по-съществени неща; и тук също подобрявахме начина, по който разговаряме, тъй като всичко в правилата ни беше замислено да ни попречи да се отвратим един от друг. Оттам и дългото съществуване на клуба, за който често ще имам повод да говоря оттук нататък.

Но го споменавам тук, за да покажа отчасти собствения си интерес – всеки един от тези правеше усилие да ни намира работа. Брейнтнъл примерно ни уреди да отпечатаме за квакерите четиридесет страници от историята им, като останалите щяха да бъдат отпечатани от Каймер; и по тази задача работихме изключително усилено, понеже цената беше ниска. Беше формат фолио, размер pro patria, 12 точки размер на шрифта,\footnote{ in pica} с дълги бележки\footnote{primer notes}. От нея съставях по един лист на ден, а Мередит го прекарваше през пресата; често оставах до единадесет вечерта или дори до по-късно, за да довърша партидата за утрешната работа, защото малките задачки изращани ни от нашите приятели от време на време ни забавяха. Но бях толкова решен да продължа да правя по един лист от фолиото на ден, че една нощ, когато тъкмо бях наредил касетите си\footnote{impos'd my forms}, и мислех, че съм приключил с работата за деня една от тях случайно се счупи и две страници станаха на нищо, веднага я разпределих и съставих\footnote{distributed and composed} наново преди да си легна; и това трудолюбие, което беше видно за съседите ни, започна да ни създава име и да ни печели доверие; в частност, беше ми казано, че при споменаването на новата печатница в търговския клуб „Всяка вечер“, общото мнение било, че ще пропадне, понеже вече имало двама печатари в града, Каймер и Брадфърд; но Д-р Беърд\footnote{Baird } (когото с теб видяхме много години след това в родното му място, Сейнт Андрюс в Шотландия) бил на противоположното мнение: „Защото трудолюбието на този Франклин,“ казал той, „надминава всичко от тоя род, което някога съм виждал; виждам го да работи когато се прибирам към къщи от клуба, и е на работа преди съседите му да са се измъкнали от легото.“ Това впечатлило останалите и скоро след това получихме предложение от един от тях да ни снабди с канцеларски материали; но за момента решихме да не се захващаме с отваряне на магазин.

Въпреки че изглежда, че сам се хваля, говоря за това трудолюбие по-свободно и с повече подробности, за да може тези от моето потомство, които прочетат разказа ми, да познаят ползата от тази добродетел, когато видят какви ползи ми е принасяла нееднократно.

По това време Джордж Уеб, който си беше намерил приятелка, която му даде заем, с който да се откупи от Каймер, дойде при нас да се цани за наемник. В този момент не можехме да го наемем; но от глупост аз му доверих като тайна, че скоро възнамерявам да започна вестник, и че тогава може да имам работа за него. Казах му, че надеждите ми за успех се основават на факта, че единственият вестник по онова време – издаван от Брадфърд – беше незначителна работа, управляван ужасно, изобщо не беше забавен, и въпреки това му носеше печалба; затова смятах, че един добър вестник надали ще срещне трудности в намирането на читатели. Казах на Уеб да не го споменава; но той каза на Каймер, който веднага, за да ме изпревари, публикува предложения да печата вестник, за което щеше да наеме Уеб. Това ме ядоса; и за да им противодействам, понеже все още не можех да започна с нашия вестник, написах няколко развлекателни парчета за вестника на Брадфърд, под заглавие "На всяка манджа - мерудия"\footnote{busybody}, което Брейнтнъл продължи няколко месеца. По този начин вниманието на обществеността беше приковано към този вестник и предложенията на Каймер, които пародирахме и осмяхме, бяха пренебрегнати. Въпреки това, той започна вестника си и след като го бута девет месеца с най-много едва деветдесетина абоната, ми го предложи на безценица; а аз от известно време бях готов да започна и го поех веднага; и след няколко години почна да ми носи изключителна печалба. 

Усещам, че съм склонен да говоря в единствено число, въпреки че партньорството ни продъжлаваше; може би причинате е, че всъщност цялото управление на бизнеса падаше върху мен. Мередит не ставаше за словослагател, беше слаб печатар и рядко трезвен. Приятелите ми оплакваха връзката ми с него, но трябваше да направя каквото мога от ситуацията.

Първите ни броеве изглеждаха доста различно от всичко излизало дотогава в провинцията; по-добър шрифт и по-добре отпечатани; но някои вдъхновени забележки, които написах относно спорът, който се водеше по онова време между губернатор Бърнет\footnote{Burnet} и Масачузетското събрание, впечатлиха първенците, предизвикаха много разговори относно вестника и неговия управител, и след няколко седмици доведоха до това, че всичките ни станаха абонати.

Примерът им беше последван от мнозина и бройката ни продължаваше постоянно да расте. Това беше една от първите ползи от това, че се бях понаучил да драскам; друга беше това, че първите мъже, когато видяха вестник в ръцете на някого, който може да борави и с писалката, намериха за подобаващо да ми правят услуги и да ме насърчават. Брадфърд все още печаташе бюлетините\footnote{votes}, законите и други обществени работи. Беше отпечатал обръщението на Камарата\footnote{the House} към губернатора по недодялан начин и с много грешки. Ние го отпечатахме елегантно и правилно и изпратихме по едно копие на всеки член. Те усетиха разлката: това даде коз в ръцете на приятелите ни в Камарата и следващата година ни избраха за техни печатари.

Сред приятелите ми там не бива да забравя г-н Хамилтън, който беше споменат по-рано, и който по онова време се беше върнал от Англия, и имаше място в Камарата. Той силно се застъпи за мен в този случай, както и в много други след това, и продължи да ме подкрепя до смъртта си\footnote{Веднъж дадох на сина му 500 паунда –[бележка в полето]}

По приблизително това време г-н Върнън ми напомни за дълга ми към него, но не ме притискаше. Написах му хитро писмо, в което признавах дълга и го молех да ме изчака още малко, която молба той удовлетвори, и щом можах му изплатих главницата с лихва и много благодарности; така грешката бе донякъде поправена.

Но тогава се сблъсках с друга трудност, която изобщо не можех да очаквам. Бащата на г-н Мередит, който трябваше да плати за нашата печатница според уверенията му към мен, беше успял да извади само сто паунда валута, които бяха платени; а на търговецът трябваше да се платят още сто, той изгуби търпени и ни даде всичките под съд. Ние платихме гаранция, но видяхме, че ако не успеем да съберем парите навреме, делото ще прилючи с присъда и нейното изпълнение, и добрите перспективи, които ни се откриваха, ще пропаднат заедно с нас, тъй като печатницата и буквите ще трябва да бъдат продадени – вероятно на половин цена - за да си платим дълга.

В тази беда двама верни приятели, чиято добрина никога не съм забравял, нито ще забравя докато мога да помня каквото и да е, дойдоха при мен отделно един от друг и без да знаят един за друг, и – без аз да ги моля – ми предложиха да ми предоставят достатъчно пари, за да мога да поема целия бизнес, ако е възможно; но не им хареса идеята да продължа съдружничеството с Мередит, когото – както те казаха – често може да бъде видян пиян на улицата или да играе неприлични игри в кръчмите, в доста голям ущърб на името ни. Тези двама приятели бяха Уилям Колмън\footnote{William Coleman} и Робърт Грейс\footnote{Robert Grace}. Тогава им казах, че не мога да предложа разваляне на съдружието, докато има надежда Мередитови да изпълнят тяхната част от споразумението ни, защото смятах, че съм им много задължен за всичко, което бяха направили, и биха направили, ако можеха; но ако в крайна сметка се провалят в изпълнението и се наложи да сложим край на съдружието, ще се чувствам свободен да приема помощта на приятелите си.

Нещата останаха така известно време, докато не казах на съдружника си, „Може би баща ти е недоволен от дела, който ти се пада в тази наша работа и не иска да плати за нас двамата, това, което искда да е само за теб. Ако е така, кажи ми, и ще се откажа то моят дял в твоя полза и ще си вървя по пътя.“ „Не,“ каза той, „баща ми наистина е разочарова и наистина не може (да плати); а аз не желая да го притеснявам повече. Виждам, че тази работа не е за мен. Отраснах като фермер и беше глупаво да идвам в града и на тридесет годишна възраст да се хващам за чирак да уча нов занаят. Много от нашите хора от Уелс отиват да се установят в Северна Каролина, където земята е евтина. Аз съм склонен да отида с тях, и да се върна към стария си поминък. Можеш да намериш приятели  да ти помогнат. Ако поемеш дълговете на компанията и върнеш на баща ми стоте паунда, които той даде, платиш малките ми лични задължения, и ми дадеш тридесет паунда и ново седло, ще се откажа от съдружието и ще го оставя цялото в твои ръце.“ Съгласих се с предложението му: оформихме го писмено, подписахме го и подпечатахме веднага. Дадох му каквото искаше и той скоро замина за Каролина, откъдето на следващата година ми изпрати две дълги писма, които съдържаха най-доброто описание на онази страна, климата, почвите, земеделието и т.н. което някой беше правил, защото той много разбираше от тези неща. Отпечатах ги във вестника и доставиха голямо удоволствие на читателите.

Щом замина, аз отидох при приятелите ми; и понеже не исках да бъда неучтив и да дам предимство на един от двамата, и от двамата взех по половината от това, което те ми бяха предложили а на мен не ми достигаше; изплатих дълговете на компанията и продължих с работата от мое име, като обявих, че съдружието е разтурено. Мисля, че това беше през или около 1729г.

По горе-долу това време се надигна вик за хартиени пари сред хората, тъй като в провинцията имаше само петнадесет хиляди паунда, и това скоро щеше да бъде намалено. Богатите жители бяха против увеличение, тъй като бяха против всякакви хартиени пари, от страх, че ще се обезценят, както се беше случило в Нова Англия, което би било в ущърб на кредиторите. В нашето Хунто бяхме обсъдили този въпрос, като аз се бях застъпил за увеличение, понеже бях убеден, че малката сума отпечатана през 1723 беше оказала много добро влияние като увеличи търговията, заетостта, и броя на жителите на провинцията, защото сега виждах, че всички стари къщи имаха обитатели, а и се строяха много нови; а си спомнях как, когато за пръв път се разходих из улиците на Филаделфия ядейки ролото си, видях табели „Дава се под наем“ на повечето къщи на Уолнът стрийт между Втора улица и Фронт стрийт; и също на много други на Честнът стрийт и други улици, което тогава ме накара да мисля, че обитателите на града го напускат един след друг.

Споровете ни на тази тема дотолкова ме погълнаха, че написах и отпечатах анонимен памфлет наречен „Естеството на хартиените пари и нуждата от тях“. Беше добре приет от обикновените хора като цяло; но богатите не го харесаха, защото увеличи и усили настояванията за повече пари, и тъй като така се случи, че те нямаха сред тях писатели, които да могат да му отговорят, съпротивата им отслабна и точката беше приета с мнозинство в Камарата.\footnote{the House} Приятелите ми там, които сметнаха, че съм им бил от полза, намериха за удачно да ме възнаградят като ми възложат отпечатването на парите; много доходна работа и голяма помощ за мен. Това беше още една полза, която умението да пиша ми донесе. 

Ползата от тази валута с времето и опита стана толкова очевидна, че никога след това не бе оспорвана сериозно; така че скоро нарастна до петдесет и пет хиляди паунда, а в 1739 до осемдесет хиляди паунда, от което време – през войната и досега – нарастна до над триста и петдесет хиляди паунда, като през цялото време търговията, строителството и населението се увеличаваха, и сега вече мисля, че има граници, отвъд които количеството може да донесе вреда.

Скоро след това чрез приятеля си Хамилтън спечелих поръчка за отпечатването на хартиените пари на Нюкасъл, което тогава също смятах за доходна работа, понеже малките неща изглеждат големи в очите на тези, които са в малки обстоятелства\footnote{small things appearing great to those in small circumstances}; и тези поръчки бяха наистина от голяма полза за мен, защото бяха много насърчителни. Той ми осигури и отпечатването на законите и бюлетините на правителството, и тази работа остана в мойте ръце докато бях в бизнеса. 

По това време отворих малък магазин за канцеларски принадлежности. Имах всякакви бланки, най-верните, които някога са излизали на пазара в нашия край, благодарение на приятеля ми Брейнтнъл\footnote{Breintnal}. Имах също хартия, пергаментова хартия, книжа за странстващи търговци\footnote{chapmen's books}, и т.н. Един Уайтмаш, съставител който познавах от Лондон, отличен работник, дойде при мен по това време и работеше за мен постоянно и усърдно; и взех чирак, сина на Акила Роуз

По това време започнах постепенно да изплащам дълга за печатницата. За да си създам добро име като търговец внимавах не само в действителност да съм работлив и пестелив, а и да избягвам да създавам впечатление, че не съм. Обличах се просто; не посещавах места за суетни забавления; никога не излизах за риба или на лов; найстина понякога разпусках от работата с книга, но това беше рядко, приятно и не правеше скандал; и, за да покажа, че не се гнуся от занаята си, понякога сам докарвах у дома с количката хартията, която купувах в магазина. Така хората ме смятаха за трудолюбив, преуспяващ млад мъж, който плаща за това, което купува, според уговорката, търговците, които внасяха канцеларски материали търсеха работа с мен; други предлагаха да ме снабдяват с книги, и нещата вървяха доволно. Междуверемнно бизнесът на Каймер и доверието в него намаляваха ежедневно, докато накрая не бе принуден да продаде печатницата, за да задоволи кредиторите си. Отиде в Барбадос, и живя там няколко години в голяма бедност. 

Чиракът му, Дейвид Хари\footnote{David Harry}, когото бях обучил докато работих при него, си купи материали и отвори своя печатница във Филаделфия. В началото се опасявах, че той ще се окаже мощен съперник, тъй като приятелите му бяха доста способни, и имаха голямо влияние. Затова му предложих партньорство, което – за мое щастие – той отказа с презрение. Той беше много горд, обличаше се като джентълмен, живееше скъпо, прекарваше много време в забавление и удоволствия навън, влезе в дългове и занемари бизнеса си; при което поръчките му секнаха; и като не намери какво да прави, последва Каймер в Барбадос, взимайки печатницата със себе си. Там чиракът наел поранешния си господар като работник; често се карали; Хари постоянно закъснявал с поръчките; накрая бил принуден да продаде буквите и да се върне към селската си работа в Пенсилвания. Човекът, който ги купил наел Каймер, за да работи с тях, но след няколко години той умрял.

Тогава във Филаделфия не ми остана друг конкурент освен стария Брадфърд; който беше богат и отпуснат, отпечатваше по нещо отвреме навреме, колкото да се намира на работа, но не беше особено загрижен за бизнеса. Но, понеже държеше пощата, се смяташе, че има повече възможности да научава новини; прието бе, че неговият вестник е по-добро място за реклама от моя, и затова имаше много повече, което беше добър доход за него и в моя вреда; защото въпреки че наистина получавах и изпращах книжа по пощата, хората не знаеха за това, защото каквото изпращах ставаше тайно, чрез подкупване на ездачите, понеже Брадфърд беше достатъчно неучтив да ми го забрани, което малко ме ядоса; и поради това го смятах за толкова долен, че когато по-късно стигнах до неговото положение, се постарах да не го имитирам.

Дотогава продължавах да съжителствам с Годфри, който живееше със семейството си в една част от къщата ми, и използваше едниния край на работилницата  за стъкларския си бизнес, макар че работеше малко, винаги погълнат от математиката си. Г-жа Годфри ме сватоса с дъщерята на една приятелка, използваше възможности да ни събира често, докато не се стигна до сериозно ухажване от моя страна, като само по себе си момичето си заслужаваше. Родителите ѝ ме насърчаваха с постоянни покани за вечеря и като ни оставяха сами, докато накрая не стана време за обяснение. Г-жа Годфри подготви нашата малка спогодба. Казах ѝ, че очаквам толкова пари с дъщерята, колкото ми трябват, за да изплатя остатъка от дълга за печатницата, което по това време вярвам, че беше не повече от сто лири. Тя ми донесе вест, че нямали такива пари на разположение; аз казах, че биха могли да си заложат къщата в заложната къща\footnote{loan office}. Отговорът на това дойде след няколко дни и беше, че не одобряват годежа; че от Брадфърд са разбрали, че печатарският бизнес не е много доходен; че скоро буквите ще се изтрият и ще има нужда от нови; че С. Каймер и Д. Хари са се провалили един след друг, и че вероятно скоро ще ги последвам; и по тази причина не съм добре дошъл в къщата им и ми е забранено да се виждам с дъщерята.

Дали те наистина си бяха променили мнението, или това беше просто хитрина, основана на предположението, че сме твърде влюбени, за да се откажем, и че по тази причина ще избягаме, което ще им позволи да дадат или задържат колкото пожелаят, не знам; но подозирах, че е второто, огорчих се, и повече не отидох. Г-жа Годфри по-късно ми донесе по-благосклонни вести относно тяхното благоразположение и искаше пак да ме въвлече; но аз заявих твърдото си решение да не се занимавам повече с това семейство. Това огорчи семейство Годфри; не постигнахме съгласие и те се преместиха, оставяйки ми цялата къща, и аз реших да не взимам повече наематели.

След като тази случка насочи мислите ми към женитба се огледах наоколо и се опитах да се представя и да започна познанства на други места; но скоро установих, че понеже като цяло печатарският бизнес се смяташе за недоходоносен, не трябваше да очаквам пари с жената, освен с такава жена, която с друго не би ми се харесала. А междувременно трудната за обуздаване младежка страст често ме вкарваше в интриги с долни жени, които попадаха на пътя ми, и които интриги бяха придружени с известни разходи и големи неудобства, като оставим настрана, че представляваха постоянен риск за здравето ми с една болест, от която се страхувах повече от всичко, макар че по чуден късмет я избегнах. Бях продължил да поддържам приятелски контакти с г-жа Рийд и семейството й, които имаха добро мнение за мен от времето, когато им бях наемател. Често ме канеха у тях и ме търсеха за съвет, като понякога успявах да съм им от полза. Изпитвах съжалиение към незавидното положение на г-ца Рийд, която беше като цяло потисната, рядко жизнерадостна и избягваше компания. Смятах, че нещастието ѝ е резултат до голяма степен на лекомислието и непостоянството, които проявих, докато бях в Лондон, въпреки че майка ѝ беше така добра да мисли, че вината е нейна, тъй като не ни беше позволила да се оженим преди да замина, и беше подкрепила другата страна в мое отсъствие. Взаимната ни обич се възобнови, но сега пред женитбата ни имаше големи пречки. Действително повечето хора смятаха бракът ѝ за невалиден, тъй като се говореше, че той имал друга жена в Англия; но поради разстоянието, това не беше лесно да се докаже; и въпреки че имаше вести за смъртта му, не бяха сигурни. Освен това, дори наистина да беше мъртъв, беше оставил много дългове, които може би щяха да бъдат потърсени от наслендика му. Въпреки всички трудности, решихме да рискуваме и я взех за жена на първи септември 1730 г. Нито едно от опасенията ни не се осъществи/сбъдна, тя се показа като добър помощник, и ме подкрепи много като наглеждаше магазина; заедно преуспяхме, и винаги сме се опитвали взаимно да се правим щастливи. Така поправих грешката си доколкото можах. 

По горе-долу това време, на една от срещите на клуба ни – не в кръчма, а в една малка стая на г-н Грейс, отделена за тази цел – предположих, че след като често в разискванията, които водехме по въпросите, правехме отпратки към книгите си, може би ще е полезно всичките да са ни под ръка по време на срещите ни, така че да можем да се консултираме с тях, когато стане нужда; и като съберем по този начин книгите си в обща библиотека, бихме могли – докато ни харесва да ги държим така – да се възползваме от книгите на всички останали членове, което би било почти толкова полезно, колкото ако всеки притежаваше всичките. Предложението беше харесано и прието, така че запълнихме единия край на стаята с такива книги, каквито можахме да отделим. Броят не беше голям колкото очакванията; и въпреки, че бяха от голяма полза, поради известни неприятности, които се случиха поради липса на достатъчно внимателно отношение към тях, след около година колекцията беше разтрогната и всеки си прибра книгите у дома. 

Тогава захванах първия си проект с обществен характер, библиотека с абонамент\footnote{subscription library}. Списах предложенията, дадох ги на Брокдън\footnote{Brockden} – нашия  велик нотариус – да ги оформи, и с помощта на приятелите ми от Хунтото набрах петдесет абоната по четиридесет шилинга за начало, и по десет шилинга на година за петдесет години напред, което беше срокът на съществуване на нашата компания. По-късно компанията ни нарастна до сто човека и се сдобихме с устав: това беше първообраза на всички северноамерикански абонаментни библиотеки, които сега са толкова многобройни. А тя самата е придобила внушителни размери, и продължава да расте. Тези библиотеки са подобрили общата култура на американците и са направили обикновените търговци и фермери толкова интелигентни, колкото повечето джентълмени в други страни, и може би донякъде са допринесли към съпротивата, която колониите като цяло оказват в защита на привилегиите си.

Бележка. Дотук написано с намерението изразено в началото, и поради това съдържа няколко малки семейни анекдота, които са без значение за другите. Следващите страници са написани много години по-късно, в отговор на съветите от писмата (по-долу) и съответно са предназначени за широка публика. Прекъсването беше причинено от събития свързани с Революцията.

\chapter{Част втора}

\textbf{Писмо от г-н Абел Джеймс, получено с „Бележки за моя живот”}\\
(получено в Париж)

“Mили и почитаеми приятелю: много пъти имах желание да ти пиша, но мисълта, че писмото може да попадне в ръцете на британците, ме възпираше, да не би някой печатар или клюкар да не публикува част от съдържанието и да нарани приятеля ни и така да си заслужа порицание. 

За моя голяма радост, преди известно време ми попаднаха около двадесет и три листа писани от твоята собствена ръка и съдържащи разказ за семейството и живота ти до 1730 година, в които имаше и бележки, също писани от теб. Прилагам копие от тях, с надеждата, че ако си продължил разказа си до по-късен период, това може да послужи да бъдат събрани първата и втората части. А ако още не си написал продължението, надявам се че няма да отлагаш. Както знаем от проповедника, животът е несигурен; и какво ще каже светът, ако милият, човечен и щедър Бен Франклин лиши приятелите си и света от такава полезна и приятна творба; творба, която ще е полезна и забавна не на малък брой хора, а на милиони? Съчиненията от този род оказват много голямо влияние върху умовете на младите и никъде не е било по-голямо отколкото в случая с дневниците на нашия известен приятел. Те почти неусетно подтикват младите да решат да опитат да станат толкова добри и изтъкнати, колкото разказвача. Ако например, когато бъде публикуван, твоят разказ накара младежта (а аз не се съмявам, че ще успее да го направи) да бъде толкова трудолюбива и сдържана, колкото ти си бил на младини, каква благословия би била тази творба! Не познавам друг жив човек, нито дори мнозина взети заедно, които да могат до такава степен да насърчат духа на трудолюбие, ранно внимание към работата, пестеливост и въздържание у младите американци. Не че си мисля, че творбата не би имала други достойнства или не би принесла друга полза, съвсем не; но първото е толкова важно, че не се сещам за друго, което да се приближава по важност.

След като показах предното писмо и съпътстващите го бележки на един приятел, получих от него следното:

\textbf{Писмо от г-н Бенджамин Вогън} \\
Париж, 31 януари, 1783 година


НАЙ-МИЛИ ГОСПОДИНЕ, след като прочетох листовете с бележките с основните събития от живота ви, които ви бяха изпратени от вашия познат квакер, ви казах, че ще ви пиша относнo причините, поради които смятам, че ще бъде полезно да бъде завършен и публикуван разказът, каквото беше неговото желание. Известно време различни обстоятелства ми пречеха да напиша това писмо, и не знам дали си е струвало чакането; но понеже така се случи, че в момента съм свободен, чрез писането поне ще прекарам времето си с нещо интересно и поучително; но тъй като изразите, които съм склонен да използвам могат да обидят човек с вашите маниери, само ще ви кажа как бих се обърнал към кой да е друг човек добър и велик колкото вас, но не толова скромен. На такъв човек бих казал, Господине, изисквам историята на живота ви със следните мотиви: Историята ви е толкова забележителна, че ако вие не ни я предоставите, някой друг със сигурност ще го направи; и може би с такъв успех, че да причини толкова вреда, колкото полза би донесло вашето лично ръководство на тази работа. Освен това историята ви ще бъде един прозорец към вътрешното положение в страната ви, който в голяма степен ще привлече заселници с добродетелни и мъжествени умове. И като вземем предвид с какво усърдие хората търсят такава информация, както и колко надалеч се носи славата ви, не мога да си представя по-ефективна реклама от вашата биография. Всичко, което ви се е случило също така е тясно свързано с обичаите и положението на един народ във възход; и в това отношение не мисля, че писанията на Цезар или Тацит могат да бъдат по-интересни за човек, който истински се интересува от човешката природа и обществото. Но тези причини, господине, са незначителни в сравнение с възможностите, която вашата история ще отвори за създаването на бъдещи велики личности; и – в съчетание с вашата книга „Изкуството на добродетелта“ (която смятате да издадете) – за подобряване на чертите на характера на много хора, и съответно за увеличаване на щастието, както в обществото, така и у дома. По-точно, двете творби, които визирам, господине, ще се превърнат в благородно правило и ще послужат за пример за самоусъвършенстване. Училищата и други образователни институции постоянно работят на базата на кухи принципи и предлагат непохватна система, насочена към куха цел; но вашата система е проста, а целта – истинска; и докато родителите и младите хора не разполагат с други честни средства да оценят и да се подготвят за разумен път в живота, вашето откритие, че задачата е по силите на много хора, ще бъде безценно! Влиянието, оказано върху характера късно в живота, е не само влияние оказано късно в живота, но и слабо влияние. На младини придобиваме основните си навици и предразсъдъци; на младини взимаме решения свързани с професията, интересите и женитбата си. Младостта, следователно, е решаваща; дори образованието на следващото поколение се решава на младини; в младостта се формират обществения и личния характер; и тъй като животът върви от младост към старост, животът трябва да започне добре още от младостта, и особено преди да вземем решения относно най-важните неща. Но вашата биография не само ще учи на самоусъвършенстване, а на самоусъвършенстването на един мъдър човек; и най-мъдрият ще получи просветление и ще подобри напредъка си, ако види подробно описано поведението на друг мъдър човек. А и защо да лишаваме по-слабите от такава помощ, когато виждаме, че в това отношение расата ни от край време се лута в тъмнина, почти без водач? Затова, господине, покажете на синовете и на бащите, колко много може да се направи; и поканете всички мъдри хора да станат като вас, а всички други да бъдат мъдри. Като виждаме колко жестоко могат да се отнасят държавниците и воините към човечеството, и колко абсурдно могат да се държат изтъкнатите хора с познатите си, ще бъде поучително да се види увеличение в броя на хората с кротки, скромни обноски; и да стане ясно колко съвместимо е величието с непревзетото отношение и завидните постижения – с доброто настроение.

Малките лични преживявания, за които ще разкажете, също ще принесат значителна полза, тъй като преди всичко имаме нужда от правила за благоразумие в обичайните ни дела; и ще бъде любопитно да се види как вие сте се справял с тях. Ще бъде нещо като ключ към живота и ще разясни много неща, които би трябвало да бъдат обяснени по веднъж на всички хора, за да имат те възможност да станат мъдри чрез предвидливост. Нещото, най-сходно със собствения опит, е това да видим чуждият опит представен по интересен начин; вашето перо със сигурност ще постигне това; нашите работи и решения ще придобият дух на простота и важност, който със сигурност ще впечатли; и съм сигурен, че сте водил работите си по толкова оригинален начин, като че става дума за политическа или философска беседа; и кое е нещото, което повече от живота заслужава експерименти и систематичен подход (като вземем предвид важността и грешките му)?

Някои хора са били добродетелни на сляпо, други са правили фантастични преувеличения, трети са използвали разума си за лоши цели, но съм сигурен, че вие, господине, ще напишете с ръката си само неща, които са едновременно мъдри, практични и добри, вашият разказ за вас самия (защото приемам, че паралелът, който правя с г-н Франклин е валиден не само по отношение на характера, а и на личната история) ще покаже, че не се срамувате от никой произход; нещо, което е още по-важно, тъй като показвате колко малка роля играе произходът при постигането на щастие, добродетел и величие. Също така, тъй като никоя цел не бива постигната без употребата на средства, ще открием, господине, че вие сам сте съставил плана, чрез който станахте значим; но в същото време можем да видим, че макар и резултатът да е превъзходен, средствата са толкова прости, колкото мъдростта може да ги направи; тоест, такива, които зависят от природата на нещата, добродетелта, мисълта и навика. Друго нещо, което вашият разказ ще покаже, е това колко удачно е всеки човек да чака момента, в който ще се появи на световната сцена. Тъй като сетивата ни са до голяма степен фокусирани върху настоящия момент, сме склонни да забравяме, че ни предстоят други моменти след настоящия, и съответно, че човек трябва да избира поведение което приляга на животa в неговата цялост. Изглежда, че вашият подход е бил приложен към живота ви и отделните негови моменти са били изпълнени със съдържание и удоволствие, вместо да бъдат измъчвани от глупаво нетърпение или съжаления. Този вид поведение е лесно за тези, които опитват да постигнат добродетел чрез примера на други велики мъже, за които търпението често е характерно. Вашият събеседник квакер, господине (тук отново ще предположа, че адресатът на писмото ми прилича на д-р Франклин), ви хвали за пестеливостта, усърдието и умереността ви, за които смята, че са пример за подражание за всички млади хора; но намирам за учудващо, че е забравил вашите скромност и себеотдайност, без които никога не бихте могли да дочакате издигането си или да се чувствате удобно преди това; което е голям урок илюстриращ нищетата на славата и колко е важно да регулираме умовете си. Ако този ваш събеседник познаваше репутацията ви толкова добре колкото аз я познавам, щеше да ви каже, Вашите предишни писания и мерки ще привлекат внимание към биографията ви и „Изкуството на добродетелта“, а от друга страна вашите биография и „Изкуство на добродетелта“ ще привлекат внимание към тях. Това е едно предимство, което съпътства многостранния характер, и което увеличава ефекта на всичко свързано с него; и е толкова по-полезно, доколкото може да се предположи, че повече хора имат нужда от средства да подорбят умовете и характерите си, отколкото имат нужда от време и желание. Но има и още една заключителна причина, господине, която ще покаже ползата от записването на вашия живот като проста биография. Този род писания изглежда малко е загубил популярност, и въпреки това е много полезен; и вашият разказ може да е особено полезен, тъй като ще позволи сравнение с животите на различни обществени главорези и интриганти, и с абсурдни монашески самомъчители или суетни литературни дребнавци. Ако насърчи писането на други съчинения подобни на вашето и накара повече хора да живеят живот, който заслужава да бъде описан, ще бъде по-ценно от всичките животописи на Плутарх взети заедно. Мили д-р Франклин, тъй като се уморих да си представям въображаем образ, който във всяко отношение съответства на един единствен човек на света, без да му отдам заслужената похвала, ще приключа писмото си с лично обръщение към вас. Искрено желая, мили господине, да откриете на света чертите на истинския си характер, тъй като обществените брожения (става дума за Американската война за независимост, 1775 – 1783 г., б.пр.) биха могли иначе да го засенчат или очернят. Като вземем предвид голямата ви възраст, предпазливия ви характер и особеният ви начин на мислене, изглежда малко вероятно предположението, че някой друг може да е достатъчно запознат с фактите от живота ви или с намеренията на ума ви. Освен всичко това, настоящата голяма революция непременно ще насочи вниманието ни към автора й, и тъй като се твърди, че принципите й са добродетелни, ще бъде много важно да се покаже, че такива принципи в действителност са й повлияли; и тъй като вашият характер е първият, който ще бъде подложен на подробно обследване, е удачно (дори само заради резултата, който ще произведе върху  огромната ви и надигаща се страна, както и върху Англия и Европа) той да остане завинаги неопетнен. Винаги съм твърдял, че за да се увеличава човешкото щастие е нужно да се докаже, че дори в настоящия момент човек не е зло и отвратително животно; и още повече, да се докаже, че добрите напътствия могат да го направят много по-добър; и до голяма степен по същата причина с нетърпение очаквам да видя утвърдено мнението, че сред човешката раса има почтени хора; иначе, в момента, в който сметнем всички за безнадеждни, добрите хора ще спрат да правят усилия, смятани за безсмислени и може би ще почнат да мислят как да докопат своя дял в битката на живота, или поне как да направят живота удобен най-вече за себе си. Затова, мили господине, захванете се с тази работа възможно най-бързо: покажете се толкова добър, колкото сте добър; толкова въздържан, колкото сте въздържан; и преди всичко, докажете, че сте човек, който от детството си обича правдата, свободата и съгласието concord, по начин, който показва, че действията ви от последните седемнадесет години от живота ви са естествени и последоветални. Нека англичаните бъдат принудени не само да ви уважават, а дори да ви обичат. Ако започнат да мислят добро за хора от вашата страна, ще се приближат до това да мислят добро за страната ви; а когато вашите сънародници видят, че англичаните мислят добро за тях, това ще ги приближи до това да мислят добро за Англия. Разширете хоризонтите си още повече; не се ограничавайте с англоговорящите, но след като вече сте се погрижил за толкова много въпроси от политиката и природата, помислете за подобряването на цялото човечество. Тъй като не съм чел нищо от живота, за който говорим, а само познавам човека, който го е изживял, говоря донякъде наслуки. Въпреки това съм сигурен, че животът и трактатът, който споменах (върху "Изкуството на добродетелта"), непременно ще отговорят на основните ми очаквания; още повече, ако вземете мерки да нагодите тези творби към по-горе изразените възгледи. Но дори да се провалят във всичко на което един ваш въодушевен почитател се надява, поне ще сте написал съчинения, които ще занимават човешкия ум; а който намери начин да достави невинно удоволствие на хората, е добавил нещо към светлата страна на живота, който иначе е твърде помрачен от тревога и твърде наранен от болка. Затова, с надежда, че ще се вслушате в молитвата отправена към вас в това писмо, мили господине, моля да се нарека ваш, и т.н.

Подпис, Бендж. Вогън


\textbf{\textit{Продължение на "Равносметка за живота ми"}}\\
\textit{Започнато в Паси, близо до Париж}\\
\textit{1784}

Мина известно време откакто получих горните писма, но досега бях все твърде зает, за да изпълня молбата, която съдържат. Освен това, най-вероятно щях да свърша тази работа по-добре, ако си бях у дома, сред книжата ми, които щяха да съдействат на паметта ми и да ми помогнат да си спомня някои дати; но тъй като не е ясно кога ще се върна, и понеже сега имам малко свободно време, ще се опитам да си спомня и да запиша каквото мога; ако доживея да се прибера у дома, там ще може да се нанесат корекции и подобрения. 

Понеже нямам под ръка копие от вече написаното, не знам дали е описан начинът, по който основах обществената библиотека на Филаделфия, която започвайки със скромно начало сега е станала толкова значима, въпреки че си спомням, че бях стигнал до момента на тези събития (1730 г.) Затова тук ще започна с разказ за това, а ако по-късно се окаже, че вече съм разказал, това може да се махне. 

По времето, когато се установих в Пенсилвания, в нито една от колониите южно от Бостън нямаше добра книжарница. В Ню Йорк и Филаделфия печатарите в действителност продаваха канцеларски материали; продаваха само хартия и т.н., алманаси, балади, както и някои по-прости учебници. Тези, които обичаха четенето, трябваше да си поръчват книги от Англия; всеки от членовете на Хунтото имаше по няколко. Бяхме напуснали кръчмата, където първоначално се срещахме, и бяхме наели стая за сбирките на клуба. Предложих всички да донесем книгите си в стаята, където не само щяха да са ни полезни за справка по време на срещите ни, а щяха да се превърнат в общо благо, тъй като всеки щеше да може да заема това, което иска да чете, у дома. Предложението ми беше прието и за известно време това ни задоволи.

Виждайки преимуществата на тази малка колекция, предложих да направим ползата от книгите по-широко достъпна като започнем обществена библиотека с абонамент. Нахвърлях плана и правилата, които щяха да са нужни, и намерих умел нотариус – г-н Чарлз Брокдън – който да оформи всичко като договорно споразумение за абонамент, според което всеки абонат се задължаваше да плати определена сума в аванс за първоначалната покупка на книги, а след това да допринася с ежегодна вноска за увеличаване на колекцията. Толкова малко читатели имаше по това време във Филаделфия – и повечето от нас толкова бедни – че не успях дори с много усилия да намеря повече от петдесет човека, най-вече млади търговци, които да се съгласят да платят за тази цел петдесет шилинга в аванс и по десет всяка година. С тези малки суми започнахме. Книгите бяха внесени; библиотеката беше отворена един ден в седмицата за заемане на книги срещу писмено обещание, че ако не бъде върната, книгата ще бъде заплатена в двоен размер. Ползата от организацията скоро пролича, и беше имитирана в други градове и провинции. Библиотеките се допълваха и посредством дарения; четеното стана модерно; и като нямаха други обществени забавления, които да ги отклоняват от ученето, нашите хора се запознаха по-отблизо с книгите, и след няколко годнини чужденицте почнаха да отбелязват, че са като цяло по-добре образовани и по-интелигентни отколкото хора от същото потекло в други страни.

Когато дойде време да подпишем гореспоменатото споразумение, което щеше да обвърже нас и наследниците ни и т.н. за срок от петдесет години, г-н Брокдън, нотариусът, ни каза, „Вие сте млади хора, но е малко вероятно някой от вас да доживее до изтичането на срока определен в документа.” Въпреки това, някои от нас все още са живи; но няколко години по-късно документът беше анулиран от устав, с който се създаде компания с безсрочно действие.

Отказите и нежеланието, с които се срещнах, докато набирах абонати, бързо ме накараха да почувствам неуместността на това човек да се представя за двигател на какъвто и да е полезен проект, за който би могло да се предположи, че ще повдигне репутацията му над тази на съседите му, когато има нужда от помощта им, за да го осъществи. Затова се покрих доколкото можах и го представих като идея на няколко приятели, които ме бяха помолили да обиколя и да го предложа на тези, за които смятат, че обичат да четат. По този начин цялата работа потръгна по-гладко и в подобни случаи винаги съм прилагал същия подход; и поради честите ми успехи мога сърдечно да го препоръчам. Временната малка жертва на честолюбие по-късно ще ще бъде възнаградена многократно. Ако известно време остане неясно на кого е заслугата, някой по-честолюбив от теб ще пожелае да си я припише. А тогава дори завистта ще е склонна да се отнесе справедливо към теб, като свали тези приписани заслуги, за да окичи с тях истинския им притежател.

Библиотеката ми позволи да се развивам чрез постоянно учене, за което отделях по час или два на ден, и така компенсирах донякъде загубата на образованието, което баща ми едно време беше предвидил за мен. Четенето беше единственото забавление, което си позволявах. Не прекарвах време по кръчми, в игри или веселби от какъвто и да е род; и усърдието ми в работата продължаваше толкова неуморно, колкото беше нужно. Дължах заеми за печатницата си; имах младо семейство, за чието образование трябваше да се погрижа, и трябваше да се състезавам за работа с двама печатари, които бяха установени в града преди мен. Положението ми обаче всеки ден ставаше по-добро. Продължавах по навик да съм пестелив. А понеже когато бях дете, баща ми, наред с другите поучения, с които ме наставляваше, често ми беше повтарял една от притчите на Соломон – „Видял ли си човек пъргав в работата си? Той ще стои пред царе, няма да стои пред прости“ – смятах трудолюбието за средство за придобиване на богатство и издигане в обществото. Макар че не мислех, че някога наистина ще стоя пред царе, което обаче се случи; защото от тогава съм стоял пред петима, а дори имах честта да седна да вечерям с един – краля на Дания. 

Имаме една английска поговорка, която гласи: „Който иска да преуспее, трябва да говори с жена си.“ Бях късметлия, че моята излезе толкова склонна към трудолюбие и спестовничество, колкото мен самия. Тя ми помагаше с радост в работата, сгъваше и шиеше памфлети, грижеше се за магазина, купуваше стари ленени парцали за 

Имаме една английска поговорка, която гласи: „Който иска да преуспее, трябва да говори с жена си.“ Бях късметлия, че моята излезе толкова склонна към трудолюбие и спестовничество, колкото мен самия. Тя ми помагаше с радост в работата, сгъваше и шиеше памфлети, грижеше се за магазина, купуваше стари ленени парцали за производителите на хартия и т.н., и т.н. Не държахме мързеливи слуги, трапезата ни беше проста, мебелите ни – най-евтините. Например дълго време закуската ми се състоеше от хляб и мляко (никакъв чай), и ядях с калаена лъжица от  глинена купичка за два пенса. Но забележи как луксът се промъква в семейството и напредва, въпреки принципите: една сутрин, след като бях извикан за закуска, я намерих в порцеланова купа със сребърна лъжица! Жена ми ги беше купила за мен без да знам, и това й беше струвало огромната сума от двадесет и три шилинга, за което тя нямаше друго оправдание освен това, че мъжът й заслужава сребърна лъжица и порцеланова купа не по-малко от съседите си. Това беше първата поява на сребро и порцелан вкъщи, които оттогава, с увеличаването на богатството ни, малко по малко достигнаха до стойност от няколко стотин лири. 

Религиозното ми образование беше презвитерианско; и въпреки че някои от догмите на тази деноминация – като например вечните заповеди на Бога, избор, предопределение – ми изглеждаха неразбираеми, други съмнителни, и въпреки че рано спрях да посещавам обществените сбирки на сектата, понеже неделя беше денят ми за учене, никога не съм бил без религиозни принципи. Никога например не съм се съмнявал в съществуването на Бога; който е създал свете и го управлява чрез Провидението си; че най-богоугодната служба е човек да прави добри дела на другите; че душите ни са безсмъртни; и че всяко престъпление ще бъде наказано, а добродетелта възнаградена, или тук, или в отвъдното. За мен това бяха основите на всяка религия; и тъй като бяха част от всички религии разпространени в нашата страна, аз ги уважавах всичките, макар и с различни степени на уважение, понеже ги намирах в малка или голяма степен примесени с други идеи, които нямаха склонност да ни вдъхновяват, да насърчават или утвърждават нравствеността, а служеха преди всичко да ни разделят и да пораждат вражди между нас. Това уважение към всички, подплатено с мнението, че дори най-лошите принасят някаква полза, ме караше да избягвам всякакви разговори, които биха могли да накърнят доброто мнение на човек за вероизповеданието му; и тъй като населението на провинцията ни се увеличаваше и постоянно имаше нужда от нови храмове, които като цяло се издигаха благодарение на дарения, никога не съм отказвал да подкрепя тези градежи, независимо от деноминацията.  

Макар че рядко участвах в общественото богослужение, все пак имах мнение за уместността му и за спобността му да е полезно, когато е проведено правилно, и редовно плащах годишния си принос в подкрепа на единственият Презвитериански пастор или събрание във Филаделфия. Той понякога ме посещаваше като приятел и ме увещаваше да идвам на служението му, и понякога успяваше да ме накара, веднъж за пет поредни недели. Ако смятах, че е добър проповедник може би щях да продължа, независимо от възможността да използвам свободното си неделно време за учене; но посланията му бяха предимно или полемични спорове или обяснения върху особените доктрини на нашата деноминация, и всичките ми се струваха сухи, безинтересни и неградивни, тъй като не втълпяваха нито налагаха дори един нравствен принцип, като целта им сякаш беше да ни направят по-скоро добри презвитерианци, отколкото добри граждани.

Накрая избра за тема на проповедта си този стих от четвъртата глава на Филипяни: „Прочее, братя мои, за това, що е истинно, що е честно, що е справедливо, що е чисто, що е любезно, що е достославно, за това, що е добродетел, що е похвала, - само за него мислете.“ Предположих, че в проповед по този текст не може да пропусне да говори за нравствеността. Но той се ограничи до само пет точки, които имал предвид апостолът, а именно: 1. да пазим съботата свята. 2. да сме усърдни в четенето на Светото писание. 3. да присъстваме на богослужение. 4. да участваме в причастието. 5. да уважаваме както подобава божиите служители. Всичките тези неща може да са добри, но понеже не бяха добрите неща, които очаквах от текста, загубих надежда да ги чуя при който и да е друг текст, отвратих се и повече не посетих проповедите му. Няколко години преди това бях съчинил малка литургия или вид молитва за лично ползване (по-точно в 1728 г.), наречена Елементи на вяра и религиозни деяния. Започнах пак да я използвам и повече не посетих общите събрания. Поведението ми може би заслужава порицание, но го оставям така, без повече да се опитвам да се извинявам, тъй като целта ми в момента е да представя фактите, а не да се извинявам за тях.

По горе долу това време се зароди в мен мисълта за дръзкия и усилен проект да постигна нравствено съвършенство. Исках да живея без никога да правя каквито и да е грешки; щях да се преборя с всичко, в което природните склонности, навика или компанията биха ме въвлекли. При положение, че знаех, или си мислех, че знам, какво е добро и какво зло, не виждах защо да не е възможно винаги да правя едното и да избягвам другото. Но скоро открих, че съм се захванал със задача по-трудна отколкото си бях представял. Докато вниманието ми беше заето с това да се пазя от един порок, често бивах изненадван от друг; навикът се възползваше от липсата на внимание; склонноста понякога се оказваше по-силна от разума. Накрая заключих, че основаното на спекулации убеждение, че е в наш интерес да сме съвършено добродетелни, само по себе си не е достатъчно да ни предпази от това да грешим; и че противните навици трябва да бъдат елиминирани, а добрите да се придобиват и установяват, преди да можем да разчитаме на постоянна и неизменна праведност в поведението си. По тази причина измислих следния метод за целта. 

В разните списъци с нравствени добродетели, с които се бях запознал в четенията си, намерих голяма разлика в броя на добродетелите, тъй като различните автори включваха повече или по-малко идеи под едно и също име. Въздържанието, например, според някои се ограничаваше до яденето и пиенето, а според други означаваше упражняване на умереност във всяко друго удоволствие, апетит, склонност или страст, телесна или умствена, или дори в нашите алчност и амбиция. Казах си, че за по-голяма яснота е добре да използвам повече имена свързани с по-малко идеи, отколкото малко на брой имена с повече идеи; и събрах всичко, което смятах за нужно или желателно под имената на тринадесет добродетели, като към всяка приложих кратко поучение, което напълно изразяваше смисъла, който й придавах. 

Имената на добродетелите, заедно с описанията им, бяха: 

1. ВЪЗДЪРЖАНИЕ. Не яж до преяждане и не пий до напиване.

2. МЪЛЧАНИЕ. Не говори, освен такива неща, които биха могли да са от полза на другите или на теб; избягвай празните приказки. 

3. РЕД. Нека всичките ти неща да си имат място; нека всяка част от работата ти да има определено време. 

4. РЕШИТЕЛНОСТ. Решавай се да правиш това, което трябва; непременно изпълявай това, което си решил.

5. ПЕСТЕЛИВОСТ. Не харчи пари освен за да направиш добро на други или на себе си; т.е. не прахосвай нищо.

6. УСЪРДИЕ. Не губи време; винаги се занимавай с нещо полезно; изхвърли всички безполезнин действия.

7. ИСКРЕНОСТ. Не използвай измама във вреда на другиго; мислите ти да бъдат невинни и справедливи, и, ако говориш, говори както подобава (на мислите ти).  

8. СПРАВЕДЛИВОСТ. Не причинявай никому неправда чрез обида или като пропуснеш да отдадеш дължимите ползи.

9. УМЕРЕНОСТ. Избягвай крайностите; въздържай се от това да негодуваш толкова срещу обидите, колкото мислиш, че трябва.

10. ЧИСТОТА. Не допускай нечистота в тялото, дрехите или жилището си.

11. СПОКОЙСТВИЕ. Не оставяй дреболий или обикновени или неизбежни произшествия да те смущават.

12. ЦЕЛОМЪДРИЕ. Не прави секс освен за здраве или поколение и никога до затъпяване, слабост или във вреда на нечий мир или репутация. 

13. СМИРЕНИЕ. Имитирай Иисус и Сократ.

Понеже намерението ми беше да привикна в упражняването на всички тези добродетели, сметнах, че вместо да се опитам да ги придобия всичките наведнъж, което щеше да разсее вниманието ми, е по-добре да се фокусирам върху тях една по една; и когато овладея една, да премина към следващата и така нататък, докато не покрия всички тринадесет; и тъй като придобиването на някои от тях би могло да помогне за придобиването на други, ги подредих в реда, в който са поместени по-горе. Поставих въздържанието на първо място, тъй като то има склонност да осигурява тези свежест и яснота на ума, които са толкова необходими, когато човек трябва да бди и да стои на пост срещу неотслабващото привличане на стари навици и силата на постоянни съблазни. Щом то веднъж бъде придобито, Мълчанието ще дойде по-лесно; поставих го на второ място тъй като желанието ми беше да увеличавам знанията си, докато напредвам в добродетелността, и понеже в разговор знанието се придобива по-скоро чрез употребата на ушите, отколкото с езика, поради което исках да се отърва от навика да дърдоря, да си играя с думите и да се шегувам, който бях започнал да придобивам, което беше приемливо само в несериозна компания. Очаквах, че Мълчанието и следващата добродетел – Реда – ще ми позволят да намеря повече време за проекта ми и за учене. Веднъж превърнала се в навик, Решителността би ме утвърдила в опитите ми да придобия всички останали добродетели; като ме освободят от оставащия ми дълг, и като увеличат богатството и независимостта, Пестеливостта и Усърдието ще улеснят упражняването на Искреност и Справедливост и т.н., и т.н. Съзнавайки че съгласно съвета на Питагор от неговите Златни стихове ще е нужно ежедневно изпитване, измислих следния начин за провеждане на изпита.

Направих малка книжка, в която отделих по една страница за всяка от добродетелите. Разграфих всяка страница с червено мастило, така че да има по седем колони, една за всеки ден от седмицата, като отбелязах всяка колона с буква за деня. Разделих колоните с тринадесет червени линии, отбелязвайки началото на всеки ред с първата буква на една от добродетелите, за да мога на този ред, в съответната колона, да отбелязвам с малка черна точка всяка грешка, която при изпита установявах, че съм направил спрямо тази добродетел в този ден.

\begin{figure}
  \centering
  \includegraphics[width=0.8\textwidth]{{notebook}.png}\\
  {Вид на страниците.}
\end{figure}

Реших да насоча вниманието си изцяло към всяка една от добродетелите последователно за по седмица. Така през първата седмица положих най-големи усилия да не допусна дори най-малкото прегрешение срещу Умереността, оставяйки останалите добродетели да следват нормалния си ход, като само отбелязвах грешките от деня. Така ако през първата седмица успеех да запазя първия ред – отбелязан с У. –  без точки, приемах, че толкова съм привикнал към тази добродетел, и толкова отвикнал от противоположия й порок, че мога през следващата седмица да насоча вниманието си и към следващата добродетел и да се опитам да запазя и двата реда без точки. Продължавайки по този начин, можех да изкарам пълен курс за тринадесет седмици и четири курса на година. И като човек, който има градина, която има нужда да бъде оплевена, който не опитва да изчисти всички бурени наведнъж, защото няма да му е по силите и обхвата, а работи леха по леха, и след като е изчистил първата, преминава към втората, така и аз се надявах, че ще мога да изпитам радостта от насърчителния си напредък в добродетелта, изчиствайки последователно редовете от техните точки, докато накрая, след няколко курса, не изпитам удоволствието да видя чиста книжка след тринадесет седмици ежедневно изпитване. 

За девиз на малката ми книжка служеха следните редове от „Катон“ на Адисън (става дума за пиесата „Катон: трагедия“ на Джоузеф Адисън за Марк Проций Катон, наричан още Катон Млади, б.пр.):

    “Тук ще стоя. Ако над нас има Сила
    (а че такава има, цялата Природа се провиква
    чрез всички свои Дела.), тя трябва да се наслаждава на добродетелта.
    И това, в което тя се наслаждава, трябва да е щастливо.“ 

Друг от Цицерон\footnote{Марк Тулий Цицерон б.пр.}: 

          "O vitae Philosophia dux! O virtutum indagatrix
          expultrixque vitiorum! Unus dies, bene et ex praeceptis
          tuis actus, peccanti immortalitati est anteponendus."

(„О, Философийо, водач във живота! О, търсач на добродетел 
и враг на порока! Ти ни учиш, че един ден добро е за предпочитане
пред цяла вечност грях.“)

Друг, от Притчите на Соломон (Книга Притчи Соломонови от Библията, б.пр.), който говори за мъдростта или добродетелта:

В десницата й е дългоденствие, а в левицата й - богатство и слава; пътищата й са приятни пътища, и всичките й пътеки – мирни. (3:16-17)

И понеже смятах, че Бог е изворът на мъдростта, намерих за добро и нужно да помоля за неговото съдействие за придобиването й; за тази цел създадох следната малка молитва, която беше поместена преди таблиците за самоизпитване, за ежедневна употреба:

“О мощна Доброта! изобилен Татко! милостиви Водителю! Увеличи в мен тази мъдрост, която открива най-истинския ми интерес! Заздрави у мен решителността да изпълнявам това, което мъдростта повелява. Приеми сърдечните услуги, които извършвам за останалите твои деца, като единствения начин, по който мога да ти се отплатя за продължаващото ти благоволение към мен.“

Понякога също така използвах една малка молитва, която взех от стихотворенията на Томсън (Джеймс Томсън, английски поет, 1700-1748, б.пр.), а именно:

    “Татко на светлина и живот, ти Най-висше Благо!
    Научи ме какво е добро; научи ме лично!
    Предпази ме от глупост, суета и порок,
    от всяко долно занимание; и изпълни душата ми
    със знание, съзнателен мир и чиста добродетел;
    свещено, изпълващо, неизбледняващо блаженство!

Понеже предписанието на Реда повеляваше за всяка част от работата ми да има определено време, една от страниците на книжката ми съдържаше следният план за заниманията през двадесет и четирите часа на естествения ден:

\begin{figure}[h]
  \centering
  \includegraphics[width=0.8\textwidth]{{schedule}.png}
\end{figure}

Захванах този план за самоизпитване и продължих да го следвам известно време с малки прекъсвания. С изненада установих, че имах доста повече недостатъци, отколкото си мислех; но имах и удоволствието да видя как намаляват. За да си спестя нуждата да подновявам от време на време малката си книжка, която беше станала цялата на дупки, защото за да я подготвя за нов курс изстъргвах от хартията точките обозначаващи стари прегрешения, пренесох таблиците и поученията върху гланцираните листа на един бележник, в който линиите бяха отбелязани с червено мастило, което оставяше трайна следа, а прегрешенията си отбелязвах с черен молив, който лесно можех да изтрия с мокра гъба. След време минавах само по един курс на година, а после по един на няколко години, докато накрая спрях съвсем, тъй като бях зает с пътувания и работа в чужбина и с много различни дела, които не позволяваха; но винаги носех малката книжка със себе си.

Програмата ми за РЕД ми създаваше най-много проблеми; и открих, че макар и да е изпълнима за хора, чиято работа им позволява да разполагат с времето си, като калфа в печатница например, майстор не е възможно да я следва точно, защото трябва да се меша със света и често да приема хора по работа по време удобно за тях. Също така намерих за крайно трудно да се приуча на ред по отношение на местата на неща, книжа и т.н. Не бях навикнал на това от дете, и понеже имах изключително добра памет, не усещах неудобството от липсата ми на систематичен подход. Затова тази точка ми струва голямо количество болезнено съсредоточение, а грешките ми толкова ме дразнеха и напредвах толкова слабо, и толкова често се връщах към старите си привички, че бях почти готов да се откажа да опитвам и да се задоволя с несъвършен характер в това отношение, като човекът, който отишъл да си купи брадва от ковача – съседа ми – и поискал цялата да е лъсната като ръба. Ковчът се съгласил да я излъска до блясък при условие, че човекът се съгласи да върти колелото; той въртял, а ковачът притискал силно и тежко широката страна на брадвата към камъка, което направило въртенето много изморително. Човекът идвал отвреме навреме от колелото да погледне как върви работата и накрая поискал да си вземе брадвата както е, без повече лъскане. „Не“, казал ковачът, „върти, върти; след малко ще блести; сега е още на петна.“ „Може,“ казал човекът, „но мисля, че най ми харесва брадва с петна.“ Сега вярвам, че е възможно много хора, след като поради липса на някакъв подход като този, който аз използвах, са намерили за трудно придобиването на добри и оставянето на лоши навици свързани с други видове пороци и добродетели, да са преминали през подобно преживяване, да са се отказали от борбата и да са заключили, че „най им харесва брадва с петна“; защото нещо, което се преструваше, че е разумът, от време на време ми внушаваше, че такава крайна безупречност като тази, която изисквах от себе си, може да е един вид нравствена суета, за която ако се разчуе, ще стана за посмешище; че съвършеният характер върви с неудобството да бъдеш мразен и обект на хорската завист; и че добронамереният човек трябва да си остави някои недостатъци, за да не смущава приятелите си.

В действителност установих, че съм непоправим по отношение на Реда; и сега, когато съм стар и паметта ми е отслабнала, усещам много остро липсата му. Но, като цяло, въпреки че никога не постигнах съвършенството, което имах амбицията да добия, а останах доста далеч от него, все пак опитът ме направи по-добър и щастлив човек, отколкото бих бил, ако не бях се опитал; като тези, които се опитват да усъвършенстват почерка си като имитират гравюри, въпреки че не успяват да постигнат превъзходството на гравюрите, все пак почеркът им се подобрява от усилието и е приемлив и същеверемнно хубав и четлив.

Може би е добре потомството ми да научи за този малък трик, на който – с Божията благословия – техният праотец дължи постоянното щастие на живота си, чак до седемдесет и деветата му година, в която това е написано. Какви обрати предстоят в остатъка е в ръцете на Провидението; но ако такива дойдат, размислите за изпитаното минало щастие би трябвало да помогнат да бъдат те понесени с повече примирение. На Въздържанието той приписва дългогодишното си здраве и остатъците от добро телосложение; на Усърдието и Пестеливостта – ранните успехи и придобиването на богатството си, както и на всичкото това знание, което му позволи да бъде полезен гражданин, и му създаде някакво име сред учените; на Искреността и Справедливостта – доверието на страната си и почетните дела, които тя му повери; а на съчетаното влияние на всичките добродетели, дори във несъвършения вид, в който успя да ги придобие – всичкото това спокойствие на характера и всичката тази бодрост в разговор, поради които компаният му е търсена и досега и приятна дори на по-младите му познати. Затова се надявам, че някои от наследниците ми ще последват примера и пожънат ползите.

Въпреки че планът ми не беше изцяло лишен от религия, може да се отбележи, че не носи белезите на нито едно от основните вярвания на коя да е секта. Нарочно ги избегнах; понеже бях напълно убеден в полезността и превъзходството на метода ми, и че може да бъде полезен на хора от всички религи, и понеже възнамерявах в един или друго момент да го публикувам, не исках в него да има каквото и да е, което би отблъснало който и да е, от която и да е религия. Имах намерение да напиша кратък коментар към всяка добродетел, в който щях да покажа нейните предимства и недостатъците съпътстващи съответния порок; и щях да нарека книгата си „ИЗКУСТВОТО НА ДОБРОДЕТЕЛТА“\footnote{Нищо не е по-вероятно да направи човек богат от добродетелта. -- Бележка в полето.}, понеже щеше да показва средствата и начина за постигане на добродетел, което щеше да я разграничи от простото призоваване към добри дела, което не поучава и не посочва начина, и така прилича на вербалния благодетел от Апостола, който призовава голите и гладните да се нахранят и да се облекат, без да им покаже къде могат да намерят дрехи и храна – Яков 2:15,16.

Но така се случи, че намерението ми да напиша и публикувам този коментар не се осъществи. Наистина от време на време си записвах кратки бележки с настроения, размишения и т.н., които да използвам в него, някои от които все още имам на разположение; но нуждата да отдавам вниманието си на лични работи в по-ранната част от живота ти и на обществени задачи оттогава, ме накараха да го отложа; и понеже в съзнанието ми той е свързан с голям и обширен порект, който изисква цялото внимание на човек, за да бъде изпълнен, и на който поредица от непредвидени занимания ми попречи да се отдам, досега е останал незавършен.

Намерението ми беше в тази творба да обясня и подкрепя доктрината, че злите деяния не са вредни, защото са забранени, а че са забранени, защото са вредни, с оглед единствено на човешката природа; и че по тази причина е в интереса на всеки, който иска да бъде щастлив в този свят, да бъде добродетелен; и изхождайки от това обстоятелство (понеже винаги има по света голям брой богати търговци, благородници, държави и принцове, които имат нужда от честни служители за да управляват работите си, и понеже споменатите се срещат толкова рядко), щях да се опитам да убедя младите хора, че няма качества, които е по-вероятно да направят човек богат от честността и почтеността. 

Списъкът ми с добродетели първоначално съдържаше едва дванадесет; но след като един мой приятел квакер любезно ме информира, че като цяло ме смятат за горд; че гордостта ми често проличава в разговор; че не се задоволявах с това да съм прав в спора, а че натяквах и се държах доста нагло, в което ме убеди с няколко примера; реших, ако мога, да се излекувам от този пророк или глупост наред с другите, и прибавих Смирението към списъка си, като дадох на думата обширно значение. 

Не мога да се похваля с много успех в действителното придобиване на тази добродетел, но външен вид го докарах. Стана ми правило да избягвам всякакви директни противоречия с чуждите мненията и всички положителни изрази на моите собствени. Дори си забраних, съгласно правилата на нашето Хунто, да използвам в езика си каквито и да е думи или изрази, които придават твърдост на мнението, като със сигурност, без съмнение и т.н., и вместо тях използвах струва ми се, доколкото разбирам, струва ми се, че нещо е така и така; или така ми изглежда в момента. Когато някой направеше твърдение, което смятах за погрешно си отказвах удоволствието рязко да го конфронтирам и веднага да му покажа нещо абсурдно в твърдението му; и в отговора си започвах с налбюдението, че в някои случаи или обстоятелства мнението му би било правилно, но в настоящия изглежда или ми се струва, че има известна разлика, и т.н. Скоро установих предимството на тази промяна в подхода ми; разговорите, в които участвах преминаваха по-приятно. Скромният начин, по който изразявах мненията си предразполагаше към по-лесно възприемане и по-малко противоречия; когато се окажеше, че греша, грешките ми ми носеха по-малко унижение, и по-лесно убеждавах другите да се откажат от грешките си и да се присъединят към мен, когато се случеше да съм прав.

И в крайна сметка, този подход, който в началото си наложих с известно усилие против естествените ми наклонности, в крайна сметка ми стана толкова удобен и така привикнах към него, че през последните може би петдесет години никой не ме е чул да се произнасям догматично. И на този навик (след почтеността) приписвам това, че толкова рано придобих тежест сред съгражданите си при предлагането на нови институции или при промени в старите и толкова влияние в обществените съвети, когато станах член; защото слабо умеех да говоря, никога не бях красноречив, много се колебаех в избора на думи, едва успявах да говоря без грешки и въпреки това като цяло успявах да събера подкрепа за предложенията си.

В действителност едва ли има друга естествена склонност, която да е толкова трудна за превъзмогване като гордостта. Прикривай я, бори се с нея, тъпчи я, задушавай я, усмирявай я колкото искаш, все пак е жива, и отвреме навреме ще надникне и ще се покаже; може би често ще я виждаш в тази история; защото дори да можех да си допусна, че напълно съм я превъзмогнал, в такъв случай най-вероятно щях да се възгордея със смирението си.

\textit{\textbf{Дотук написано в Паси, 1784 г.}}

\chapter{Част трета}

"Сега започвам да пиша у дома, август, 1788 г., но не мога да разчитам на помощта на книжата си, тъй като голяма част от тях бяха загубени по време на войната. Въпреки това намерих следното."\footnote{Това е странична бележка. --Б.}

След като споменах един голям и обширен проект, който бях замислил, изглежда удачно тук да бъде казано нещо за проекта и неговата цел. Първите ми хрумвания по въпроса са поместени в следната кратка бележка, случайно запазена, а именно:

Наблюдения от четенията ми по история, в Библиотеката, 19 май 1731 г.

„Че големите събития в света, войните, революциите и т.н., се изпълняват и влияят от групи хора”

„Че гледната точка на тези групи хора се определя от настоящия им общ интерес или това, което те смятат за такъв.“

„Че различните гледни точки на различните групи хора причиняват цялото объркване.“

„Че докато дадена група изпълнява общия си план, всеки човек гледа собствения си интерес.“

“Че щом дадена група постигне общата си цел, всеки член се концентрира върху своя собствен интерес; който, пречейки на останалите, разделя групата на фракции и увеличава объркването.“

“Че малко от хората на обществени длъжности действат само с оглед на доброто на страната си, независимо какво твърдят; и макар и действията им да са принесяли истинско добро на страната им, все пак хората преидмно са смятали, че интересите на страната им и техните собствени сочат в една посока, а не са действали изцяло подтиквани от благодетелни принципи.“

„Че още по-малък е броят на тези общественици, които действат с оглед доброто на цялото човечество.“

“Струва ми се, че в момента има основателни причини за издигане на „Обединена партия за добродетелта“, която да обедини добродетелните и добри мъже от всички националности, и която да бъде ръководена според подобаващи добри и мъдри правила, към които добрите и мъдри мъже вероятно ще се придържат с по-голямо единодушие, отколкото обикновените хора към обикновените закони.“

“В момента мисля, че който подхване работата правилно и е достатъчно квалифициран, не може да не угоди Богу и да не успее. 
Б. Ф.“

Въртейки този проект в мислите си, и възнамерявайки да го подхвана по-късно, когато обстоятелствата ми предоставят нужното свободно време, от време на време си записвах мислите, които ми хрумваха във връзка с него. Повечето от бележките ми са загубени, но намерих една, която излгежда представлява чернова на същината на символ на вярата, който съдържа, според тогавашните ми разбирания, основите на всяка известна религия, и от който липсва всичко, което може да възмути поддръжниците на която и да е религия. Изразен е със следните думи, а именно:

„Че има един Бог, който е създал всичко.“

„Че той ръководи света чрез провидението си.“

„Че трябва да му бъде служено чрез обожание, молитва и отдаване на благодарност.“

„Но че службата най-угодна на Бог е правенето на добро на другите.“

“Че душата е безсмъртна.“

“И че Бог със сигурност ще възнагради добродетелта и накаже порока в този свят или в отвъдния.“\footnote{В Средновековието Франклин – ако такъв феномен като Франклин въобще би бил възможен в Средновековието – най-вероятно е щял да бъде основател на монашески орден. --Б.}

Идеите ми по това време бяха, че първоначално сектата трябва да се започне и разпространи само сред млади неженени мъже; че всеки, който желае да бъде приет, не само трябва да заяви съгласието си с такъв символ на вярата, но и да се упражни в тринадесетседмичното самоизпитване и приложение на добродетелите, като в по-горе описания модел; че съществуването на такова общество трябва да бъде държано в тайна, докато не стане значително, за да се избегнат настояванията за включване на неподходящи хора, но че всеки от членовете трябва да потърси сред познатите си чистосърдечни, благоразположени младежи, на които планът постепенно да бъде открит с благоразумна предпазливост; че членовете трябва да подпомагат взаимно интересите, бизнеса и напредъка си в живота, като поставят съветите, помощта и подкрепата си на взаимно разположение; че за да се отличаваме, ще се наречем Обществото на Свободните\footnote{Free and Easy}: свободни, понеже от навика да упражняваме добродетелите сме свободни от пороци; и особено чрез практикуването на усърдие и спестовничество сме свободни от дълг, който ограничава човек и го прави вид роб на кредиторите му. 

Толкова мога да си спомня сега за проекта, освен че го споделих отчасти с двама млади мъже, които го приеха с известен ентусиазъм; но притеснените ми обстоятелства по онова време и нуждата да не се отделям от работата си ме накараха да отложа изпълнението му; и разнообразните ми занимания – обществени и лични – ме накараха да продължа да отлагам, така че след толкова дълго отлагане вече нямам нито сила, нито деятелност достатъчни за такова предприятие; макар че все още си мисля, че планът беше осъществим и можеше да бъде много полезен чрез създаването на голям брой добри граждани; и не бях обезсърчен от привидният мащаб на начинанието, тъй като винаги съм смятал, че един човек с поносими способности може доведе до големи проблеми и постигне велики дела сред хората, ако първо си направи добър план и, след това, отказвайки се от всички забавления или други занимания, които биха го разсеяли, превърне изпълнението на този план в единствено свое занимание.

В 1732 г. за първи път публикувах Алманаха си, под псевдонима Ричард Сондърс; той продължи да бъде издаван от мен около двадесет и пет години и беше известен с името Алманаха на бедния Ричард. Опитвах се да го направя едновеременно забавен и полезен, и съответно стана толкова търсен, че пожънах значителна печалба от него, продавайки годишно близо десет хиляди. И като забелязах, че се чете почти навсякъде – почти нямаше махала в провинцията, в която да го няма – сметнах, че е подходящо средство за предаване на напътствия на обикновените хора, които почти не купуваха други книги; затова запълвах всички малки пространства между забележителните дни в календара с поговорки, предимно такива, които насърчават усърдието и спестовничеството като средства за придобиване на богатство и по този начин осигуряване на добродетел; понеже е по-трудно на човек в нужда винаги да действа честно, както – да използвам тук една от поговорките – на празния чувал му е трудно да стои прав.

Събрах тези поговорки, които съдържаха мъдростта на много епохи и народности, и, под формата на словоизлиянието на един мъдър стар човек към хора присъстващи на търг, ги превърнах в свързана проповед приложена като увод към алманаха от 1757 г. Събиране във фокус по този начин на тези пръснати съвети им позволи да направят по-голямо впечатление. Произвидението срещна всеобщо одобрение и беше копирано във всички вестници на Континента, препечатано във Великобритания като прокламация, която да се разлепя по къщите; бяха направени два френски превода и много бройки закупени от духовници и богаташи, за да ги разпространяват безплатно сред бедните членове енорията или наемателите си. Понеже насърчаваше въздържание от излишни разходи за ненужни вносни предмети, в Пенсилвания някои сметнаха, че е допринесло за създаването на това изобилие от пари, което можеше да се види в продължение на няколко години след публикацията му.

Гледах на вестника си като на още едно средство за разпространяване на поучения и по тази причина често препечатвах в него откъси от Спектейтър и други нравствени писатели; а понякога пулибкувах и кратки собствени произведения, които бяха първоначално съчинени за четене пред нашето Хунто. Сред тях има един Сократов диалог, който се стреми да докаже, че не е правилно злият човек да бъде наречен разумен, каквито и възможности и дарования да има; и една беседа върху себеотрицанието, която показва, че добродетелта не е в безопасност, докато нейното упражняване не се превърне в навик и не се освободи от съпротивата на противните наклонности. Тези могат да бъдат намерени в броевете от около началото на 1735 г.

В ръководството на вестника си внимателно избягвах всякакво клеветничество и лични нападки, които в последните години позорят страната ни. Винаги, когато идваха да ме молят да пусна нещо такова и авторите се позоваваха – както ставаше обикновено – на свободата на словото, и спореха, че вестниците са като дилижанс, в който всеки, който плати, има право да се вози, отговарях, че бих отпечатал произведението отделно, ако искат, и че авторът може да поръча колкото копия има желание да разпорстрани лично, но че не се наемам да разпространявам злословията му; и че след като бях сключил договор с абонатите си да им предоставям съдържание, което е или полезно, или забавно, не бих могъл да напълня вестникът им с лични преприни без да им причиня очевидна неправда. Сега, много от печатарите ни не изпитват скрупули и задоволяват човешката злоба като публикуват фалшиви обвинения насочени към най-почтените сред нас, разпалвайки омразата чак до дуели; и освен това са достатъчно недискретни, за да публикуват долни разсъждения за правителствата на съседните ни държави или дори за поведението на най-добрите ни национални съюзници, които биха могли да имат гибелни последствия. Тези неща споменавам като предупреждение за младите печатари, а и за да ги насърча да не замърсяват пресите си и да не позорят професията си с такива безславни практики, а да отказват твърдо, тъй като могат да се убедят от моя пример, че подобно поведение като цяло няма да накърни интересите им.

През 1733 г. изпратих един от работниците си в Чарлстън, Южна Каролина, където имаха нужда от печатар. Осигурих му преса и букви като част от споразумение за съдружие, според което щях да получавам една трета от печалбата от бизнеса и да поемам една трета от разходите. Той беше учен човек и честен, но неук по отношение на счетоводството; и макар че понякога ми пращаше пари, не можах да получа от него никакъв разчет, нито да разбера състоянието на съдръжничеството ни докато беше жив. Щом умря, бизнесът премина в ръцете на вдовицата му, която – родена и отгледана в Холандия, където съм чувал, че счетоводството е част от женското образование – не само ми изпрати възможно най-ясния разчет, който можа да намери за минали сделки, а продължи да ми се отчита най-редовно и точно всяко следващо тримесечие, и ръководеше бизнеса толкова добре, че не само успя да отгледа честно децата си, а при края на споразумението ни успя да купи от мен печатницата и да установи сина си в нея. 

Споменавам тази история преди всичко за да препоръчам този вид образование за нашите млади жени, като нещо, което вероятно ще е по-полезно за тях и за децата им от музика или танци, като ще ги запази от загуби причинени от измами на хитреци и като им позволи може би да продължат печеливш семейния бизнес с установени връзки, докато някои от синовете не порасне, за да се захване с него за трайна полза и обогатяване на семейството.

Окол 1734 година сред нас пристигна от Ирландия млад презвитериански проповедник на име Хемфил Hemphill, който проповядваше с добър глас прекрасни проповеди – на пръв поглед импровизирани – с които събираше значителен брой слушатели с различна вярвания, които заедно им се възхищаваха. Проповедите му ми харесваха, понеже не бяха от догматичния вид, а силно наблягаха на практикуването на добродетелност, или което на религиозен жаргон се нарича добри дела, и – като останалите – станах един от постоянните му слушатели. Тези обаче от нашето събрание, които се смятаха за ортодоксални презвитерианци не одобриха доктрината му и бяха подкрепени от повечет от старите духувници, които го обвиниха в иноверство пред синода, за да му запушат устата. Станах ревностен негов поддръжник и допринесох с всичко, с което можах, за да събера подкрепа за него и известно време водихме битки за него с известни надежди за успех. По случая имаше доста драскане за и против; и след като установих, че макар да беше елегантен проповедник, е слаб писател, поставих перото си на неговите услуги и написах за него два или три памфлета и една дописка в Газетата\footnote{собствения му вестник, the Pennsylvania Gazette, най-вероятно} от април 1735. Въпреки че памфлетите бяха прочетени с голямо желание, както обикновено става с противоречивите писания, скоро загубиха популярност и се питам дали и едно копие от тях се е запазило до днес.

По време на споровете една нещастна случка навреди много на каузата ни. Един от противниците ни, като го чул да проповядва проповед, която много се харесала, помислил, че я е чел някъде по-рано, или поне част от нея. След като потърсил, накрая намерил въпросната част в един от Британските прегледи, в проповед на Д-р Фостър. Тази засечка отврати много от нашите и те съответно изоставиха каузата му, което причини по-бързия ни разгром в синода. Аз обаче му останах верен, понеже по-скоро одобрявах това, че ни предаваше добри проповеди съчинени от други, отколкото лоши собствено производство, макар че вторият подход беше този на повечето от нашите учители. По-късно ми призна, че никоя от проповедите му не била негова; като добави, че паметта му била такава, че му позволявала да запомни и повтори всяка проповед след само един прочит. След като загубихме, той ни остави и отиде да си опита късмета другаде, а аз напуснах събранието (т.е църквата) и никога повече не се върнах, макар че в продължение на много години плащах вноската си за издръжката на проповедниците.

През 1733 бях започнал да уча езици; скоро се научих на толкова френски, че да мога с лекота да чета книгите. След това се захванах с италианския. Един познат, който също го учеше, често ме изкушаваше да играя шах с него. Понеже реших, че това ми отнема твърде много от времето за учене, накрая отказах да играя повече, освен при това условие победителят във всяка игра да получи правото да поставя задача – или част от граматиката да бъде научена наизуст, или превод и т.н. – която победеният да трябва да изпълни преди следващата среща на честна дума. Понеже бяхме горе-долу на едно ниво скоро се победихме в езика. По-късно, след малък напън, придобих достатъчно испански, за да чета и техните книги.

Вече спомеах, че бях прекарал само една година в латинско училище, при това много млад, след което напълно занемарих езика. Но след като се бях запознал с френския, италианския и испанския, когато погледнах веднъж един латински Нов Завет, се изнанадах, че разбирам много повече от езика, отколкото си бях представял, което ме насърчи отново да се отдам на изучаването му, където също пожънах успех, тъй като предишните езици много бяха улеснили напредъка ми. 

Тези обстоятелства ме карат да мисля, че има някакво противоречие в обичайния подход към изучаването на езици. Казват ни, че е правилно да се започне с латинския и – след като сме го овладели – ще бъде по-лесно да научим тези модерни езици, които произлизат от него; и все пак не започваме с гръцкия, за да ни е по-лесно да научим латинския. Вярно е, че ако намериш начин да издрапаш и да се добереш до върха на едно стълбище без да ползваш стъпалата, ще ти е по-лесно да стигнеш до тях слизайки; но със сигурност, ако започнеш с най-долните, ще ти е по-лесно да стигнеш до върха; и затова бих предложил на вниманието на тези, които съблюдават образованието на младежта ни, дали – понеже много от тези, които захващат латинския се отказват след няколко години без да са го овладели отлично, и това, което са научили остава почти безполезно, така че времето им е загубено – не би било по-добре да се започне с френски, след това да се продължи с италиански и т.н.; защото дори след известно време да се откажат от ученето на езици и никога не стигнат до латинския, все пак ще са придобили един или два езика, които – понеже са съвременни/живи – може да са им полезни в ежедневието.

След десетгодишно отсъствие от Бостън и след като бях постигнал охолство отидох до там, за да посетя роднините си, което по-рано не можех да си позволя без притеснение. На връщане минах през Нюпорт, за да видя брат си, по това време установен там с печатницата си. Поранешните ни дрязги бяха забравени и срещата ни беше много сърдечна и нежна. Здравето му бързо се влошаваше и ме помоли в случай, че умре, което той смяташе ще се случи не след дълго, да прибера сина му, който тогава беше на десет години,  у дома си и да го обуча в печатарския занаят. Така и направих, като го пратих на училище за няколко години преди да го взема в работата. Майка му продължи с бизнеса докато той не порасна, когато му помогнах с комплект нови букви types, понеже бащините му вече се бяха поизносили. Така се отплатих на брат си предостатъчно за услугите си, от които го лиших като го напуснах така рано.

През 1736 г. загубих един от синовете си, момче на четири години, от едра шарка прихваната по обичайния начин. Дълго и горчиво съжалявах и все още съжалявам, че не го бях ваксинирал. Това казвам за родителите, които избягват тази процедура, защото предполагат, че никога няма да си простят, ако детето умре от нея; моят пример показва, че скръбта може да е същата и в двата случая, и че по тази причина по-безопасният трябва да бъде избран.

Нашият клуб, Хунтото, се оказа толкова полезен и даваше на членовете такова удоволствие, че някои искаха да представят и приятелите си, което нямаше как да стане без да надхвърлим бройката, която бяхме определили за приемлива, а именно, дванадесет. От началото бяхме приели за правило да държим организацията в тайна, към което се придържахме доста стриктно; целта беше да се избегне кандидатирането на неподходящи личности, на някои от които би ни било трудно да откажем. Аз бях един от противниците на каквото и да е увеличаване на бройката, като вместо това направих предложение в писмен вид всеки член да се опита самостоятелно да организира подчинен клуб със същите правила относно темите и т.н., и без да ги информира за връзката с Хунтото. Предложените преимущества бяха, че голям брой други млади хора ще се подобрят благодарение на нашите организации; възможността по-добре да следим общите настроения сред населението по всякакви случаи, тъй като членът на Хунтото би могъл да предлага теми, които на нас ни се харесват и да докладва в Хунтото какво се е случило в другия клуб; подкрепата за частните ни бизнес интереси чрез по-широка възможност за реклама и увеличаването на влиянието ни върху обществените дела, както и способността ни да правим добро, чрез разпространяването през няколко клуба мненията на Хунтото.

Проектът беше одобрен и всеки член се захвана със създаването на собствен клуб, но не всички успяха. Пет или шест бяха завършени и бяха наречени с различни имена като Лозата, Съюзът, Групата и т.н.\footnote{the Vine, the Union, the Band} Бяха полезни за членовете си, а на нас доставиха доста забавление, информация и поучение, освен че до голяма степен отговориха на очакванията ни за въздействие върху общественото мнение при определени случаи, за което ще дам някои примери, когато разказът ми стигне до тях.  

Първото ми повишение беше избирането ми за писар на Общото събрание през 1736 г. Тази година никой не се противи на избирането ми; но следващата година, когато пак ме предложиха (изборът, като този за членове на събранието, се провеждаше ежегодно), един нов член държа дълга реч срещу мен, за да подкрепи някакъв друг кандидат. Въпреки това бях избран, което ми беше още по приятно, понеже освен заплатата за писарската служба, мястото ми даваше по-добра възможност да поддържам интереса си сред членовете, което ми осигури работа по печатането на бюлетини, закони, хартиени пари и други случайни обществени задачи, които като цяло бяха доста доходоносни.

Затова не ми се хареса съпротивата на този нов член, който беше богат и образован джентълмен с таланти, които изглежда след време щяха да го направят много влиятелен в Камарата, което се и случи по-късно. Въпреки това не се опитах да спечеля благоволението му чрез подмазване, а след известно време, приложих следния различен подход. След като чух, че имал в библиотеката си определена много рядка и интересна книга, му изпратих бележка, в която изразявах желанието си да видя книгата и го молех да ми направи услугата да ми я даде на заем за няколко дни. Той ми я изпрати веднага, а аз я върнах около седмица по-късно с друга бележка, в която изразявах голямата си благодарност за услугата. Следващият път, когато се видяхме в събранието, той ме заговори (което преди това никога не беше правил) и при това много учтиво; и винаги след това изразяваше готовност да ми услужи при всякакви случаи, така че станахме големи приятели и приятелството ни продължи до смъртта му. Това е още един пример за истината, която се съдържа в една стара максима, която бях научил, която гласи: „Който веднъж ти е направил добро, ще е по-склонен да повтори, от този, когото ти си задължил с благодеянието.“ И показва колко по-полезно и разумно е да се прекратяват неприятелските действия, вместо да се негодува, отвръща и продължава с тях.

В 1737 г. полковник Спотсуд, покойният губернатор на Вирджиния, а по това време и ръководител на пощите postmaster-general, недоволен от наместника си във Филаделфия, поради някаква немарливост и неточности в счетоводството му, го уволни и предложи работата на мен. Аз с готовност я приех и намерих, че ми дава големи предимства; защото – въпреки че заплатата беше малка – улесняваше съобщенията, които правеха вестника ми по-добър, увеличи тиража му, както и броя на рекламите, така че в накрая ми докара значителен доход. Вестникът на стария ми конкурент съответно западна, а аз бях много доволен без да му върна за неговия октаз, докато беше ръководител на пощата, да позволи ездачите да носят вестника ми. Така той много пострада от немарливостта си в счетоводството; и го споменавам като урок за тези млади мъже, които може би ще бъдат заети с ръководството на чужди работи, че винаги ще трябва да представят сметка и да изплащат дължимите суми с голяма яснота и точност. Придържането към тези принципи е най-силната от всички препоръки при търсенето на нова работа и за разрастване на бизнеса.

Сега започнах полека да насочвам мислите си към обществени дела, започвайки обаче с малки неща. Градската стража беше едно от първите неща, които забелязах, че се нуждаят от регулация. Полицаите на съответните квартали се редуваха в ръководството; полицаят събираше определен брой стопани да му помагат всяка нощ. Тези, които избираха да не участват му плащаха по шест шилинга на година в замяна, което трябваше да е за наемане на заместници, но в действителност беше много повече от достатъчно за тази цел, което превръщаше полицейската служба в доходоносно предприятие; а полицаят, за едно малко for a little drink, често събираше такива негодяи за стражи, с каквито честните стопани предпочитаха да не се събират. Често обиколките бяха занемарявани, а по-голямата част от нощта прекарвана в пиене. По тази причина написах едно есе за пред Хунтото, което представяше тези нередности, но поставяше ударението най-вече върху несправедливостта на полицейския данък от шест шилинга по отношение състоянието на данъкоплатците, понеже бедна вдовица чието цяло имущество вероятно възлизаше на по-малко от петдесет паунда плащаше толкова, колкото най-богатият търговец, който имаше стока за хиляди паундове в складовете си.

Най-общо, предложих като по-ефективна стража да бъдат наемани подходящи мъже, които да са постоянно заети с тази работа; а като по-справедлив начин за издръжка на разходите събирането на данък пропорционално на имуществото. След като беше одобрена от Хунтото, идеята беше предадена на другите клубове, но така, сякаш сама се е зародила във всеки от тях; и въпреки че планът не беше изпълнен веднага, все пак, като подготви умовете на хората за промяната, разчисти пътя за закона, който беше приет няколко години по-късно, когато членовете на клубовете ни бяха придобили повече влияние.

По горе-долу това време написах една статия (първо за пред Хунтото, но после беше публикувана) относно различните произшествия и недоглеждания, които причиняват пожари, която предупреждаваше за опасностите и предлагаше средства за тяхното отстраняване. Беше намерена за много полезна и даде начало на един проект, който беше осъществен малко след написването й, относно създаването на компания за по-бързото потушаване на пожари, и за взаимопомощ при преместването и запазването на имущество в случай на опасност. Намериха се тридесет сътрудници за осъществяване на плана. Точките на споразумението ни задължаваха всеки член да държи на разположение и готови за употреба определен брой кожени кофи, както и здрави чанти и кошници (за опаковане и пренос на имущество), който да бъдат занасяни при всеки пожар; и се разбрахме веднъж месечно да се събираме и да прекарваме една вечер заедно, за да обсъждаме и споделяме каквито идеи ни хрумнат свързани с пожарите, и които могат да подобрят подхода ни към такива случаи.

Ползата от организацията скоро пролича, и понеже много повече хора пожелаха да се присъединят, отколкото беше удобно за една команда, бяха посъветвани да сформират друга, което съответно те направиха; и нещата продължиха по този начин, образуваше се команда след команда, докато накрая не станаха толкова многобройни, че включваха повечето от мъжете с имот; и в момента, в койт пиша това, макар че са минали повече от петдесет години от основаването на първата команда, която създадох, наречена Съюзна пожарна команда, тя все още процъфтява, въпреки че първоначалните членове всичките са умрели с изключение на мен и още еди, който е по-стар от мен с една година. Малките глоби събирани от членовете при отсъствие от месечните срещи се използват за покупката на пожарни коли, стълби, куки fire-hooks и други полезни приложения implements за всяка команда, дотолкова, че се питам дали има друг град по света, който да е по-добре снабден за гасене на начеващи пожари; и всъщност благодарение на тези организации градът не е губил повече от една или две къщи наведнъж при пожар, а пламъците често са били угасяни преди къщата, в която са започнали да е изгоряла на половина.

През 1739 г. от Ирландия дойде при нас преподобния г-н Уаитфийлд, който беше придобил име там като пътуващ проповедник. Първоначално му беше позволено да проповядва в някои от нашите църкви; но духовниците не го харесаха и скоро му отказаха достъп до амвона, и той се видя принуден да проповядва в полето. Множеството от хора от всички секти и деноминации, които идваха на проповедите му, и от което и аз бях част, беше огромно, и изключителното влияние на ораторството му въху слушателите му, и уважението и възхищението, които произвеждаше в тях въпреки, че направо ги обиждаше като ги уверяваше, че са по природа наполовина зверове, наполовина дяволи, беше за мен тема за спекулации. От хора, които не мислят за религията или са безразлични към нея, изведнъж изглежда всички започнаха да стават религиозни, така че човек не можеше да мине през града вечер без да чуе да се пеят псалми сред различните семейства на всяка улица. 

Понеже събиранията на открито се оказаха неудобни поради това, че са изложени на несгодите на времето, едва беше предложено да се построи дом за срещи и хора определени за събиране на средства, и вече бяха събрани средства за закупуване земя за издигане на сградата, която беше дълга сто стъпки и широка седемдесет, горе-долу като Уестминстър хол по размер; и работата беше захваната с такъв дух, че беше завършена много по-рано отколкото можеше да се очаква. Грижата за домът и земята беше възложена на попечители с изричната цел да бъдат на разположение на всеки проповедник от всяко религиозно учение, който би искал да каже нещо на хората във Филаделфия; целта на строежа не беше да приюти която и да е секта, а жителите като цяло; така че дори ако цариградския мюфтия беше изпратил мисионер да ни проповядва мюсюлманство, той щеше да намери амвон, от който да говори.

След като ни напусна, г-н Уайтфийлд продължи с проповедите си сред колониите чак до Джорджия. Заселването на тази провинция скоро беше започнало, но вместо да се извършва от здрави, усърдни земеделци навикнали на труд – единствените годни за тази работа – то се извършваше от семействата на разорени бакали и други несъстоятелни длъжници, много от които мързеливи и навикнали на бездействие, извадени от затворите, и голям брой от които, щом попаднеха в гората, понеже не знаеха как да разчистят земята и не можеха да понесат несгодите съпътстващи живота в ново селище, загиваха оставяйки много безпомощни деца без грижа. Видът на тяхното злощастно положение вдъхновило великодушното сърце на г-н Уатфийлд с идеята да построи там Дом за сираци, в който те биха могли да бъдат подслонени и образовани. Щом се върна на север, той проповядва на тази тема и събра много пари, тъй като красноречието му имаше чудно въздействие върху сърцата и кесиите на слушателите му, за което сам съм пример. 

Не че не одобрявах плана, но понеже в Джорджия нямаше материали и работници по онова време, а предложението беше да се изпратят от Филаделфия, което щеше да е много скъпо, смятах че щеше да е по-добре да се построи дом тук, а децата да бъдат доведени в него. Такъв беше моят съвет; но той твърдо държеше за първоначалния си план, отхвърли препоръката ми, а аз затова отказах да допринеса. Така се случи, че скоро след това присъствах на една от проповедите му, по време на която усетих, че има намерение да приключи с призив за дарения, и мълчаливо реших, че от мен няма да получи нищо; в джоба си имах една шепа медни пари, три или четири сребърни долара, и пет златни пистоли. Докато продължаваше с проповедта започнах да умеквам и заключих, че ще дам медните пари. Още един щрих от риториката му ме накара да се засрамя и реших да дам и среброто; и приключи толкова възхитително, че си изпразних джобовете в дискуса, злато и всичко останало. На тази проповед присъства и един от нашия клуб, който – споделяйки мнението ми относно строежа в Джорджия – като предпазна мярка си беше изпразнил джобовете у дома преди да дойде. Към края на проповедта обаче почувствал силно желани да даде, и помолил един свой съсед, който стоял наблизо, да му даде нещо назаем за тази цел. За нещастие молбата била насочена към може би единствения човек в групата, който бил достатъчно твърд, за да не се поддаде на влиянието на проповедника. Отговорът му бил: „По всяко друго време, приятелю Хопкинсън, бих ти дал назаем; но не и сега, понеже ми се струва, че не си на себе си.“

Някои от враговете на г-н Уайтфийлд предполагаха, че ще използва тези дарения за собствена облага; но аз, който го познавах отблизо (понеже бях зает с отпечатването на неговите Проповеди и Дневници и т.н.), никога не се съмнявах ни най-малко в почтеността му, и до днес съм определено на мнение, че във всичките си дейности той беше съвършено честен човек, и ми се струва, че свидетелството ми в негова полза би трябвало да има по-голяма тежест, като се има предвид, че нямахме религиозна връзка. Наистина той понякога се молеше за моето обръщане, но никога не изпита удоволствието от това се убеди, че молитвите му са били чути. Нашето приятелство беше просто гражданско, взаимно искрено, и продължи до смъртта му.

Следният пример ще открие нещо за отношенията ни. При едно от идванията му в Бостън от Англия ми писа, че скоро ще идва към Филаделфия, но не знае къде може да се подслони, тъй като старият му приятел и домакин, г-н Бензет Benezet, се беше преместил в Джърмантаун. Отговорът ми беше, „Знаеш къде е къщата ми; ако можеш да се справиш със скромните условия в нея, ще си най-сърдечно добре дошъл.“ Той ми отговори, че ако съм направил предложението си заради Христа, със сигурност ще бъда възнаграден. А аз отвърнах, „Не ме разбирай погрешно; не беше заради Христа, а заради теб.“ Един от общите ни познати се пошегува, че знаейки, че когато се ползват от услуга светците обкновено прехвърлят отговорността за разплатата от собствените си плещи към рая, съм се опитал да я закотвя на земята.

Последно видях г-н Уайтфийлд в Лондон, където посъветва с мен относно дома за сираци и намерението му да го даде за основаването на колеж. 

Имаше силен и ясен глас и произнасяше думите и изреченията си така съвършено, че беше възможно да бъде чут и разбран на голямо разстояние, особено при положение че слушателите му съблюдаваха пълна тишина, колкото и многобройни да бяха. Една вечер проповядваше от върха на стълбището пред Съда, който е в средата на Маркет стрийт и от западната страна на Секънд стрийт, която я пресича под прав ъгъл. Слушателите му изпълваха и двете улици на значително разстояние. Понеже бях един от по-задните на Маркет стрийт ми стана любопитно да науча на какво разстояние може да бъдет чут като се отдалеча заднишком надолу по улицата към реката; и установих, че гласът му се чуваше отчетливо докато стигнах до Фронт стрийт, когато някакъв шум от улицата го заглуши. Тогава си представих полукръг с радиус равен на разстоянието ми от него изпълнен с хора, и като оставих по два квадратни фута на човек, изчислих, че може спокойно да бъде слушан от повече от тридесет хиляди. Така се примирих с разказите от вестниците, които говореха, че проповядвал на двадесет и пет хиляди души в полето, както и с древните истории за генерали, които са надъхвали цели армии, в които понякога се бях съмнявал.

Понеже често му бях слушател започнах лесно да различавам новите му проповеди от тези, които често беше проповядвал по време на пътуванията си. Последните бяха толкова изпипани от многото повторения, че всеки акцент, всяко ударение, всяка извивка на гласа се извършваше така съвършено и беше така добре поставена, че дори да не се интересуваше от темата, човек не можеше освен да се наслади на проповедта; удоволствиете беше много сходно с това, което дава отлична музикална композиция. Това е едно предимство, което пътуващите проповедници имат пред тези, които не пътуват, понеже последните не могат чрез толкова много повторения да подобрят начина, по който проповядват дадена проповед.

Това, че отвреме на време пишеше и отпечатваше по нещо, даваше голямо предимство на враговете му; непредпазливи изрази и дори грешни мнения изказани в проповед биха могли да бъдат обяснени по-късно, или интерпретирани по друг начин в светлината на други, които може би са ги съпътствали, или биха могли да бъдат отречени; но написаното остава. Критицете бясно нападаха писанията му и то привидно с такива разумни аргументи, че успяваха да намалят броя на поддръжниците му или да предотвратят увеличаването им; поради което съм на мнение, че ако никога не беше написал нищо, щеше да остави след себе си много по-многобройна и значителна секта, и репутацията му в този случай би могла все още да расте, дори след смъртта му, понеже ако нямаше нещо написано, на което да се основе критика, последователите му щяха да могат да му припишат всички видове превъзходство, които ентусиазираното им възхищение би ги накарало да искат да е притежавал.

Бизнесът ми сега постоянно се разрастваше а състоянието ми всеки ден ставаше по-леко, понеже вестникът ми беше станал доста печеливш, като за известно време почти единствен в тази и околните провинции. Изпитах и истината на наблюдението, „че след като спечелиш първите сто паунда, е по-лесно да спечелиш вторите,“ тъй като парите имат склонност да се размножават.

Успехът на съдружието в Каролина ме насърчи да участвам в други и да помогна на няколко от работниците си, които се бяха отличили с добро поведение, като им позволя да отворят печатници в различни колонии при същите условия като при тази в Каролина. Повечето се справиха добре, като накрая на срока на съдружието ни успяха да купят буквите от мен и да продължат да работят самостоятелно, благодарение на което бяха отгледани няколко семейства. Съдружията често приключват със свади; но в това отношение бях щастлив, че при мен всичките протекоха и приключиха приятелски, струва ми се в голяма степен благодарение на предпазливо подготвените договорни условия, в които много изрично се бяхме споразумяли относно всичко, което се очаква да свърши всеки от съдружниците, така че нямаше за какво се караме, поради което бих препоръчал тази предпазна мярка на всички, които започват съдружие; защото колкото и голямо уважение да има между двама съдружници, и каквото и доверирие да си имат в момента, в който сключват договора, малки ревности и търкания jealousies and disgusts могат да се появят, наред с усещане за ideas of несправедливо разпределение на тежестта и отговорностите в работата и т.н., които често приключват с разваляне на приятелството и прекъсване на отношенията, може би със съдебни дела и други неприятни последствия.

Като цяло имах много причини да съм доволен от това, че се установих/успях в Пенсилвания. Имаше, обаче, две неща, за които съжалявах, а именно, липсата на мерки за отбрана и на възможности за цялостно образование за младите; нямаше нито войска, нито колеж. По тази причина през 1743 г. съставих едно предложение за основаване на академия; и смятайки преподобния г-н Питърс, който беше без работа, за подходящ ръководител на подобна институция, споделих тогава проекта си с него; но той отказа, тъй като имаше наум по-доходоносни изгледи за служба при собствениците (на колониите, б.пр.), които се осъществиха; и понеже по това време не познавах друг подходящ за такава отговорност, оставих планът настрана за известно време. На следващата година, 1744, имах повече успех с предлагането и основаването на Философско дружество Philosophical Society. Статията, която написах за тази цел ще може да бъде намерена сред съчиненията ми, когато бъдат събрани. 

По отношение на отбраната, Испания от няколко години беше във война с Великобритания, а Франция в крайна сметка се включи на нейна страна, което ни постави в голяма опасност; и понеже продължителните и измъчени опити на губернатора ни, Томас, да убеди квакерското ни Събрание да приеме закон за войската и да вземе други мерки за сигурността на провинцията се оказаха безплодни, реших да пробвам какво може да се направи с доброволна организация. За да популяризирам идеята първо написах и обнародвах памфлет наречен ПРОСТА ИСТИНА, в който осветих беззащитното ни положение, както и нуждата от задружие и дисциплина за отбраната ни, и обещах до няколко дни да предложа организация за тази цел, към която всеки може да се присъедини. Памфлетът имаше внезапно и изненадващо въздействие. Поискаха от мен да изготвя усатав на организацията, и след като приготвих една негова чернова с няколко приятеля, определих среща на гражданите в голямата сграда, която вече споменах. Залата беше доста пълна; бях приготвил няколко отпечатани копия и осигурих писалки и мастило пръснати из цялата стая. Говорих им малко по темата, прочетох устава и го обясних, а след това раздадох копията, които бяха подписани с голямо желание без каквито и да е възражения.

След като множеството се разотиде и бяха събрани копията, намерихме около хиляда и двеста подписа; и понеже в провинцията бяха разпространени още копия, накрая броя на присъединилите се надхвърли десет хиляди. Всички те скоро се снабдиха както можаха с оръжия, създадоха роти и полкове, избраха си офицери и се събираха всяка седмица, за да бъдат обучавани как да боравят с оръжието и в други дялове на военната  дисциплина. Жените със собствена организация осигуриха копринени знамена, които подариха на ротите, а преди това изрисуваха с различни гербове и девизи mottos, които аз предоставих.

При срещата си офицерите от филаделфийския полк ме избраха за свои командир; но понеже сметнах, че съм неподходящ, отказах и препоръчах г-н Лорънс, добър човек и мъж с влияние, който беше назначен. След това предложих да организираме лотария, с която да покрием разходите по изграждане на батарея (battery, укрепление с оръдия) под града и закупуването на оръдие. Средствата бяха набрани скорострелно и укреплението скоро беше издигнато. Купихме няколко стари топа от Бостън, но понеже бяха недостатъчни, писахме до Англия за още, като по същото време помолихме и собствениците си, макар и без големи очаквания за успех.

Междувременно полковия ръководител Лорънс, Уилям Алън, г-н Ейбръм Тейлър и моя милост бяхме изпратени от членовете на организацията в Ню Йорк с поръката да вземем няколко оръдия назаем от губернатора Клинтън. Той първоначално ни представи своя безпрекословен отказ; но по време на вечеря със съветниците му, където обичаят беше да се пие много мадейра, постепенно започна да омеква и каза, че ще ни даде назаем шест. След още няколко тоста качи на десет; и в крайна сметка много добродушно отстъпи осемнадесет. Бяха хубави оръдия за снаряди от осемнадесет фунта със лафети, които скоро след това транспортирахме и поставихме на нашето укрепление, където членовете всяка нощ стояха на стража докато войната продължаваше, и където и аз наред с останалите изпълних дълга си като обикновен войник.  

Действията ми по този въпрос се харесаха на губернатора и на съвета; направиха ме свои довереник, и ме съветваха по всяка мярка, за която тяхното съдействие изглежда можеше да бъде полезно на организацията. Извиквайки на помощ религията им предложих да обявят пост с цел да бъде насърчена реформацията и да се измоли Божията благословия върху начинанието.  Те прегърнаха идеята; но тъй като беше първият пост организиран някога в провинцията, секретарят не намери прецедент, по който да изработи призива. Тук ми дойде на помощ образованието ми от Нова Англия, където се обявява пост всяка година: написах го в обичайния стил, беше преведен на немски, отпечата и на двата езика и разпространен из провинцията. Това даде на духовниците от различните секти възможност да окажат въздействие върху паствата си да се включат в организацията, и най-вероятно скоро всички освен квакерите щяха да са членове, ако мирът не беше настъпил малко след това.

Някои от приятелите ми смятаха, че чрез действията си в тази насока ще обидя въпросната секта и по този начин ще загубя положението си my interest в Събранието и провинцията, където те имаха мнозинство. Един млад джентълмен, който също имаше приятели в Камарата, и искаше да ме наследи в ролята ми на техен писар, ми откри, че било решено да ме сменят при следващия избор; и че по тази причина добронамерено ме съветвал да си подам оставката, тъй като това щяло да отговаря по-добре на честта ми от уволнение. Отговорих му, че съм чел или чувал някъде за един общественик, който имал за правило никога да не иска служба и никога да не отказва такава, ако му бъде предложена. „Одобрявам“, казах аз, „това правило и ще го следвам с едно малко допълнение; никога няма да искам, никога няма да отказвам и никога няма да отхвърлям служба. Ако искат да дадат писарската ми служба на друг, ще ми я отнемат. Не искам с отказа си от нея да губя правото си отвреме на време да поразявам враговете си.“ Въпреки това не чух нищо повече по този въпрос; отново бях избран единодушно при следващия избор. Може би щяха да са доволни, ако ги бях напуснал доброволно, тъй като не харесваха близостта ми с членовеете на съвета, които се бяха присъединили към губернаторите във всички спорове относно военната подготовка, с които Камарата дълго беше тормозена; но ревността ми за организацията не беше достатъчна, за да ги накара да ме отстранят, а друга причина нямаха.

Действително имах причини да смятам, че защитата на страната не беше неприятна на никои от тях, при положение, че не се изискваше от тях да участват в нея. И открих, че много по-голяма част от тях, отколкото си бях представял, очевидно подкрепят отбранителната война, макар и да са против нападателната. Много памфлети за и против се публикувах по темата, някои от които от добри квакери, в подкрепа на защитата, които вярвам убедиха повечето от младежите им.

Една случка с пожарната ни команда ми позволи да добия представа за преобладаващите настроения сред тях. Беше предложено да подкрепим плана за издигане на укрепление като с наличните средства – по това време около шестдесет паунда – купим билети от лотарията. Според правилата ни с парите не беше възможно да се разполагаме с парите преди срещата след предложението. Командата се състоеше от тридесет члена, от които двадесет и двама бяха квакери, а само осем с други убеждения. Ние осемте се появихме на срещата на време; но въпреки че мислехме, че някои от квакерите ще се присъединят към нас, в никакъв случай не бяхме сигурни в мнозинството си. Изглежда само един квакер, г-н Джеймс Морис, се противеше на мярката. Той изрази голямото си съжаление, че въобще е била предложена, тъй като всички Приятели (квакерите, бел. пр.) били против нея и щяла да произведе такова несъгласие, каквото би могло да разтури командата. Казахме му, че не виждаме причина за това; че сме малцинство, и ако Приятелите са срещу мярката и съберат повече гласове, ние трябва и сме задължени според обичая на всички дружества, да се подчиним. Когато стана време за работата, беше предложено да проведем гласуването; той се съгласи, че можем да го направим по правилата, но понеже можел да ни увери, че няколко от членовете възнамерявали да присъстват, за да се противопоставят, би било елементарна проява на добра воля да изчакаме малко да се появят. 

Докато обсъждахме този въпрос дойде един сервитьор да ми каже, че два джентълмена долу искат да говорят с мен. Слязох и намерих двама квакери от членовете ни. Казаха ми, че осем човека са се събрали в една механа tavern наблизо; че са решени да дойдат и да гласуват с нас при нужда, но се надяваха да не се наложи, и поискаха да не ги викаме на помощ, ако можем да се справим без тях, тъй като гласът им за тази мярка би ги въвлякъл в противоречия с техните старейшини и приятели. Уверен по този начин в мнозинството се качих горе и след известно привидно колебание се съгласих да изчакаме още час. Г-н Морис призна, че това е изключително любезно fair. Нито един от приятелите му, които бяха срещу мярката, не се появи, което много го изненада; и след като часът изтече гласувахме решението с осем срещу един; и тъй като от двадесет и двамата квакери осем бяха готови да гласуват с нас, а тринадесет показаха с отсъствието си, че не са склонни да се противопоставят на мярката, след това прецених, че сред квакерите пропорцията на тези, които са искрено против мярката, е само един към двадесет и един; защото всички тези бяха редовни членове на тази общност и с добро име сред тях, а и всички надлежно бяха уведомени какво ще се гласува на събранието. 

Почитаемият и учен г-н Логан, който през целия си живот беше принадлежал на тази секта, беше един от хората, които написаха обръщение към тях, в което заявяваше подкрепата си за отбранителната война, като подкрепяше мнението си с многобройни силни аргументи. Даде ми шестдесет паунда, с които да купя билети от лотарията за укреплението, и ми заръча да оползотворя наградите, които евентуално щяха да бъдат изтеглени, за същата цел. Разказа ми следната история за своя поранешен господар – Уилям Пен – относно отбраната. Беше дошъл от Англия на младини с този собсвтвеник (Пен, б. пр. ), като негов секретар. Било военно време и корабът им бил преследван от въоръжен съд, предполагаем враг. Капитанът подготвил защитата; но казал на Уилям Пен и другарите му квакери, че не очаква помощ от тях, и че могат да се оттеглят в каютата, което всички те направили, с изключение на Джеймс Логан, който избрал да остане на паулбата и бил разпределен при едно оръдие. Предполагаемият враг се оказал приятел, така че нямало битка; но когато секретарят слязъл да предаде новината, Уилям Пен остро го смъмрил задето останал на паулбата и се захванал със защитата на кораба, противно на принципите на Приятелите, особено при положение, че капитанът не го изискал. Мъмренето било пред цялата компания и затова доста жегнало секретаря, който отвърнал „Ти защо не ми нареди да сляза долу, като съм ти слуга? Но докато мислеше, че има опасност, нямаше нищо против да остана и да помогна в битката.“

Многото ми години в Събранието, където квакерите винаги са били мнозинство, ми дадоха много възможности да наблюдавам трудностите, които им създаваше техният принцип срещу войната винаги, когато ги помолеха от името на короната да предоставят средства за военни цели. От една страна не искаха да обидят правителството с директен отказ; а от друга, чрез изпълнение на молбата в противоречие на принципите си – приятелите си, квакерската общност; и от това следваха разнообразни опити да се избегне удовлетворяването на молбата и начини да се прикрие това удовлетворяване, когато то стане неизбежно. Накрая обичайният подход беше да дадат парите с думите, че са за „употреба на краля“ и никога да не питат как са били използвани.

Но ако искането не идваше направо от короната, изразът не беше толкова удачен и трябваше да бъде намерен друг. Така веднъж, когато имаше нужда от барут (мисля за гарнизона в Луисбърг Louisburg), и губернаторът на Нова Англия помоли Пенсилвания за някакво дарение, за което губернатор Томас много притисна Камарата, те не можеха да отпуснат пари за закупуване на барут, понеже той е средство за война; но гласуваха помощ за Нова Англия в размер на три хиляди паунда, които да бъдат предадени на разположение на губернатора, и ги определелиха за закупуването на хляб, брашно, жито или друго зърно. Някои от съвета в желанието си да вкарат Камарата в още по-затруднено положение посъветваха губернатора да не приема дарението, тъй като не е това, което е поискал; но той отвърна: „Ще взема парите, защото много добре разбирам, какво имат предвид; друго зърно значи барут,“ който той съответно купи, а те не възразиха.\footnote{Виж гласовете. -- Бележка в полето.}

За този случай загатвах по времето, когато в нашата пожарна команда се опасявахме, че предложението ни в полза на лотарията няма да успее, и бях казал на моя приятел г-н Синг, един от членовете, „Ако се провалим, нека да предложим закупуването на пожарна с парите; квакерите не могат да възразят на това; и след това, ако ти номинираш мен, а аз теб, за комисията за тази цел, ще купим едно голямо оръдие, което определено е пожарна.“ „Виждам“, каза той, „че дългото пребиваване в Събранието ти влияе добре; с двусмисления си проект ще надминеш тяхното „жито или друго зърно.“ 

Квакерите изпадаха в такива затруднения, защото бяха установили и обявили като един от принципите си, че всякакъв вид война е незаконен, от който, дори да си променяха мнението, не можеха след това лесно да се отърват, понеже вече беше обнародван. Това ми напомня на по-разумното отношение на една друга от нашите секти, тази на новобаптистите\footnote{Dunkers, NeuTaeufer, виж английската уики}. Запознах се с един от основателите й, Майкъл Уелфеър, малко след като се появи. Той ми се оплака, че поддръжниците на другите изповедания жестоко ги оклеветявали и им приписвали отвратителни принципи и практики, с които те нямали нищо общо. Казах му, че винаги е било така с новите секти, и че си мисля, че може би ще е добре да публикуват принципите на вярата и правилата на дисциплината си, за да сложат край на подобни нападения. Той каза, че такова предложение е било направено сред тях, но не е било прието, по следната причина: „Когато първоначално се събрахме като общност,“ казва той, „Бог беше намерил за угодно да освети умовете ни дотолкова, че да видим, че някои от доктрините, които смятахме по-рано за верни, всъщност са погрешни; и че други, които смятахме за грешни, са в действителност истини. От време на време Му е било угодно да ни осветли още малко, и с времето принципите ни се подобряват, а грешките намаляват. Понастоящем не сме сигурни, че сме достигнали края на този процес и съвършеното духовно или богословско знание; и се опасяваме, че ако отпечатаме символа на вярата ни, може да се почувстваме обвързани и ограничени от него и може би ще изгубим желание да продължим да се подобряваме, а последователите ни още повече, тъй като ще сметнат, че това, което ние, техните старейшини и основоположници, сме направили е нещо свято, от което никога не трябва да се отделят.“

Да се види такава скромност в секта е може би уникално събитие в историята на човечеството, понеже всяка друга секта смята, че разполага с цялата истина, и че всички, които не са съгласни с нея дълбоко се лъжат; както човек, който пътува в мъгливо време, който вижда тези, които са по-напред по пътя, обвити в мъгла, както тези зад него, както и хората в полето от двете страни, но близо до него всичко изглежда ясно, макар в действителност и той да е точно толкова в мъглата, колкото всички тях. За да избегнат този вид затруднения квакерите в последните години постепенно намаляват участието си в обществената служба в Събранието и сред съдиите, избирайки да се откажат по-скоро от властта си, отколкото от принципа.

Ако следвах последователността във времето, трябваше по-рано да спомена, че след като през 1742 г. изобретих отворена печка за по-добро затопляне на стаи и в същото време пестене на гориво, благодарение на това, че свежият въздух, който се всмуква в печката се затопля докато влиза, подарих един модел на г-н Робърт Грейс, един мой приятел от ранните ми години, който като притежател на пещ за леене на желязо намираше отливането на плочите за тези печки за доста печеливша работа, тъй като търсенето растеше. За да насърча търсенето написах и публикувах памфлет с името „Описание на новоизобретените пенсилвански печки; в което са обяснени подробно конструкцията и начинът им на действие; преимуществата им спрямо всички други начини за отопление на стаи представени; и всички възражения срещу тяхната употреба оборени и премахнати”, и т.н. Този памфлет произведе добър резултат. Губернатор Томас беше толкова доволен от конструкцията на печката, както е описана в него, че предложи да ми даде патент, който да ми даде изключителното право да ги продавам в продължение на няколко години; но отказах заради един принцип, който винаги е натежавал в моите очи в такива случаи, а именно, че понеже чуждите изобретения ни принасят голяма полза, трябва да се радваме, ако ни се отвори възможност да послужим на другите със собствено изобретение; и това трябва да правим свободно и щедро.

Един лондонски търговец на желязо, обаче, след като зае голяма част от памфлета ми и го преработи в свой, и след като направи няколко малки промени в устройството, които доста навредиха на действието му, изкара патент за печката там, и ми казват, че направил малко състояние от него. И това не е единственият случай, в който други са взимали патенти за мои изобретения, макар че не винаги с такъв успех, а аз никога не обжалвах, понеже имам малко желание сам да печеля от патентите, а мразя споровете. Употребата на тези печки в много къщи, както в тази, така и в съседните колонии, е помогнала много – и продължава да помага – на населението да пести дърва.

След сключването на мира работата по военната организация приключи и отново насочих мислите си към основаването на академия. Първата стъпка, която предприех беше да включа в плана няколко активни приятеля, повечето от които от Хунтото; следващата беше да напиша и публикувам памфлет наречен „Предложения относно образованието на младежите в Пенсилвания“. Разпространих го сред първенците безплатно; и в момента, в който сметнах, че умовете им са били леко подготвени, започнах да подписка за събиране на средства за отваряне и издръжка на академия; парите щяха да се изплащат ежегодно в продължение на пет години; сметнах, че ако разделя вноските по този начин, подписката ще събере повече, и вярвам, че беше така, защото в крайна сметка възлизаше на не по-малко от пет хиляди паунда, ако си спомням правилно. 

Във въведението на тези предложения обясних тяхното публикуване не със собствената си инициатива, а с тази на някой си господин джентълмен с обществена настройка, избягвайки, доколкото можех – според обичайното си правило – да се представям пред обществеността като автор на каквито и да е проекти в нейна полза. 

За да преминат към директо изпълнение на плана, подписалите избраха измежду себе си двадесет и четири настоятели, и назначиха г-н Франсис, по това време министър на правосъдеито, и мен да съставим устав за управлението на академията; щом това беше сторено беше наета сграда, бяха назначени учители, и училището отвори, струва ми се, в същата година, 1749.

Броят на учениците се увеличаваше бързо, сградата скоро се оказа твърде малка, и се бяхме заели да търсим удобно разположена земя с намерението да строим, когато Провидението хвърли на пътя ни вече построена голяма сграда, която с малки изменения можеше да ни послужи. Това беше сградата спомената по-рано, издигната от слушателите на г-н Уайтфийлд, и се сдобихме с нея по следния начин.

Трябва да се отбележи, че понеже даренията за строежа на тази сграда бяха дело на хора от различни секти, при избора на настоятели, които да се разпореждат със сградата и земята, се внимаваше да не се даде надмощие на която и да е от сектите, за да не би това надмощие след време да бъде използвано като средство за обземане на цялата сграда за употреба от тази секта, в противоречие с първоначалното намерение. По тази причина беше назначаван по един настоятел от всяка секта, а именно, един от англиканската църква, един презвитерианец, един баптист, един моравец Moravian, и т.н., като в случай на смърт, освободеното място се запълваше чрез избор сред благодетелите. Така се случи, че моравецът не се хареса на колегите си, и когато умря, те решиха да не взимат друг от тази секта. При това положение затруднението беше как да се избегне избирането на втори от някоя от другите секти при новия избор. 

Няколко души бяха обсъдени, и по тази причина отхвърлени. Накрая някой ме споменал, като отбелязал, че съм просто честен човек без принадлежност към коя да е секта, което ги убедило да ме изберат. Ентусиазмът, който беше съпътствал строежа на сградата отдавна беше се охладил, а настоятелите не бяха имали успех с набирането на нови средства за наема на земята и покриването на някои други дългове по сградата, което им причиняваше голямо притеснение. Като член на двете групи настоятели – тези на сградата и тези на Академията – имах добра възможност да преговарям и с двете, и в крайна сметка успях да ги доведа до споразумение, според което настоятелите на сградата щяха да я отстъпят на тези на академията, а последните се наемаха да погасят дълговете и винаги да държат на разположение в сградата една голяма зала за случайни проповедници, според първоначалното намерение, както и да поддържат безплатно училище за обучение на бедни деца. Съответните споразумения бяха изготвени в писмен вид, и с плащането на дълговете настоятелите на академията преминаха във владение на имота; чрез разделяне на просторната и висока зала на етажи и на различни стаи за различните училища на долния и горния етаж, цялото скоро беше пригодено за нуждите ни, а учениците преместени в сградата. Работата и грижите по намиране на работници, закупуване на материали и надзираване ан работата се паднаха на мен; и се захванах с нея енергично, още повече че по това време не се застъпваше с частния ми бизнес, тъй като предишната година бях взел taken един много способен, усърден и честен сътрудник, г-н Дейвид Хол, чийто характер познавах много добре/когото познавах много добре, понеже беше работил за мен четири години. Той свали от плещите ми всички грижи около печатницата, като ми плащаше точно моят дял от печалбата. Сътрудничеството ни продължи осемнадесет години, и беше успешно и за двама ни.

Настоятелството на академията след време беше преобразувано в корпорация с разрешение от губернатора; средствата на академията се увеличиха с попълнения от Великобритания и дарения на земя от собствениците на колониите, а оттогава и Събранието направи значителни дарения; по този начин беше основан настоящият Университет на Филаделфия. От началото продължават да ме преизбират за настоятел, вече са близо четиридесет години, и съм имал голямото удоволствие да видя някои от младежите, които получиха образованието си в него, да се отличават с подобрените си умения и да служат на обществени длъжности като украшения за страната.

Когато се освободих от грижата за частния си бизнес, както споменах по-горе, се ласкаех с мисълта, че благодарение на достатъчното, макар и умерено, състояние, което бях придобил, си бях осигурил за остатъка от живота си свободно време за философски въпроси и забавлениe. Купих всичките апарати на д-р Спенс, който беше дошъл от Англия да изнася лекции тук, и продължих с експериментите си с електричество с голяма пъргавина; но въсприемайки ме сега като свободен човек, обществеността ме улови за своите нужди, като всяка част от гражданското ни управление почти едновременно ми възложи някаква отговорност. Губернаторът ме назначи в комисията по мира commision of peace; общината ме избра за общинския съвет, а по късно и за старейшина; а гражданите ме избраха за техен представител в Събранието. последната служба ми се стори още по приятна, понеже в крайна сметка се бях уморил да седя там и да слушам разисквания, в които - като писар - не можех да участвам, и които често бяха толкова безинтересни, че бях принуден да се забавлявам като съставям магически квадрати или кръгове, или каквото и да е, за да не ми досадят; и сметнах, че изборът за член на Събранието ще увеличи способността ми да правя добро. Няма да се преструвам, че амбициите ми не бяха поласкани от всички тези повишения; със сигурност бяха; защото като се вземе предвид скромното ми начало, бяха голямо нещо за мен; и ми бяха още по-приятни, тъй като бяха спонтанни свидетелства за доброто мнение, което обществото имаше за мен, и които не бях търсил.

Опитах се малко да служа като мирови съдия, като гледах няколко дела; но след като установих, че за надеждното изпълняване на тази длъжност са нужни по-добри познания по общо право от моите, постепенно се оттеглих от тази служба, извинявайки се с по-високите си задълженията си като законодател в Събранието. Гласуваха ми доверие за тази работа и ме преизбираха последователно десет години без нито веднъж да помоля който и да е от избирателите да гласува за мен или да показвам директно или индиректно, че искам да бъда избран. Щом бях избран за член на Камарата, синът ми беше назначен за писар. 

На следващата година стана нужда да се сключи споразумение с индианците в Карлайл и губернаторът изпрати съобщение на Камарата, предлагайки да номинира някои от членовете си, които да се присъединят към някои членове на съвета като комисари за тази цел.\footnote{ Виж гласуването the vote, за да опишеш нещата по-точно. --Бележка в полето} Камарата излъчи един говорител (г-н Норис) и мен; и с това задание съответно отидохме в Карлайл и се срещнахме с индианците. 


Понеже тези хора имат изключителна склонност към напиване, а когато са пияни са много свадливи и разюздани, забранихме стриктно да им се продава всякакъв алкохол; а когато се оплакаха от ограничението, им казахме, че ако по време на преговорите се въздържат от пиене, когато приключим с работата, ще им дадем ром в изобилие. Те обещаха да направят така и удържаха на обещанието си, защото нямаше откъде да се снабдят с алкохол, и преговорите се проведоха в много добър порядък и бяха приключени с взаимно удовлетворение. След което те поискаха и получиха ромa; това беше следобеда; бяха близо сто мъже, жени и деца, и бяха настанени във временни колиби издигнати около квадратен площад малко извън града. Вечерта чухме голяма глъч от там и комисарите излязохме да видим какво става. Установихме, че са наклали голям огън в средата на площада; всички бяха пияни – мъжете и жените – и се караха и биеха. Тъмно оцветените им тела, полуголи, осветени само от мъждивата светлина на огъня, гонещи се и налагащи се едно друго с горящи главни, наред с ужасяващите им викове, създаваха сцена, която най-силно наподобява схващанията ни за ада, доколкото можем да си го представим; безредието не даваше признаци на отслабване и ние се върнахме по квартирите си. Към полунощ една част от тях дойдоха  с голяма врява да тропат на вратата ни за повече ром, на което ние не обърнахме внимние. 

Съзнавайки, че беше неприлично да ни смущават по този начин, на другия ден ни изпратиха трима от съветниците си да ни се извинят. Говорителят призна грешката им, но я приписа на рома; и след това се опита да извини рома с думите „Великият Дух, който създаде всичко, създаде всяко нещо с цел, и каквато цел той му е определил, за тази цел трябва винаги да бъде използвано. Сега, когато създаде рома, той каза 'Нека това да бъде за индианците, да се напиват с него,' и така трябва да бъде.“ И действително, ако Провидението планира да изтреби тези диваци, за да освободи място за заселници, изглежда съвсем вероятно ромът да е средството избрано за тази цел. Той вече е унищожил всички племена, които по-рано населяваха крайбрежието.

През 1751 г. д-р Томас Бонд, особено близък мой приятел, замисли идеята за основаване на болинца във Филаделфия (план с много предимства, който ми е бил приписван, но първоначално беше негов) за прием и лечение на бедни хора, както жители на провинцията, така и чужденци. Много ревностно и активно опитваше да събере средства за нея, но понеже предложението беше новост в Америка и първоначално недобре разбрано, той няма голям успех. 

Накрая дойде при мен с комплимента, че обществен проект не е възможно да се прокара без моето участие. „Защото“, казва ми, „тези, които моля да се запишат като дарители, често ме питат, Допита ли се до Франклин за тази работа? И какво мисли той за нея? И когато им кажа, че не съм (понеже си мисля, че е далече от интересите ти), те не се записват, а казват, че ще си помислят.“ Разпитах за естеството на плана и вероятните съпътстващи го ползи, и след като получих от него много задоволителен отговор, не само се записах като дарител, но взех присърце намирането на други, които да се включат в подписката. Преди да започна да търся дарители, обаче, се опитах да подготвя умовете на хората като напиша по темата във вестниците, какъвто беше обичайният ми подход в такива случаи, но което той беше пропуснал.

След това записванията вървяха по-лесно и бяха по-щедри; но започнаха да отслабват и видях, че няма да стигнат без някаква помощ от Събранието, поради което предложих да поискаме такава, което беше сторено. Първоначално членовете от селските райони не харесаха проекта; възразиха, че ще е от полза само за града, и че по тази причина гражданите сами би трябвало да платят за него; и освен това подложиха на съмнение широката подкрепа за него сред гражданите. Противоположните ми твърдения, че за него има такава подкрепа, че без съмнение могат да се съберат две хиляди паунда с доброволни дарения, те сметнаха за екстравагантно предположение и напълно невъзможно.

На това изградих плана си; и след като поисках позволение да вкарам проектозакон за преобразуване на дарителската организация в корпорация, според искането на дарителите, и за отпускането на определена сума пари за техните цели – което позволение беше дадено основно заради съображението, че Камарата може да отхвърли закона, ако не го хареса – го изготвих така, че най-важната клауза да е условна, сиреч „И се постановява, според гореспоменатото право, че, след като гореспоменатите дарители са се събрали и са избрали управителите си и ковчежник, и са събрали от собствените си дарения капитал в размер на ----- (чиято годишна лихва ще се използва за настаняване на бедни болни в гореспоменатата болница, свободно от заплащане за храна, грижа, консултации и лекарства), и са представили задоволителни доказателства за горното пред говорителя на Събранието, ще бъде законно и позволено на този говорител, и той е задължен според настоящия закон да го направи, да подпише заповед към ковчежника на провинцията за изплащането на две хиляди паунда в две годишни вноски на ковчежника на гореописаната болница, които да бъдат използвани за основаването, изграждането и завършването на същата.“ 

Това условие прокара закона; защото членовете, които бяха против отпускането на средства, сега си въобразиха, че ще могат да си спечелят име за благотворителност без да им се наложи да платят, и го подкрепиха; след това, когато събирахме дарения сред хората, използвахме условното обещание на закона като допълнителна причина да се дарява, тъй като дарението на всеки ще бъде удвоено; така клаузата сработи и по двата начина. Съответно даренията скоро надхвърлиха нужната сума, и ние поискахме и получихме обществения дар, което ни позволи да изпълним плана. Скоро беше издигната удобна и хубава сграда; постоянната употреба е доказала полезността на институцията и тя процъфтява до днес, и не си спомням за друг мой политически маньовър, чийто успех да ми е доставил по-голямо удовослтвие, или при който – след размишление – по-лесно да съм оправдавал употребата на хитрост.

По горе-долу това време един друг ръководител на проекти, преподобният Гилбърт Тенънт, дойде при мен с молба да му помогна в набирането на средства за нова сграда за събирания. Щеше да се ползва от група богомолци, които той беше събрал сред презвитерианците, които първоначално били ученици на г-н Уайтфийлд. Понеже не желаех да досаждам на съгражданите си с твърде чести молби за дарения отказах твърдо. Тогава ме пожела да му предоставя списък с имената на хората, за които от опит знам, че са щедри и с обществен дух. Сметнах, че няма да е удачно, след като любезно бяха откликнали на молбите ми, да ги посоча, за да бъдат тормозени от други просяци, и затова отказах да дам и такъв списък. Тогава пожела поне да му дам съвет. „Това ще направя с готовност,“ казах аз; „и на първо място те съветвам да да се обърнеш към тези, които знаеш, че ще дадат нещо; след това към тези, за които не си сигурен дали ще дадат нещо или не; и накрая не пропускай тези, за които си сигурен, че няма да дадат, понеже за някои от тях може да грешиш.“ Той се засмя и ми благодари, и каза, че ще последва съвета ми. Наистина го направи, защото не пропусна никой да помоли, и събра много повече средства, отколкото очакваше, с които издигна обширната и много елегантна сграда за събирания на Арк стрийт. 

Въпреки че градът ни беше разгърнат с красива равномерност, а улиците бяха широки, прави, и се пресичаха под прави ъгли, за срам той допусна тези улици да останат непавирани дълго време и в мокро време колелата на тежките коли ги разораваха в море от кал, така че беше трудно да бъдат пресечени; а при сухо време прахът беше крайно неприятен. Живеех had lived близо до пазара, който наричаха Джързи маркет и наблюдавах с болка как жителите газят в калта докато си купуват провизии. Накрая една ивица земя преминаваща през средата на пазара беше покрита с тухли, така че, веднъж стигнали до пазара, имаха твърда земя под краката си, но за да стигнат до там често бяха в кал до над обувките. Като говорих и писах по темата в крайна сметка предизвиках павирането на улицата с камък между пазара и тухления тротоар до къщите от двете страни на улицата. За известно време това осигури лесен сух достъп до пазара; но понеже останалата част от улицата бе беше павирана, когато кола излизаше от калта и стъпваше на паважа отърсваше и оставяше калта си върху него и скоро и скоро и той беше покрит с кал, която никой не отстраняваше, понеже градът още нямаше чистачи. 

След като разпитах, намерих един беден работлив човек, който беше готов да се заеме с чистенето на улицата, като я премита два пъти седмично и отнася калта от пред къщите на всички съседи, срещу сумата от шест пенса месечно, които да бъдат плащани от всяка къща. Тогава написах и отпечатах статия, която излагаше предимствата, които кварталът би могъл да получи срещу тази малка сума; по-лесното поддържане на чистота в къщите, понеже няма да се внася толкова кал с обувките на хората; ползата за магазините от повечето продажби и т.н., и т.н., тъй като купувачите по-лесно ше могат да стигнат до тях; и понеже във ветровите време прахът няма да покрива стоката и т.н., и т.н. Изпратих по едно копие от тази статия на всяка къща и след ден-два минах да проверя кой би подписал договор да плаща тези шест пенса; беше единодушно подписан и за известно време изпълняван стриктно. Всички жители на града бяха възхитени от чистотата на паважа около пазара, понеже беше от полза за всички, и това увеличи всеобщото желание да бъдат павирани всички улици, и направи хората по-склонни да бъдат обложени с данък за тази цел. 

След известно време изготвих проектозакон за павирнето на града и го внесох в Събранието. Това беше точно преди да отида в Англия, в 1757 г., и не беше приет докато ме нямаше\footnote{Виж резултатите от гласуването. – Бележка в полето}, а след това само с промяна в начина на оценка, която според мен не беше за добро, но предвиждаше допълнително и изграждането на улично осветление, освен павирането на улиците, което беше голямо подобрение. Идеята за осветяване на целия град първоначално впечатли хората благодарение на личния пример на покойния г-н Джон Клифтън, който нагледно демонстрира ползата от лампите като постави една пред вратата си. Заслугата за тази обществена придобивка също ми е била приписвана, но в действителност е негова. Аз единствено последвах примера му и мога да претендирам само за известен принос по отношение на формата на лампите ни, които са различни от кълбовидните лампи, които първоначално ни доставяха от Лондон. Тези се оказаха неудобни в това отношение: не пропускаха въздух отдолу; по тази причина пушекът  не излизаше лесно отгоре, а циркулираше в кълбото, отлагаше се от вътрешната страна, и скоро замъгляваше светлината, която  трябваше да дават; като освен това създаваше грижа по ежедневното им чистене; а случаен удар беше в състояние да счупи лампата и да я направи напълно безполезна. По тази причина предложих направата на лампи с четири плоски страни, дълга фуния отгоре, която да тегли пушека, и отвори отдолу, които да пропускат въздух, който да улесни издигането на пушека; по този начин лапмпите оставаха чисти и не се затъмняваха за няколко часа като лондонските лампи, а продължаваха да светят силно до сутринта, а случаен удар най-често счупваше само една страна, което лесно се поправяше. 

Понякога съм се питал защо лондончани не се научиха да правят дупки на уличните си лампи виждайки, че дупките от долната стана на лампите, които се използват във Воксхал, ги държат чисти. Но изглежда че понеже тези дупки са направени с друга цел, а именно, да се пали фитилът по лесно посредством ленено влакно, което виси през тях, другата полза от тях е останала незабелязана; и затова няколко часа след запалването на лампите лондонските улици са зле осветени.

Тези подобрения ми напомнят за едно, което предложих когато бях в Лондон на д-р Фодъргил, който е един от най-добрите мъже, които познавам, и голям насърчител на полезни проекти. Забелязах, че в сухо време не метяха улиците и не отнасяха лекия прах; а го оставяха да се събере, докато мокрото време не го превърнеше в кал, и тогава, след като отлежеше няколко дни толкова дълбока, че нямаше пресичане освен по пътеки поддържани чисти с метли от бедни хора, с голямо усилие беше събирана с гребла и товарена на отворени коли, чийто страни позволяваха част от мръсотията да се разплисква и пада при всяка неравност по пътя, понякога притеснявайки пешеходците. Причината която даваха да не се метат улиците в сухо време беше, че прахът ще влезе в прозорците на магазините и къщите. 

Една неочаквана случка ми показа колко метене може да бъде свършено за малко време. Еда сутрин заварих пред вратата ми на Крейвън стрийт една бедна жена да мете паважа ми с брезова метла; изглеждаше много бледа и слаба, като току що прекарала болест. Попитах я кой я е наел да мете там; тя каза, „Никой, но съм много бедна и в нужда и метем пред вратите на ботагите, и се надяваме да ни дадат нещо.“ Помолих я да измете цялата улица и и казах, че ще и дам един шилинг; това беше в девет часа; в дванадесет дойде за шилинга. Заради бавната скорост с която я видях да работи в началото ми беше трудно да повярвам, че е успяла да свърши работата толкова скоро, и изпратих слугата си да провери, който докладва, че цялата улица е изметена до блясък, а всичкият прах е събран в канавката, която беше по средата; а следващият дъжд го отнесе всичкия, така че паважът и дори каналът бяха съвършено чисти. 

От това прецених, че ако една слаба жена може да измете такава улилца за три часа, силен, пъргав мъж би могъл да го направи за половината време. И ми позволи тук да отбележа предимството на това да има само една канавка в средата на такава тясна улица, вместо две, една от всяка страна, близо до пътя за пешеходците; когато вали, дъждът от цялата улица се събира в средата в течение достатъчно силно, за да отнесе всичката кал, която попадне на пътя му; но ако се раздели в два канала, често е твърде слабо, за да отнесе калта от който и да е от двата, и само я втечнява, така че колелата на колите и конските копита я разнасят и оставят по паважа, който се замърсява от това и става хлъзгав, а понякога я разплискват по пешеходците. Предложението, което изпратих на добрия доктор, беше следното:

„За по-ефективното чистене и поддържане на чистотата на улиците в Лондон и Уестминстър се предлага да бъдат наети няколко отговорници, които да замитат праха в сухо време и да събират калта в останалото време, всеки в няколкото улици и алеи в неговия район; да им бъдат осигурени метли и другите нужни инструменти за тази цел, които да се съхраняват в съответните сергий stands, готови да бъдат раздадени на бедните хора, които могат да наемат за тази работа. 

В топлите летни месеци прахът да се събира на купчини на удачно отстояние преди да се отворят магазините и прозорците на къщите, и да бъде изнасян по същото време с покрити коли от чистачите. 

Когато се събира, калта да не бъде оставяна на купчини, които да бъдат разнасяни отново от колелата на колите и краката на конете, а чистачите да бъдат снабдени с носилки, не вдигнати високо на колела, а ниски като шейни, с решетъчни дъна, които покрити със слама да задържат калта хвърлена в тях, но да позволяват на водата да се оттече, което ще ги олекотява значително, тъй като водата придава по-голямата част от тежестта и; тези шейни да бъдат поставяни на удобно разстояние, а калта да бъде донасяна до тях посредством колички; а те да остават където са докато калта се оцеди, когато ще бъдат отнасяни с коне.“

Оттогава се породиха в мен известни съмнения относно изпълнимостта на последната част от предложението, тъй като някои от улиците са тесни и би било трудно да се поставят шейни за оцеждане без да затруднят преминаването твърде много; но все още смятам, че първото предложение, да се измита и изнася праха преди да отворят магазините, е много лесно изпълнимо през лятото, когато дните са дълги; защото вървейки по Странд и Флийт стрийт в седем часа една сутрин забелязах, че нито един от магазините не беше отворен, макар да беше три часа след изгрев; жителите на Лондон предпочитайки доброволно да живеят повече на светлината от свещи и да спят на слънце, и все пак да се оплакват често, и донякъде абсурдно, от данъка върху свещите и високата цена  на лойта.

Някой може да помисли, че не си струва да се мисли за тези дреболии, или да се споменават; но след като вземе предвид, че макар и прахът, който ще влезе в очите на един човек или в един магазин във ветровит ден, да не е от голямо значение, все пак големият брой случаи в един многолюден град и честото им повторение придава на проблема тежест и значимост, може би няма да порицае жестоко тези, които отделят известно внимание на тези привидно незначителни въпроси. Човешкото щастие е резултат не толкова на случаи на голям късмет, които рядко ни сполетяват, колкото на малки ежедневни удобства. Така ако научиш беден млад човек как да се бръсне и да си поддържа бръснача в ред, може би ще допринесеш повече за щастието в живота му, отколкото ако му дадеш хиляда гвинеи. Парите може скоро да бъдат похарчени оставяйки само съжалението, че са били прахосани за глупости; но в другия случай на него му се спестява честото раздразнение от чакането при бръснари, от понякога мръсните им пръсти, лош дъх и тъпи бръсначи; бръсне се в най-удобното за него време и ежедневно изпитва удоволствието, че това става с добър инструмент. С такива размишления наум си позволих предните няколко страници, надявайки се, че може би ще се намерят в тях хрумвания, които по едно или друго време могат да бъдат полезни на един град, който обичам, и в който живях щастливо много години, а може би и на някои от нашите градове в Америка. 

След като известно време бях зает при министъра на пощите на Америка като контрольор върху дейността на няколко служби и потърсих сметка за грешките на ръководителите им, при смъртта на министъра през 1753 г. бях посочен от министъра на пощите в Англия за негов наследник, заедно с г-н Уилям Хънтър. Американският клон до този момент не беше плащал нищо на британския. Двамата щяхме получаваме общо шестстотин паунда годишно, ако можехме да изкараме толкова печалба. За да успеем, имаше нужда от различни подобрения; някои от тях неминуемо изискваха първоначално вложение, така че през първите четири години пощите задлъжняха спрямо нас с деветстотин паунда. Но скоро започна да ни се отплаща; и преди да бъда сменен поради каприз на министрите, за което ще кажа по-късно, бяхме успели да увеличим чистите приходите за короната до три пъти тези от Ирландските пощи. Откакто направиха онзи неразумен ход не са получили от тях нито петак. 

През тази година работата в пощите ме отведе на пътешествие до Нова Англия, където Кембриджкия колеж, по собствена подбуда, ме удостои със степента Магистър на изкуствата. Колежът Йейл в Кънектикът по-рано ми беше направил подобен комплимент. Така без да съм учил в колеж споделих академичните им почести. Бяха ми дадени заради подобренията и откритията ми в този дял от естествената философия, който се занимава с електричеството.

През 1754, поради заплахата от война с Франция, със заповед на the Lords of the Trade, беше свикан конгрес от представители на различните колонии, който трябваше да се събере в Олбани, и да обсъди с вождовете на Шестте нации средствата за защита на тяхната и на нашата страна. След като получи нареждането за това, губернатор Хамилтън съобщи на Камарата за него и поиска членовете й да подготвят подходящи подаръци за индианците, които да бъдат подарени на срещата; и избра говорителя (г-н Норис) и мен да се присъединим към г-н Томас Пен и секретаря г-н Питърс като представители на Пенсилвания. Камарата одобри номинацията и предостави материали за подаръка, въпреки че не им хареса много; и се срещнахме с другите представители в Олбъни към средата на Юни.

По пътя за натам проектирах и нахвърлях един план за съюзяване на всички колонии под едно правителство, доколкото би било нужно за защита и други важни общи цели. Когато минахме през Ню Йорк показах там проекта си на г-н Джеймс Алегзандър и г-н Кенеди, двама джентълмени, които знаят много за обществените дела, и, насърчен от тяхното одобрение си позволих да го представя пред Конгреса. Тогава се оказа, че няколко от представителите бяха изготвили подобни планове. Първо беше поставен предварителен въпрос, дали трябва да се основе съюз, което беше прието с единодушие. След това беше съставена комисия с по един представител от всяка колония, която да разгледа плановете и да докладва report. Така се случи, че моят беше избран, и съответно, с няколко изменения, беше докладван.

Според този план общото правителство щеше да бъде ръководено от президент, назначен и издържан от короната, и голям съвет, който да бъде избиран от представителите на хората в колониите в съответните им събрания. Дебатите по този въпрос в Конгреса вървяха ежедневно, успоредно с индианската работа. Повдигнати бяха много възражения и изникнаха много трудности, но в крайна сметка всички бяха преодолени, и планът беше единодушно одобрен, и бяха поръчани копия, които да бъдат пратени на the Board of Trade и на събранията на различните провинции. Съдбата му беше особена: събранията не го приеха, тъй като сметнаха, че дава твърде много привилегии на короната, а в Англия го намериха за твърде демократичен. 

По тази причина the Board of Trade не го одобри, нито го препоръча за одобрението на негово величество; вместо това беше съставен друг план, който уж предлагаше по-добри решения на същите проблеми, и според него губернаторите на провинциите заедно с някои членове на съответните си съвети щяха да се съберат и да наредят  набирането на войска, строежа на укрепления и т.н. използвайки средства от хазната на Великобритания, които после щяха да бъдат изплатени чрез закон за облагане на Америка с данък приет от Парламента. Планът ми, заедно с доводите ми в негова подкрепа, може да бъде намерен сред политическите ми текстове, които са публикувани.

Следващата зима бях в Бостън и имах възможност да разговарям надълго с губернатор Шърли относно двата плана. Част от нещата, които си казахме по това време също могат да се намерят в тези текстове. Различните и противоположни причини, поради които хората не харесваха плана ми, ме карат да подозирам, че действително беше правилната среда; и все още съм на мнение, че щеше да е по-щастливо и за двете страни на водата, ако беше възприет. Обединени по този начин, колониите щяха да са достатъчно силни, за да се защитят; нямаше да има нужда да се праща войска от Англия; разбира се последвалият предлог за облагане на Америка с данък и кървавата разпра, която причини, щяха да бъдат избегнати. Но такива грешки не са нови; историята е пълна с грешките на държави и принцове.

Огледай обитаемия свят – малцина знаят
що е за тях добро, или пък знаейки, го следват!

Тези, които са заети с управлението, понеже са много натоварени с работа, като цяло не обичат да правят усилие да мислят за нови проекти и да ги изпълняват. По тази причина най-добрите обществени мерки рядко се приемат благодарение на уроците на историята, а по-скоро се налагат от случая. 

Губернаторът на Пенсилвания, когато изпращал плана на Събранието, изразил одобрението си за него „тъй като му се струвало, че е изготвен с голяма яснота и сила на разума, и затова го препоръчал като съвсем достоен за тяхното най-концентрирано и искрено внимание.” Под ръководството на един определен член Камарата обаче се захвана с него по време, по което отсъствах, което намерих за не много справедливо, и без да му обърне каквото и да е внимание го отхвърли, което ми причини немалко унижение.

По време на пътуването ми до Бостън през тази година се срещнах нашия нов губернатор, г-н Морис, който току що беше пристигнал в града от Англия, и с когото бях близък от по-рано. Той беше довел със себе си комисия, която да замени г-н Хамилтън, който беше подал оставка уморен от пререканията, в които инструкциите, които получаваше от собствениците, го караха да влиза. Г-н Морис ме попита дали и той трябва да очаква толкова неудобно управление. Аз казах, „Не; напротив, можеш да очакваш много удобно управление, ако само внимаваш да не влизаш в спорове със Събранието.“ „Мили приятелю“, казва той доволно, “как можеш да ме съветваш да избягвам споровете? Знаеш, че обичам споровете; те са едно от най-големите ми удоволствия; въпреки това, за да покажа, че уважавам съвета ти, ти обещавам, че ако е възможно ще ги отбягвам.“ Той имаше известни причини да обича да спори, понеже беше красноречив, остроумен софист, и по тази причина като цяло имаше успех в споровете. От дете беше възпитан по този начин, като съм чувал, че баща му създал у децата си навика да спорят едно с друго за негово забавление докато седял на масата след вечеря; но смятам, че тази практика е неразумна; защото според моите наблюдения тези спорещи, противоречащи и оборващи хора като цяло са нещастни в делата си. Понякога печелят победи, но никога не спечелват добра воля, от която биха имали най-много полза. Разделихме се, той продължавайки за Филаделфия, а аз за Бостън.

На връщане научих в Ню Йорк за решенията на Събранието, които подсказваха, че въпреки това, което ми беше обещал, вече беше в голяма препирня с Камарата; и докато той беше губернатор битката не спираше. И аз взех участие в нея; защото, щом се върнах в Събранието, ме назначиха във всички комитети заети с отговаряне на речите и съобщенията му, а в комитетите винаги искаха от мен да изготвя първата чернова. Отговорите ни, като съобщенията му, често бяха хапливи, а понякога непристойно обидни; и понеже знаеше, че пиша за Събранието, човек можеше да си помисли, че щом се срещнем едва можем да не си прережем гърлата; но той беше толкова добродушен човек, че сблъсъците не породиха лични дрязги между нас и често обядвахме заедно. 

Един следобед, в разгара на тази обществена препирня, се срещнахме на улицата. „Франклин“, казва той, „трябва да дойдеш с мен у дома и да останеш за вечерта; ще имам компания, която ще ти хареса;“ и като ме хвана под ръка ме отведе в къщата си. Във веселия разговор на по чаша вино след вечеря той ни каза, на шега, че много се възхищавал на идеята на Санчо Панса, който поискал да управлява черен народ когато му предложили да бъде губернатор, тъй като в този случай, ако не можел да се съгласи с поданиците си, би могъл да ги продаде. Един от приятелите му, който седеше до мен, ми каза „Франклин, защо още си на страната на проклетите квакери? Не е ли по-добре да ги продадеш?“ „Губернаторът“, казах аз, „Още не ги е очернил достатъчно.“ Той действително се беше постарал във всичките си съобщения да очерни Събранието, но те избърсваха чернилката със скоростта, с която той я полагаше и му отвръщаха, като я поставяха сгъстена обратно на неговото собствено лице; затова установявайки, че е вероятно сам да бъде негрофициран, и той като г-н Хамилтън се умори от сблъсъците и остави губернаторството.\footnote{Законите ми по времето на Морис, военни и т.н. – Бележка в полето}

Всички тези обществени караници бяха причинени от собствениците, нашите наследствени губернатори, които, винаги когато се налагаше да се харчат пари за отбраната на провинцията им, с невероятна лошотия даваха на наместниците си инструкции да не одобряват закона за събиране на съответните данъци, ако огромните им имения не са изрично освободени в същия закон; и дори бяха карали наместниците да полагат клетва, че ще следват тези инструкции. В продължение на три години Събранията се противопоставяха на тез инструкции, макар че накрая бяха принудени да отстъпят. Накрая капитан Дени, който наследи губернатор Морис, си позволи да пренебрегне тези инструкции; ще разкажа по-долу как се стигна до всичко това.

Но избързах с разказа си: все още има някои събития, които се случиха по времето на управлението на губернатор Морис, които трябва да бъдат споменати. 

Беше започнала един вид война с Франция и правителството на Масачузетс Бей очакваше атака срещу Краун Пойнт и изпрати г-н Куинси в Пенсилвания и г-н Паунол, по-късно губернатор Паунол, в Ню Йорк за да помолят за помощ. Аз му продиктувах обръщението му към тях, което беше добре прието. Те гласуваха помощ в размер на десет хиляди паунда в продоволствия. Но губернаторът отказа да подкрепи закона (който включваше тази сума наред с други суми отпуснати за нуждите на короната), освен ако не се влючи текст, който освобождава имотите на собствениците от понасянето на каквато и да е част от нужната данъчна тежест, и Събранието, макар че имаше голямо желание да отпусне средствата за Нова Англия, не знаеше как да го направи. Г-н Куинси положи много усилия за да издейства съгласието на губернатора, но той се инатеше. 

Тогава предложих един начин да свършим работата без губернатора, чрез платежно нареждане към попечителите на, което по закон Събранието имаше право да тегли. Наистина по това време в касата office имаше малко или никакви пари, и по тази причина предложих падежът на нарежданията да е след една година и да носят лихва от пет процента. С тези нареждания предположих, че продоволствията лесно ще могат да бъдат закупени. След много малко колебание, Събранието прие предложението. Нарежданията бяха отпечатани веднага, а аз бях в комисията която трябваше да ги подпише и продаде. Щяха да бъдат изплатени от лихвата на всички книжни пари дадени по това време на заем в провинцията, както и от приходите от акциз, и понеже се знаеше, че това ще е предостатъчно, веднага бяха осребрени, и бяха приети не само за закупуване на продоволствия, а много от хората с пари, които имаха в брой под ръка, ги вложиха в тези нареждания, които намериха за доходоносни, тъй като принасяха лихва докато бяха в тях, а можеха във всеки момент да бъдат използвани като пари; така че всичките бяха изкупени с желание и след няколко седмици не бяха останали въобще. Така тази важна работа беше свършена с моите средства. Моят(? Или Г-н) Куинси се отблагодари на Събранието с хубава възпоменателна реч, отиде си много доволен от успеха на мисията си, и оттогава таи към мен най-сърдечни приятелски чувства.

Британското правителство, избирайки да не позволи съюзяването на колониите, което беше предложено в Олбани, и по този начин да не довери тяхната отбрана на съюза, да не би да не станат от това твърде войнствени  и да почувстват собствената си сила – понеже по това време имаше спрямо тях ревности и подозрения – изпрати за тази цел генерал Брадък с два полка редовна английска войска. Той акостира в Александрия, Вирджиния, и оттам направи преход до Фредериктаун, Мериленд, където спря за коли. Опасявайки се поради някакви сведения, че е развил страшни предразсъдъци спрямо тях, че не са склонни да му съдействат, нашето Събрание пожела да го посетя, не като техен пратеник, а като министър на пощите, под претекст, че искам да му предложа да уговорим начина, по който най-бързо и сигурно да се предават съобщения между него и губернаторите на различните провинции, с които е необходимо постоянно да кореспондира, и за което те предложиха да платят. Синът ми ме придружи по време на това пътуване.

Заварихме генерала във Фредериктаун, чакайки нетърпеливо завръщането на тези, които беше изпратил из селата на Мериленд и Вирджиния да събират коли. Останах с него няколко дни, вечерях с него ежедневно, и имах възможност да премахна всичките му предразсъдъци като му дадох информация за това какво всъщност беше сторило събранието преди неговото идване и какво още имаха желание  да направят, за да улеснят действията му. Тъкмо преди да си замина се върнаха изпратените да събират коли, от което стана ясно, че са събрани само двадесет и пет, а не всички от тях бяха годни за употреба. Генералът и всичките офицери бяха изненадани и обявиха експедицията за приключена, тъй като ставаше невъзможна, и възнегодуваха срещу невежество на министрите, което ги беше отвело в край, в който няма с какво да си пренесат запасите, багажите и т.н., за което бяха нужни поне сто и петдесет коли.

Аз случайно казах, че смятам, че е жалко, че вместо това не са дебаркирали в Пенсилвания, тъй като в този край почти всеки фермер си има собствена кола. Генералът жадно eagerly се хвана за думите ми и каза „Тогава вие, сър, който сте човек с влияние там, вероятно можете да ни ги осигурите; моля ви да се заемете с това.“ Попитах какви условия ще бъдат предложение на собствениците на колите; и помолих да поставя на хартия условията, които на мен ми се струваха нужни. Така и направих, а условията бяха приети, и незабавно бяха съставени комисия и инструкции. Тези условия могат да се видят от съобщението, което разпространих щом пристигнах в Ланкастър, което е любопитно заради големия и внезапен ефект, който произведе. Ще го предам цялото, както следва:

               "СЪОБЩЕНИЕ.
                        "Ланкастър, 26 Април, 1755.

Понеже за служба на силите на негово величество, които сега се събират при Уилс Крийк, са нужни сто и петдесет коли с впряг от четири коня към всяка кола и хиляда и петстотин оседлани или товарни коне, а негово превъзходителство ген. Брадък благоволи да ме упълномощи да сключвам договори за наем на гореспоменатите, обявявам, че с тази цел ще бъда в Ланкастър от днес до следващата сряда вечер и в Йорк от следващия четвъртък сутринта до петък вечерта, където ще съм готов да приемам коли и впрягове или самостоятелни коне при следните условия:

1. За всяка кола с четири годни коня и колар ще бъдат заплатена дневна надница от петнадесет шилинга; за всеки здрав кон със седло за товари или друго седло и принадлежности– два шилинга на ден; и за всеки здрав кон без седло – осемнадесет пенса на ден. 2. заплатата ще се брои от момента в който се присъединят към войската при Уилс Крийк, което трябва да е преди или на идния двадесети май, и ще бъде изплатено разумно обезщетение за времето нужно за пътуването до Уилс Крийк и обратно до дома след освобождаване от службата. 3. всяка кола с впряг и всеки оседлан кон ще бъдат оценени от безпристрастни лица избрани от мен и собственика; и в случай на загуба на колата, впряга или други коне по време на служба, ще бъде отпуснато и изплатено обезщетение съгласно оценката. 4. при нужда при сключването на договора на собственика на всяка кола и впряг или кон ще бъде платен от мен на ръка аванс за седем дни, а остатъкът ще бъде платен от ген. Брадък или от ковчежника на армията, при освобождаване от служба, или от време на време, според желанието на собственика. 5 никои от коларите или хората, които се грижат за наетите коне нямат задължение и няма да бъдат привиквани при никакви обстоятелства да служат като войници или да извършват дейност различна от това да се грижат за колата и конете си. 6. Всичкият овес, царевица, или друг фураж, който колите или конете ще донесат в лагера, и който надхвърля нужното за прехрана на конете ще бъде взет за нуждите на армията като за същия ще бъде заплатена разумна цена.

Бележка. -- Синът ми, Уилям Франклин, е упълномощен да сключва такива договори с всички лица в окръг Къмбърленд.

                                        "Б. ФРАНКЛИН"

До жителите на окръците Ланкастър, 
    Йорк и Къмбърленд.“

"Приятели и сънародници,

Оказвайки се случайно в лагера при Фредерик преди няколко дни, заварих генералът и офицерите изключително раздразнени поради това, че не са успели да се снабдят с коне и коли, както са очаквали, че ще могат да направят в тази провинция; но поради разногласия между губернатора и Събранието не са били отпуснати пари, нито са били направени постъпки за тази цел.

Беше направено предложение незабавно в тези окръзи да се изпратят въоръжени части, които да вземат толкова от най-добрите коли и коне, колкото трябват, и да принудят толкова хора да се заемат с грижата по конете и карането на колите, колкото са нужни.

У мен се породи опасението, че преминаването на британските войски през тези окръзи с тази цел, особено като се вземе предвид настроението в което са и негодуванието което изпитват към нас, би било съпътствано от големи неудобства за жителите, и по тази причина с по-голямо желание се захванах да опитам първо какво би могло да се направи с честнии разумни средства. Хората от тези отдалечени back окръзи неотдавна се оплакаха на Събранието, че нямат достатъчно пари; отваря ви се възможност да получите и разделите помежду си доста значителна сума; защото ако експедицията продължи, както е повече от вероятно, повече от сто и двадесет дни, наемът за тези коли и коне ще възлезе на повече от тридесет хиляди паунда, които ще ви бъдат платени в сребро и злато от кралската хазна.

Службата ще бъде лека и лесна, понеже армията рядко ще прави преходи от повече от дванадесет мили на ден, а колите и конете с багаж трябва да се движат заедно с армията и не по-бързо от нея, тъй като пренасят неща, които са жизненоважни за доброто състояние на войската; и за доброто на войската биват поставени на най-сигурно място, както по време на преход, така и по време на престой in a camp. 

Ако действително сте добри и верни поданици на негово величество, каквито вярвам, че сте, сега имате възможност да му послужите по най-удачен начин и да си улесните живота; защото ако трима или четирима ако не могат поотделно да заделят от стопанството си си кола и четири коня и колар, то заедно могат да го направят, един предоставяйки колата, друг – кон или два, а трети – коларя, и да си разделят парите; но ако не искате доброволно да послужите на своя крал и на страната си, когато ви се предлагат такова добро заплащане и такива разумни условия, верността ви ще бъде подложена на големи съмнения. Работата на краля трябва да бъде свършена; толкова голям брой войска, дошла от толкова далече за вашата защита не трябва да стои без работа, заради вашето нежелание да да свършите това, което е разумно да се очаква от вас; коли и коне трябва да бъдат намерени; най-вероятно ще бъде използвано насилие и ще трябва да търсите обезщетение от неволята, а случаят ви може би няма да привлече внимание или състрадание. 

Аз нямам личен интрес в тази работа, тъй като освен удовлетворението от опита да направя добро, единственото възнаграждение за усилията ми ще бъде труда ми. Ако този подход за набавяне на коли и коне се окаже неуспешен, ще трябва да пратя известие на Сър Джон Сейнт Клеър, хусаря, който незабавно ще навлезе в провинцията с войска с тази цел, което много ще ме натъжи, защото съм много искрен и истински ваш приятел и доброжелател, Б. ФРАНКЛИН“

Получих от генерала около осместотин паунда, които да бъдат раздадени като аванс на собсвтвениците на колите и т.н.; но понеже сумата се оказа недостатъчна, дадох в аванс двеста пуанда от собствените си пари, и след две седмици сто и петдесетте коли с двеста петдесет и деветте товарни коня бяха на път към лагера. Съобщението обещаваше изплащането на обезщетение съгласно оценката в случай на загуба на колата или коня. Твърдейки, че не познават ген. Брадък, или колко може да се разчита на обещанията му, собствениците обаче настояха да стана гарант за това, което и направих. 

Една вечер докато бях в лагера и вечерях с офицерите от полка на Дънбар, той изложи пред мен тревогата си относно подчинените си, които – по негови думи – като цяло не били богати и трудно можели да си позволят в тази скъпа страна да си набавят нужните запаси за такъв дълъг преход през пустош, където няма къде да се купи каквото и да е. Аз им съчувствах и реших да потърся начин да им помогна. Въпреки това не му казах нищо за намеренията си, а вместо това на следващата сутрин писах на комисията на Събранието, която имаше на разположение известно количество обществени средства, и сърдечно ги насърчих да разгледат случая на офицерите и им предложих да изпратя подарък с необходимите неща и закуски. Синът ми, който имаше известен опит с лагерния живот и нуждите му, ми  изготви списък, които приложих в писмото си. Комитетът одобри и с такова усърдие се зае с работата, че – водени от сина ми – запасите пристигнаха в лагера заедно с колите. Състояха се от двадесет пакета, всеки от които съдържаше:

  6 фунта буци бяла захар           1 Глостърко сирене
  6 фунта добра нерафинирана захар .            1 буренце с 20 фунта добро масло
  1 фунт добър зелен чай
  1 фунт добър черен чай                    2 дузини стара Мадейра
  6 фунта добро мляно кафе      2 галона ямайски ром 
  6 фунта шоколад           1 бутилка суха горчица.
  1-2 центнера най-добра бяла бисквита  2 два добре опушени бута
  1-2 фунта пипер           1-2 дузини сушени езици
  1 кварта най-добър бял винен оцет     6 фунта ориз
                                       6 фунта стафиди

Тези двадесет пакета, добре опаковани, бяха качени на също толкова на брой коне и всеки пакет заедно с коня беше предназначен за подарък за един офицер. Бяха приети с голяма благодарност, а ръководителите на двата полка изразиха признателността си за любезността в най-благодарствени писма до мен. Генералът също беше много доволен от начина, по който се справих с намирането на колите и т.н., и с готовност възстанови парите, които бях дал в аванс, благодари ми многократно и пожела да продължа да му съдействам като изпращам провизии след армията. Заех се и с тази работа и се занимавах с нея докато не чухме за разгрома им, като дадох повече от хиляда паунда стерлинги от собствените си пари за тази услуга, за което му изпратих сметка. За мой късмет тя го достигнала няколко дни преди битката и той веднага ми върна платежно нареждане до ковчежника за кръглата сума от хиляда паунда, като остави остатъка за следващата сметка. Смятам това плащане за добър късмет, тъй като никога не успях да си върна остатъка, за което ще разкажа повече по-долу.

Мисля, че този генерал беше смел човек и може би щеше да се справи добре като офицер в някоя европейска война. Но имаше прекалено голямо самочувствие, твърде високо мнение за способностите на редовната войска и твърде ниско за американците и индианците. Джордж Крогън, нашият преводач от индиански, се присъедини към него в похода му със сто индианеца, които можеха да бъдат от голяма полза за армията като водачи, разузнавачи и т.н., ако се беше отнесъл добре с тях; но той се отнесъл презрително с тях и не им обръщал внимание, и те постепенно го напуснали.

В разговор с мен един ден той изложи пред мен намеренията си. „След като превзема Форт Дюкен,” казва той, „ще продължа към Ниагара; и след като превзема и него, към Фронтнак, ако времето ми позволи; и предполагам, че така ще стане, защото Дюкен едва ли може да ме задържи повече от три или четири дни; а не виждам какво може да попречи на похода ми към Ниагара.“ Понеже бях обмислял преди това как по време на похода по тесния път, който ще бъде проправен за армията му през гора и храсти, тя ще трябва да се разтегли в дълга линия, както сведенията за минал един разгром на хиляда и петстотин французи, които бяха навлезли в страната на ирокезите, в мен се бяха породили известни съмнения и страхове относно кампанията. Но си позволих само да кажа, „Със сигурност, сър, ако стигнете успешно до Дюкен, с тази хубава войска, така добре снабдена с артилерия, това укрепление, което все още не е напълно укрепено и, както се чува, няма силен гарнизон, едва ли ще може да се съпротивлява дълго. Единствената опасност, за която се сещам, която би могла да попречи на напредъка ви, са индиански засади. Поради постоянната практика, индианците са много умели в правенето и изпълняването им; а проточената ви линия, дълга почти четири мили, може да направи армията уязвима откъм изненадващи странични нападения, които да я разкъсат като връв на няколко парчета, които поради разстоянието не могат да се придвижат навреме, за да си помогнат едно на друго.“


Той се усмихна на невежеството ми и отвърна, „Тези диваци може действително да са страшен враг за вашата недодялана американска нередовна войска, но е невъзможно, сър, те да направят каквото и да е впечатление на кралската редовна и дисциплинирана войска.“ Съзнавах, че има нещо неприлично в това да споря с военен по теми свързани с неговата професия, и не казах нищо повече. Врагът, обаче, противно на моите опасения, не се възползва от уязвимото разположение на армията по време на прехода, а я остави безпрепятствено да се приближи на девет мили от мястото; и там, в момент в който войската била в по-прибрана формация, защото тъкмо била преминала река и предните части спрели, за да изчакат всички да преминат, и в част от гората по-открита от всички вече преминати, атакувал авангарда със силен огън от иззад дървета и храсти, което бил първият знак, по който генералът разбрал, че има враг наблизо. Понеже авангардът не бил в строй, генералът побързал да го подкрепи с войска, което било извършено в голям безпорядък, през коли, багажи и добитък; а вражеският огън започнал и отстрани; офицерите, понеже били на коне, били лесно разпознаваеми и избрани за мишени и много бързо паднали; а войниците били скупчени наедно, нямали и не чували никакви заповеди, и стояли докато две трети от тях били избити; и тогава, обхванати от паника, всички побегнали през глава.

Коларите всички взели по един кон от впряговете си и търтили на бяг. Техният пример бързо бил възприет от други; така че всички коли, провизии, артилерия и запаси останали във врага. Генералът, ранен, бил изнесен с премеждия; секретарят му, г-н Шърли, бил убит близо до него; и от осемдесет и шест офицери, шестдесет и трима били или убити, или ранени, а седемстотин и четиринадесет мъже от хиляда и сто били убити. Тея хиляда и сто били избрани от цялата армия; останалите били назад с полковник Дънбар, който трябвало да последва с по-тежката част от запасите, продоволствията и товарите. Бегълците, без някой да ги преследва, пристигнали в лагера на Дънбар и паниката им незабавно обхванала него и всичките му хора; и въпреки че сега имал над хиляда мъже, а врагът, който победил Брадък не надхвърлял четиристотин индианци и французи общо, вместо да настъпи и да се опита да спаси част от загубената чест, той наредил всичките запаси, амуниция и т.н. да бъдат разрушени, за да може да има повече коне в бягството си към населените части, и по-малко коли за влачене. Там губернаторите на Вирджиния, Мериленд и Пенсилвания поискали от него да остави войска по границата, за да защити жителите, но той продължил с бързия си ход през цялата страна и не се усетил в безопасност преди да достигне Филаделфия, където можеше да разчита на защитата на жителите.

Цялата тази работа накара нас американците за първи път да започнем да подозираме, че възвишените ни идеи за храбростта на редовната британска войска може би не отговарят на истината.

Също по време на първия си преход, от мястото където акостираха до отвъд колониите, те бяха плячкосали и грабили населението, напълно разорили някои бедни семейства, освен че обиждаха, упражняваха насилие над хората и ги арестуваха, когато протестираха. Това беше достатъчно, за да разбие илюзиите ни относно защитници като тези, ако въобще някога сме се нуждаели от защита. Колко различно беше поведението на френските ни приятели през 1781 г., когато по време на преход през най-населената част от страната ни – от Роуд Айлънд до Вирджиния, близо седемстотин мили – не дадоха повод дори за най-малкото оплакване за изгубено прасе, кокошка, или дори ябълка. 

Капитан Орми, който беше един от адютантите на генерала, и който беше сериозно ранен и по тази причина беше отнесен заедно с него и продължи с него до смъртта му, която дойде след няколко дни, ми каза, че през целия първи ден бил безмълвен, а вечерта казал само, “Кой би си помислил?” Че през целия втори ден отново мълчал, а накрая само казал, “Следващия път по-добре ще знаем как да се справим с тях;” и след няколко минути умрял.

Книжата на секретаря, заедно с всичките заповеди, инструкции и кореспонденция на генерала, попаднаха във вражески ръце, и някои от тях бяха избрани, преведени на френски и публикувани, за да докажат враждебните намерения на британския двор преди обявяването на война. Сред тях видях някои писма на генерала до министерството, които хвалеха големите ми заслуги към армията и ме препоръчваха на тяхното внимание. Дейвид Хюм, който няколко години по-късно беше секретар на Лорд Хертфорд, когато последният беше дипломат във Франция, и по-късно на ген. Конуей, когато последният беше външен министър, ми каза, че сред книжата в този офис намерил писма от Брадък\footnote{Braddock}, които силно ме препоръчвали. Но тъй като експедицията беше неуспешна, изглежда че заслугите ми не са били сметнати за много полезни, защото тези препоръки никога не ми принесоха каквато и да е полза. 

Що се отнася до услуги получени лично от него, поисках само една, а именно, да заповяда на офицерите си да не набират повече войници сред нашите платени слуги, и да уволни тези, които вече бяха постъпили във войската. Това искане той удовлетвори с готовност и неколцина бяха върнати на господарите им след моята молба. Когато ръководството на армията се падна на Дънбар, той не бе така щедър. След отстъплението му, или по-точно бягството му, към Филаделфия, аз го помолих да освободи слугите на трима бедни фермери от окръг Ланкастър, които беше набрал за войската, като му напомних за нарежданията, които покойният генерал беше издал по този въпрос. Той ми обеща, че ако господарите отидат да се срещнат с него в Трентън, където щеше да бъде след няколко дни на път за Ню Йорк, ще им предаде там слугите им. Съответно те си направиха труда и разхода да отидат до Трентън, където той отказал да изпълни обещанието си, за тяхна голяма загуба и огорчение. 

Щом се разбра за загубата на конете и колите, всички собственици дойдоха при мен за компенсацията, която се бях обвързал да платя. Исковете им ме притесниха много, след като им казах, че парите са налице у ковчежника, но че първо трябва да се получи заповед от ген. Шърли за изплащането им, и ги уверих, че вече съм писал на генерала за това; и че тъй като той беше на известно разстояние не можехме да очакваме бърз отговор, и че трябва да са търпеливи; всичко това не беше достатъчно удовлетворително, и някои започнаха да ме съдят. В крайна сметка ген. Шърли ме избави от това ужасно положение като назначи комисия да разгледа исковете и нареди плащанията. Те възлязоха на почти двадесет хиляди паунда, което, ако трябваше да платя, щеше да ме разори. 

Преди да научим за този разгром двамата доктори Бонд дойдоха при мен и ми предложиха да се включа в подписка за набиране на средства за тържествена заря по случай превземането на форт Дюкен\footnote{fort Duquesne (на английски явно се произнася Дукейн, виж уики)}. Аз погледнах сериозно и казах, че според мен ще има достатъчно време да се подготвим за празненствата, след като научим, че имаме повод да празнуваме. Те изглежда се изненадаха, че не откликнах веднага на предложението им. “Защо по д------е!”, казва единият от тях, “нима предполагаш, че няма да превземем форта?” “Не съм сигурен, че няма да бъде превзет, но съм сигурен, че военните дела са предмет на голяма несигурност\footnote{subject to great uncertainty}.” Изложих пред тях причините за съмненията ми; подписката беше изоставена, а инициаторите си спестиха унижението, което щяха да изпитат, ако бяха подготвили зарята. По някакъв друг случай д-р Бонд казал, че не харесвал Франклиновите лоши предзнаменования\footnote{forebodings}. 

Губернатор Морис, който преди загубата на Брадък не беше спрял да тормози Събранието със съобщения целящи да го принудят да приеме закони за събиране на данъци за отбрана на провинцията, които изключват от данъчно облагане имотите на собствениците на колониите наред с някои други, сега поднови набезите си с още по-големи усилия и с повече надежда за успех, тъй като опасността и нуждата бяха нарастнали. Събранието обаче, вярвайки, че правдата е на тяхна страна, и вярвайки, че ще се откажат от основно право, ако позволят на губернатора да се меша в техните финансови проектозакони, не смекчи позицията си. В един от последните – относно отпускането на петдесет хиляди паунда – той беше предложил промяна от една дума. Проектозаконът гласеше “че всички имоти ще бъдат обложени, включително тези на собствениците на колонията.” Неговата промяна беше, вместо “включително” да се чете “с изключение на”: малка, но доста съществена измяна. Въпреки това, щом новините за провала достигнаха до Англия, нашите приятели там, на които ние се бяхме погрижили да изпратим всичките отговори, с които събранието беше отвърнало на съобщенията на губернатора, вдигнаха шум срещу собствениците поради лошотията и несправедливостта на заповедите, които даваха на губернатора си; като някои стигната дотам да кажат, че като възпрепятстват отбраната на провинцията губят правата си върху нея. Това ги притесни и наредиха на главния си отговорник по приходите\footnote{receiver-general} да добави пет хиляди паунда от техните пари към каквато сума отдели Събранието за тази цел.

Това беше съобщено на Камарата, беше прието наместо техния дял от общия данък, и беше подготвен и приет нов закон, който ги изключваше. С този закон бях назначен за член на комисията, която щеше да разпредели парите – шестдесет хиляди паунда. Бях участвал дейно в подготовката на закона и в набирането на подкрепа за приемането му, а в същото време подготвих и проектозакон за установяването и дисциплинирането на доброволна войска, който прекарах през Камарата без много трудности, тъй като в него се бях погрижил да оставя Квакерите да избират свободно. За да увелича подкрепата за идеята за сдружението нужно за създаването на войска написах един диалог\footnote{Този диалог и законът за войската са в “Джентълмен'с мегъзин” от февруари и март 1756. – бележка в полето} повдигащ и оборващ всички възражения срещу създаванто на една такава войска, които можах да измисля, който беше отпечатан, и който стори ми се имаше голям ефект. 

Докато различните роти в града и окръга се формираха и научаваха как да се упражняват, губернаторът ме убеди да поема контрол над северозападния ни пограничен район, в който врагът беше проникнал, и да организирам защитата на жителите като набера войска и издигна отбранителна линия от укрепления\footnote{build a line of forts}. Заех се с тази военна работа, въпреки че не смятах, че разполагам с нужните умения. Даде ми генерално пълномощно и пакет с непопълнени пълномощни за офицери, които да давам на когото сметна за добре. Не срещнах трудности при набора на мъже, като скоро имах петстотин и шестдесет под командата си. Синът ми, който беше служил като офицер в армията в предишната война с Канада, ми беше адютант и ми помагаше много. Индианците бяха изгорили Гнейдънхът\footnote{Gnadenhut}, село заселено от моравците, и бяха избили жителите; но се смяташе, че селото е удобна позиция за едно от укрепленията. 

За да стигнем до него, събрах ротите във Витлеем\footnote{Bethlehem}, главното селище на тези хора. За моя изненада, мястото беше добре подготвено за отбрана; разоряването на Гнейдънхът ги беше накарало да усетят опасността. Главните сгради бяха защитени с дървено укрепление\footnote{stockade}; бяха закупили някои оръжия от Ню Йорк и дори бяха поставили известно количество малки павета между прозорците на високите си каменни къщи, за да ги хвърлят жените им по главите на индианците, в случай, че се опитат да проникнат в тях със сила. Въоръжените братя стояха на стража и се сменяха редовно като във всеки град с гарнизон. В разговор с епископа, Шпангенберг, споменах че това ме е изненадало; защото знаейки, че бяха получили позволение от Парламента да бъдат освободени от военна повинност в колониите, бях предположил, че носенето на оръжие противоречи на убежденията им. Той ми каза, че неносенето на оръжие не е един от установените им принципи, но че по времето, когато издействали закона за освобождаване от военни задължения, се смятало, че много от хората споделят този принцип. При този случай обаче, за тяхна изненада, установили, че малко от хората се придъжали към него. Изглежда че или са били в заблуждение относно собствените си убеждения, или са заблудили Парламента; но здравият разум, подпомогнат от непосредствена опасност, понякога е твърде силен за капризни мнения.

В началото на януари се захванахме с изграждането на укрепления. Изпратих един отряд към Минисинк\footnote{Minisink}, със заповеди да изградят едно за защита на горната част на окръга, и още един с подобни заповеди за долната част; а сам реших да се насоча с останалите мъже към Гнейдънхът, където се смяташе, че има по-непосредствена нужда от укрепление. Моравците ми осигуриха пет коли за инструментите, продоволсвията, багажите ни и т.н. 

Точно преди да тръгнем от Витлеем при нас дойдоха единадесет фермера, които индианците бяха изгонили от плантациите им, и ни помолиха да ги снабдим с оръжие, за да могат да се върнат и да си вземат добитъка. Дадох на всеки от тях пушка с подходящи муниции. Малко след като започнахме похода заваля и продължи да вали цял ден; по пътя нямаше къща, в която да се подслоним, до вечерта, когато стигнахме къщата на един германец. В нея и в плевнята му се сгушихме мокри до кости. Добре беше, че не ни нападнаха по време на похода ни, защото оръжията ни бяха от най-простите и мъжете не можеха да държат затворите\footnote{gun locks} си сухи. Индианците умеят да правят приспособления за тази цел, каквито ние нямахме. През деня срещнали единадесетте фермера споменати по-горе и убили дест от тях. Единият, който се спасил, ни разказа, че пушката му и тези на неговите спътници не могли да произведат изстрел, защото спусъчните механизми\footnote{the priming} били мокри от дъжда.

На другия ден времето беше хубаво и не\footnote{ние?} продължихме с прехода си и пристигнахме в изоставеното Гнейдънхът. Наблизо имаше дъскорезница, около бяха оставени няколко купчини с дъски, с които скоро си направихме заслони; действие, което беше крайно нужно в този суров сезон, тъй като нямахме палатки. Първата ни работа беше да погребем по-успешно мъртъвците, които заварихме там, които селяните от околността бяха полу-погребали. 

На следващата сутрин направихме план на укреплението си и отбелязахме основите. Обиколката му беше четиристотин петдесет и пет стъпки, за които щяхме да имаме нужда от също толкова на брой заострени в единия край греди\footnote{palisades} с диаметър една стъпка. Незабавно вкарахме в употреба седемдесетте си брадви и понеже мъжете бяха умели в сеченето на дървета, напредвахме бързо. Като видях, че дърветата падат така бързо, от любопитство погледнах часовника си в момента, в който двама мъже започваха да секат един бор; в рамките на шест минути го бяха повалили, а като измерих диаметъра му, установих, че е четиринадесет инча. Всеки бор беше достатъчен за три кола от по осемнадесет фута, заострени в единия край. Докато се правеха те, останалите мъже изкопаха ров околовръст, дълбок три стъпки, в който трябваше да бъдат поставени коловете; а като разделихме предните от задните колела на колите се сдобихме с десет колички с по два коня всяка, с които да пренесем трупите от гората до мястото. След като ги поставихме, дърводелците направиха по цялото продължение на вътрешната страна платформа, на която да стоят мъжете, когато стрелят през процепите. Имахме един swivel gun\footnote{swivel gun}, който поставихме в един от ъглите; щом го поставихме веднага произведохме няколко изстрела с него, за да уведомим индианците, в случай, че такива се намираха на разстояние от което да ни чуят,  че имаме такива; и така нашето укрепление, ако такова превъзходно име може да бъде дадено на бедното ни заграждение\footnote{miserable stockade}, беше завършено в рамките на една седмица, въпреки че през ден валеше токова силно, че мъжете не можеха да работят.

Това ми позволи да забележа, че когато мъжете са заети, са най-доволни; защото в дните, в които можаха да работят бяха добродушни и ведри\footnote{good-natur'd and cheerful}, и прекарваха вечерта весело като имаха съзнанието, че са свършили добра работа през деня; но в дните, в които бяха стояли без работа, бяха свадливи и се бунтуваха, оплакваха се от свинското, от хляба и т.н., и постоянно бяха вкиснати, което ми напомни за един морски капитан, който имаше това правило винаги да държи мъжете заети с работа; и когато веднъж помощникът му\footnote{his mate} му казал, че са свършили всичко и няма какво повече да им възложи, той казал “О, накарай ги да лъснат котвата.”  
Този вид укрепление, колкото и да е примитивен, е достатъчна защита срещу индианците, които нямат оръдия. Като се видяхме установени на сигурно и с място, където да отстъпим при нужда, излязохме да обгледаме околностите. Не срещнахме индианци, но видяхме местата по съседните хълмове, където се бяха крили, за да наблюдават действията ни. Заслужава да се спомене начинът, по който устройваха наблюдателниците си. Понеже беше зима, имаха нужда от огън в тях; но светлината на обикновен огън на повърхността би ги издала от разстояние. Затова те бяха изкопали дупки в земята, три фута широки и малко по-дълбоки; видяхме, че със сатърите си бяха изрязали въглени от обгорени главни пръснати из гората. С тези въглени бяха наклали малки огньове в дъното на дупките и забелязахме по следите от телата им в бурените и тревата, че бяха лежали около дупките с краката в тях, за да са на топло, което при тях е особено важно. Този вид огън не можеше да ги разкрие нито със светлината си, нито с пламъка си, искрите си, нито дори с пушека си: изглежда, че не бяха твърде многобройни, и че бяха видяли, че сме твърде много, за да ни нападнат с надежда за успех. 

За свещеник имахме един ревностен презвитериански духовник, г-н Бийти\footnote{Beatty}, който ми се оплака, че като цяло мъжете не присъствали на молитвите му и не обръщали внимание на насърченията му\footnote{exhortations}. Когато се записваха им беше обещан – освен заплата и храна – дажба\footnote{a gill} ром на ден, който се раздаваше редовно, половината сутринта, другата половина вечерта; и забелязах, че те също толкова редовно отиваха да го получат в точно уречения час; затова казах на г-н Бийти “Може би е под достойнството на професията ти да сервираш ром, но ако ти го раздаваш, и то само след молитва, ще имаш всички мъже на разположение.” Той хареса идеята, зае се с работата и – с помощта на няколко души, които раздаваха алкохола – се справяше с нея задоволително, а молитви никога не са се радвали на по-широк интерес или били по-редовно посещавани; и си помислих, че този подход е за предпочитане пред наказанията, които някои военни закони предвиждат за неприсъствие на религиозна служба.

Едва бях приключил тази работа и бях снабдил укреплението с провизии, когато получих писмо от губернатора, с което той ми съобщаваше, че е свикал събранието, и че желае моето присъствие там, ако състоянието на работите по границата е такова, че не е нужно да оставам повече. Понеже приятелите ми от събранието също ме притискаха в писмата си, ако е възможно, да присъствам на срещата, и понеже и трите ми планирани укрепления бяха завършени, а жителите бяха съгласни да останат по земята си при такава защита, реших да се върна; още с по-голямо желание, тъй като един офицер от Нова Англия, полковник Клейфъм\footnote{colonel Clapham}, човек с опит във индианските войни, който беше по това време на посещение в нашето укрепление, се съгласи да поеме командването. Дадох му пълномощно, и след като строих гарнизона, го прочетох пред тях и го представих като човек, който поради опита си във военните работи е много по-добре подготвен за техен командир от мен самия; и като ги насърчих малко\footnote{little exhortation} се сбогувах с тях. Бях придружен до Витлеем, където останах няколко дни, за да се възстановя от умората. Първата нощ в добро легло едва успях да мигна, понеже беше толкова различно от колибата в Гнейдън, където спях увит в едно-две одеала на твърдия под.

Докато бях във Витлеем разпитах относно обичаите на моравците: някои от тях ме бяха придружили и всичките бяха много мили с мен. Научих, че всички споделят плодовете от труда си\footnote{worked for a common stock}, че се хранят на обща маса и спят в общи спални на големи групи. В спалните забелязах процепи под тавана, по цялото протежение, на определено отстояние един от друг, за които сметнах, че разумно са оставени, за да може да влиза свеж въздух. Бях в църквата им, където ме забавляваха с добра музика: органът беше акомпаниран от цигулки, обои, флейти, кларинети и т.н. Научих, че проповедите им обикновено не са насочени към смесена публика от мъже, жени и деца, както е обичайно при нас, а че понякога се събират женените мъже, понякога жените им, след това младите мъже, младите жени и децата, всяка група поотделно. Проповедта, която чух, беше насочена към последните, които влязоха и бяха настанени да седнат в редици на пейки; момчетата под ръководството на един млад мъж, техният учител, а момичетата водени от млада жена. Посланието\footnote{the discourse} ми се стори добре пригодено за техните способности и беше предадено с приятен, приятелски тон, като ги приканваше\footnote{coaxing them}, така да се каже, да бъдат добри. Те се държаха много послушно, но излгеждаха бледи и не много здрави, което ме накара да подозирам, че ги държат твърде много на закрито или не им дават да се упражняват достатъчно.

Разпитах дали са верни приказките, че моравските бракове са по жребий. Казано ми бе, че жребий се използва само при определени случаи; че обикновено, когато един млад мъж усети, че иска да се ожени, съобщава на старейшините от неговия клас\footnote{class}, които се съветват с възрастните жени, които ръководят младите жени. Тъй като старейшините от двата пола познават добре характерите и наклонностите на съответните си ученици, те най-добре можели да преценят кои двойки са подходящи и като цяло решенията им били приемани; но ако така се случело, че за един млад мъж се намерели две или три еднакво подходящи млади жени, се прибягвало до жребий. Възразих, че ако изборът не е по взаимно съгласие на двамата, някои от тях може да се окажат много нещастни. “Така може да стане”, каза ми информаторът ми, “и когато ги оставим сами да избират;” което аз наистина не можех да отрека. 

Щом се завърнах във Филаделфия, установих, че военните работи вървят като по вода, като тези от жителите, които не бяха квакери, като цяло се бяха включили, организирали в компании\footnote{companies}, бяха си избрали капитани, лейтенанти и\footnote{ensigns}, според новия закон. Д-р Б. ме посети и ми разказа за усилията, които бе положил да разпространи сред хората добро мнение за закона, и приписа големи залсуги на тези опити. Аз тщеславно бях приписал всичко на моя Диалог; но понеже не знаех дали не е прав, го оставих да се радва на мнението си, което смятам е най-добрият подход в такива случаи, като цяло. По време на събранието си офицерите ме избраха да съм ръководител на полка, което приех този път. Не си спомням колко компании\footnote{companies} имахме, но строявахме около хиляда и двеста снажни\footnote{well-looking} мъже, артилерийска рота\footnote{company}, които бяха снабдени с шест бронзови топа\footnote{field-pieces}, в чиято употреба толкова се бяха усъвършенствали, че можеха да стрелят по дванадесет пъти в минута. Първия път когато прегледах моята част, те ме придружиха до къщата ми и пожелаха да ме поздравят с няколко залпа изстреляни пред вратата, които събориха и счупиха няколко от стъклата на електрическия ми прибор. А новите ми почести се оказаха не по-малко крехки; тъй като цялата ни организация скоро след това беше разтурена с отмяна на закона в Англия. 

Веднъж по време на краткия период, в който бях полков ръковдител, докато се подготвях за пътуване до Вирджиния, на офицерите в моя полк им хрумна, че би било добре да ме придружат на излизане от града до долния ферибот\footnote{(as far as the Lower Ferry)}. Точно когато възсядах коня се появиха пред вратата ми, между тридесет и четиридесет човека, на кон, и всичките с униформите си. Не бях предупреден за проекта им, иначе щях да го възпрепятствам, понеже по природа изпитвам отвращение към официални церемонии независимо от случая\footnote{averse to the assuming of state on any occasion}; и бях доста раздразнен от появата им, тъй като не можех избегна компанията им. Нещо, което още повече влоши положението беше, че щом започнахме да се движим, те изкараха сабите си и яздиха така с тях по целия път. Някой писал за случката на собственика на колонията и това го обидило много. Такава чест не му била оказвана, когато идвал в провинцията, нито на някои от губернаторите му; и каза, че подобавала само на принцове с кралска кръв, което, ако ме питат, може и да е верно, тъй като съм бил, и все още съм, неук по въпросите на етикета в такива случаи. 

Тази глупава случка обаче много засили неговата неприязън срещу мен, която и преди това не беше малко, поради поведението ми в Събранието във връзка с освобождаването на на имотите му от облагане с данък, на което аз винаги горещо се възпротивявах, и то не без остри забележки относно колко подло и несправедливо\footnote{meanness and injustice} е да настоява за това. Той ме обвини пред външното министерство\footnote{the ministry}, че съм голямо препятствие пред целите на краля\footnote{the king's service}, понеже с влиянието си в Камарата възпрепятствам приемането на правилните закони за събиране на пари, и представи случката с този парад като доказателство за това, че искам да изтръгна властта над провинцията със сила от ръцете му. Също така поиска от Сър Евърърд Фоукнер\footnote{Everard Fawkener}, министъра на пощите,\footnote{the postmaster-general}, да ме лиши от службата ми; но това не произведе друг ефект освен да предизвика нежното порицание на Сър Евърърд. 

Въпреки непрестанната караница между губернатора и Камарата, в която аз, като член, имах значително участие, между мен и този джентълмен продължи културен диалог, и никога нямаше между нас лични дрязги\footnote{personal difference}. Понякога си мисля, че отговорите на съобщенията му, за които се знаеше че аз ги пишех, не пораждаха у него никаква или малко обида заради професионален навик, и че понеже беше учил за адвокат, може би гледаше на нас двамата просто като на адвокати представящи двама спорещи клиенти в съдебен процес, той – собствениците, а аз – Събрението. По тази причина понякога ме посещаваше\footnote{call} приятелски, за да ме посъветва по някои трудни въпроси / тънки моменти\footnote{difficult points}, а понякога, макар и не често, се вслушваше в съветите ми.

Действахме заедно когато трябваше да снабдим армията на Брадък\footnote{Braddock} с продоволствия; а когато достигнаха до нас стряскащите новини за разгрома му, губернаторът бързо изпрати да ме повикат, за да се посъветва с мен относно мерките срещу обезлюдяването на пограничните райони\footnote{desertion of the back counties}. Вече не си спомням какво го посъветвах; но мисля, че беше, да бъде изпратено писмо до Дънбар, и да бъде убеден, ако е възможно, да разположи войските си по границата за тяхна защита, докато не стане възможно да продължи експедицията с помощта на подкрепления от колониите. А след връщането ми от границата, искаше да се заема с ръководството на една такава експедиция с провинциална войска за разрушаването на Форт Дюкен\footnote{Duquesne}, докато Дънбар и мъжете му бяха заети с друго; и предложи да ме назначи за генерал. Аз нямах такова добро мнение за собствените си способности във военните дела, каквото той заявяваше, че има, а вярвам, че заявленията му отиваха отвъд истинските му чувства; но може би мислеше, че популярността ми ще улесни събирането на мъже, а влиянието ми в Събранието – отпускането на пари за да им се заплати, и то без да се облагат с данък имотите на собствениците на колониите. Като видя, че не съм толкова въодушевен да се заема с това, колкото очакваше, остави проекта и скоро след това напусна управлението, като беше наследен от кап. Дени\footnote{Captain Denny}.

Преди да продължа с разказа за участието ми в обществените работи при управлението на администрацията на новия губернатор, може би не е лошо да споделя тук за израстването и развитието на философската\footnote{philosophical} ми репутация.

През 1746 се срещнах в Бостън с един д-р Спенс, който скоро беше пристигнал от Шотландия и ми показа някои експерименти с електричество. Те не бяха изпълнени по най-добрия възможен начин, тъй като той не беше много умел; но понеже имаха за предмет нещо доста ново за мен, ме изненадаха и ми доставиха удоволствие. Скоро след като се върнах във Филаделфия, нашата библиотека получи като подарък от г-н П. Колинсън\footnote{P. Collinson}, сътрудник на Лондонското кралското дружество, стъклена тръба и кратко описание как може да се употреби за такива експерименти. С голямо желание се захванах за възможността да повторя нещата, които бях видял в Бостън; и след много опити успях да се науча да изпълнявам и тези, за които имахме описание от Англия, като добавих и някои нови. Казвам много опити, защото известно време къщата ми беше непрестанно пълна с хора, които идваха да видят тези нови чудеса. 

За да споделя донякъде този товар с приятелите ми, поръчах да бъдат издухани още няколко подобни стъклени тръби, така че след време имахме няколко изпълнителя. Сред тези главният беше г-н Кинърсли\footnote{Kinnersley}, изобретателен съсед, когото насърчих да показва експериментите срещу пари, понеже беше загубил бизнеса си\footnote{останал без работа/being out of business}, и за когото приготвих две лекции, в които експериментите бяха така подредени, и придружение с такива обяснениея, че по-първите да улеснят разбирането на последващите. Той се снабди с елегантен апарат за целта, в който всички малки инструменти, които аз бях направил грубо за себе си, бяха добре изработени от майстори \footnote{instrument-makers}. Лекциите му бяха посещавани от много хора и доставяха голямо удоволствие; и след време той тръгна из колониите, показваше ги във всяка столица и си докара малко пари. Действително на Западните индийски острови\footnote{West India islands} беше трудно да се представят експериментите, поради обикновено високата влажност на въздуха.\footnote{from the general moisture of the air.}

Тъй като бяхме задължени на г-н Колинсън за подарената тръба и т.н., сметнах за редно да го уведомя за успехите ни с нея, и му писах няколко писма с описание на експериментите ни. Той ги прочете пред Кралското дружество, където първоначално не били сметнати за достатъчно интересни, че да бъдат отпечатани в тяхното списание\footnote{Transactions}. Изпратих на д-р Мичъл\footnote{Mitchel}, мой познат, една от статиите, които бях написал за г-н Кинърсли, относно еднаквостта\footnote{the sameness} на светкавиците и електричеството, и той ми каза, че са били прочетени, но осмени, от познавачите. След като обаче показах статиите на д-р Фодъргил\footnote{Fothergill}, той прецени, че са твърде ценни, за да останат непубликувани и ги даде на Кейв\footnote{Cave} да ги публикува в неговото Джентълменс мегъзин\footnote{Gentleman's Magazine'} но той реши да ги отпечата отделно като памфлет, а д-р Фодъргил написа предговора\footnote{preface}. Кейв изглежда добре си беше направил сметката, защото с допълненията цялото набъбна до книжка кварто формат\footnote{swell'd to a quarto volume}, която беше преиздадена пет пъти, а не му струва нищо откъм авторски права\footnote{copy-money}.

Мина известно време, обаче, преди тези статий да бъдат забелязани в Англия. Така се случи, че след като едно копие попадна в ръцете на the\footnote{Count de Buffon}, философ със заслужено голяма репутация във Франция и в действителност из цяла Европа, той убеди г-н Далибар\footnote{Dalibard} да ги преведе на френски и те бяха публикувани в Париж. Публикуването обиди абатът Ноле\footnote{Nollet}, preceptor\footnote{preceptor} по приордна философия на кралското семейство, и умел експериментатор, който беше произвел и публикувал теория на електричеството, която беше добила широка популярност по това време. В началото не успял да повярва, че такава творба идва от Америка, и казал, че трябва да е била изфабрикувана от враговете му в Париж, за да подринат системата му to\footnote{decry his system}. По-късно, след като го уверили, че наистина съществува този човек Франклин във Филаделфия, в което той се съмнявал, той публикува сборник с писма, повечето адресирани до мен, в които защитаваше теорията си и отричаше достоверността\footnote{denying the verity} на експериментите ми, както и на заключенията изведени от тях\footnote{positions deduced from them}. 

Известно време имах намерение да отвърна на абата и всъщност започнах отговора; но след като съобразих, че писанията ми съдържаха описания на експерименти, които всеки може да потвърди, а ако не могат да бъдат потвърдени, не биха могли да бъдат защитени; или наблюдения поставени като предположения, а не представени догматично, които съответно не изискваха от мен да ги защитавам; и след като прецених, че един спор между двама души пишещи на различни езици би могъл да бъде удължен значително от грешки в превода и съответните неразбирателства – много от писмата на абата бидейки основани на грешка в превода – реших да оставя статиите ми да говорят сами за себе си, смятайки, че е по-добре да прекарам времето, което можех да заделя от обществените работи, в правене на нови експерименти, вместо в спорове относно тези, които вече бяха направени. Затова никога не отвърнах на г-н Ноле\footnote{Nollet}, а последвалите събития не ми дадоха основание да съжалявам; защото приятелят ми г-н льо Роа\footnote{le Roy}, от Кралската академия на науките се захвана с каузата ми и го обори; книгата ми беше преведена на италиански, немски и латински; а идеите й постепенно бяха възприети от всички философи в Европа, предпочетени пред тези на абата; така че той доживя да се види единственият от сектата си, с изключение на г-н Б----- от Париж, който му беше непосредствен ученик\footnote{disciple}. 

Това, което придаде на книгата ми по-внезапна и обща известност бе успехът на един от предложените експерименти, относно привличането на светкавици от облаците, извършен от господата Далибард и дьо Лор\footnote{Dalibard and De Lor} в Марли\footnote{Marly}. Това привлече общественото внимание навсякъде. Г-н дьо Лор, който имаше инструменти за експериментална философия и изнасяше лекции в тази област на науката се зае с пресъздаването на това, което той наричаше Филаделфийските експерименти; и след като ги беше изпълнил пред краля и двора, всички любопитни парижани се стекли при него, за да ги видят. Няма да удължавам този разказ с описание на този главен\footnote{capital} експеримент, нито на безкрайното удоволствие, което изпитах при успеха на един друг, който направих скоро след това с хвърчило във Филаделфия, тъй като и двата могат да бъдат намерени в историята на електричеството\footnote{histories of electricity}. 

Д-р Райт, един английски лекар, писал от Париж на един свой приятел от Кралското дружество\footnote{Royal Society}, и му разказал за високото мнение, което учените в чужбина имат за моите експерименти, както и учудването им, че писанията ми са получили толкова малко внимание в Англия. При това Обществото отново се захванало с разглеждане на писмата, които им бяха прочетени; а известният д-р Уотсън направи обобщение на съдържанието\footnote{a summary account of them} им и на останалите, които след това бях изпратил в Англия по темата, което придружи с известни похвали за писателя. След това обобщението беше отпечатано в списанието на Кралското дружество\footnote{their Transactions}. Също така ми присъдиха\footnote{presented me} златния медал на сър Годфри Копли\footnote{Godfrey Copley} за 1753 година, като награждаването\footnote{the delivery} било придружено от много хубава реч на президента лорд Макълсфийлд\footnote{Macclesfield}, в която ми била оказана голяма почит. 

Новият ни губернатор, капитан Дени, ми донесе гореспоменатия медал от Кралското дружество и ми го връчи по време на едно празненство, което градът организира в негова чест. Наред с него прибави и много учтиви думи относно неговото високо мнение за мен, като каза, че отдавна бил запознат с характера ми\footnote{my character}. След вечеря, когато компанията започна да пие, както беше обичайно по това време, той ме дръпна настрана в съседна стая и ми откри, че бил посъветван от приятелите си в Англия да култивира приятелство с мен, като с човек, който може да му даде най-добрите съвети и най-ефективно да допринесе за улесняване на управлението му\footnote{his administration}; че по тази причина най-много от всичко иска да се разбира добре с мен, и ме помоли да бъда сигурен\footnote{begged me to be assured}, че при всички случаи е готов да ми услужи по всеки начин, който е във властта му. Също ми разказа надълго за добрите чувства на собственика към провинциата, и как щяло да бъде от голяма полза за всички нас – и в частност за мен – ако дълго продължилата съпротива срещу мерките му, бъде прекратена, а хармонията между него и хората бъде възстановена; за постигането на което се смятало, че никой не можел да допринесе повече от мен; и че съм можел да разчитам на подобаващи признание и възнаграждения\footnote{recompenses} и т.н., и т.н. Пиещите, като видяха, че не се върнахме веднага при тях на масата, ни изпратиха един декантер мадейра, от който губернаторът се възползва свободно\footnote{без задръжки made liberal use of}, и пропорционално на това нарастваха неговите молби\footnote{solicitations} и обещания.

Отвърнах му в този дух: че благодарение на Бога положението ми е такова, че не са ми нужни подаръците\footnote{favours} на собственика; и че като член на Събранието не бих могъл да ги приема; че въпреки това не тая лична неприязън към собственика, и че винаги когато изглежда, че обществените мерки, които предлага, целят доброто на хората, никой няма да ги прегърне и подкрепи по ревностно от мен; като казах, че поранешната  ми съпротива беше основана на това, че мерките, които той искаше да прокара, очевидно целяха да обслужат неговия интерес в голям ущърб на този на народа; че съм му много задължен (на губернатора) за изразеното добро мнение за мен\footnote{professions of regard to me}, и зе може да разчит ана мен, че ще направя всичко, което е във властта ми, за да улесня управлението му\footnote{his administration} доколкото е възможно, като същевременно се надявам, че не е дошъл със същите печални\footnote{unfortunate} заръки, които спъваха предшественика му.

По този въпрос той не каза нищо тогава; но когато по-късно започна работа със Събранието, те отново се появиха и споровете бяха възобновени, а аз бях деен в отпора както винаги, като първо бях основен автор\footnote{penman} на молбата да разкрие инструкциите си на събранието\footnote{to have a communication of the instructions}, а след това и на коментари\footnote{remarks} по тях, които могат да бъдат намерени в стенограмите от онова време, както и в Историческия преглед, който публикувах по-късно. Но между нас не се породи лична вражда; често бяхме заедно; той беше начетен, беше видял много от света, и беше много забавен и приятен събеседник. Донесе ми първите сведения, че старият ми приятел Джеймс Ралф\footnote{Jas.?? Ralph} е още жив; че е смятан за един от най-добрите писатели на политически теми в Англия; че е бил зает\footnote{employed} в спора между принц Фредерик и краля и е спечелил пенсия от триста на година; че репутацията му като поет не е голяма, след като Поуп проклел поезията му в Дънсиад\footnote{the Dunciad}; но прозата му беше смятана за добра\footnote{as good as any man's}. 


[15]\footnote{[15] Многото единодушни решения на Събранието -- коя дата?--[Бележка в полето]} След като в крайна сметка събранието видя, че собствениците продължиха да спъват заместниците си с инструкции, които не съответстваха нито на интересите на хората, нито бяха в служба на короната, то реши да отправи петиция към краля срещу тях, и ме назначи да отида в Англия като техен представител и да представя и подкрепя петицията. Камарата беше изпратила един проектозакон на губрнатора, с който отпускаше шестдесет хиляди паунда за нуждите на краля (десет хиляди от които да се оставят на разположение на тогава генерала, лорд Лоудаун\footnote{Loudoun}), който губернаторът твърдо отказа да приеме, съгласно инструкциите си. 

Бях уговорил преминаването си с капитан Морис, капитан на the paquet\footnote{the paquet} в Ню-Йорк, и запасите ми вече бяха качени на борда, когато лорд Лоудаун\footnote{Loundoun} пристигна във Филаделфия, спешно, както ми каза, за да се опита да постигне споразумение между губернатора и Събранието, за да не бъдат възпрепятствания делата на краля поради техините разногласия. Съответно поиска губернаторът и аз да се срещнем с него, за да чуе какво имат да кажат двете страни. Срещнахме се и обсъдихме работата. От името на Събранието аз изказах всички аргументи, които могат да се намерят в обществените вестници\footnote{public papers??} от онова време, които бяха написани от мен, и са отпечатани в стенограмите\footnote{minutes} на Събранието; а губернаторът пледира като посочи инструкциите си; клетвата, която е положил, да ги спазва, и че ще бъде разорен, ако не се подчини, и въпреки това изглежда беше готов да рискува\footnote{да се жертва/ to hazard himself}, ако лорд Лоудаун\footnote{Loudoun} го посъветва да го направи. Неговa светлост избра да не го направи, въпреки че в един момент ми се струваше, че съм го убедил да го нарпави; но най-накрая реши да препоръча Събранието да отстъпи; и ме помоли да използвам влиянието си с тях за тази цел, като заяви, че няма да щади кралските войски за защитата на нашите граници, и че ако не продължим да плащаме за защитата си, те ще станат уязвими откъм вражески набези\footnote{expos'd to the enemy}.

Запознах Камарата със станалото и след като им представих набор от резолюции, които бях съставил, с които заявявахме правата си, както и че не се отказваме от тези си права, а само временно спираме да ги упражняваме в този случай по принуда, срещу което протестираме, те се съгласи да се откажат от проектозакона и да съставят друг, който отговаря на условията на собственика. Губернаторът разбира се прие последния, а аз можех да продължа пътуването си. Но междувременно корабът\footnote{the paquet} беше отплавала със запасите ми за преминаването, което беше известна загуба за мен, а единствената компенсация ми беше благодарността на негова светлост за услугите ми, като цялата заслуга за постигнатото споразумерие беше приписана на него.

Той тръгна за Ню Йорк преди мен; и тъй като той определяше кога плават лодките\footnote{paquet-boats}, а там бяха останали две, една от които, според думите му, щяла да отплава съвсем скоро, пожелах да ми каже точното време, за да не я изпусна поради някакво мое забавяне. Отговорът му беше, “Наредих да плава следващата събота; но между нас мога да ти кажа, че ако пристигнеш до понеделник сутринта ще си навреме, но не отлагай повече.” Поради някакво случайно забавяне при един ферибот, пристигнах чак в понеделник по обед и много се опасявах, да не е отплавала, тъй като вятърът беше благоприятен; но скоро се успокоих като научих, че още не е отплавала и няма да отплава до следващия ден. Човек би предположил, че съвсем скоро щях да отплавам към Европа. Така си мислех; но по това време не бях добре запознат с характера\footnote{character} на негова светлост, една от чийто най-силни черти беше нерешителността. Ще дам няколко примера. Беше към началото на април, когато пристигнах в Ню Йорк, а мисля, че стана близо краят на юни преди да отплаваме. По това време имаше две лодки\footnote{paquet-boats}, които дълго бяха стояли в пристана\footnote{in port}, но изчакваха\footnote{were detained} for писмата на генерала, които винаги щяха да се готови утре. Още една лодка \footnote{paquet} пристигна; тя също беше задържана; а преди да отплаваме се очакваше четвърта. Нашата беше първата, която щеше да отплава, тъй като беше стояла в пристанището най-дълго. Във всичките имаше пътници, а някои от тях очакваха отпътуването си с изключително нетърпение, а търговците се тревожеха за писмата си и застрахователните си нареждания\footnote{the orders they had given for insurance} (беше военно време) за есенните стоки\footnote{fall goods!} но тревогата им не помагаше; писмата на негова светлост не бяха готови; и въпреки това всеки, който го посетеше, винаги го намираше на писалището, с писалка в ръка, и заключваше, че той трябва да пише изобилно\footnote{abundantly}.

Една сутрин, отивайки сам да го поздравя\footnote{to pay my respects}, видях в приемната му един Инис\footnote{Innis}, вестоносец от Филаделфия, който беше дошъл от там спешно с пратка от губернатор Дени за генерала. Предаде ми някои писма от приятелите ми там, което ме накара да го попитам кога ще се върн, къде е отседнал, за да мога да пратя някои писма по него. Каза ми, че има нареждане да остане до девет сутринта на следващия ден за отговора на генерала към губернатора, и смята да тръгне незабавно. Дадох му писмат си още на същия ден. Две седмици по-късно го срещнах на същото място. “Скоро ли се върна, Инис?”, “Върнал! Не, още не съм заминал.” “Как така?” “Всеки ден от две седмици идвам тук за писмото на негова светлост по заповед, но още не е готово.” “Как е възможно това, при положение, че е такъв писател? защото винаги го виждам на писалището му.” “Да”, казва Инис, “но той е като свети Георги по знаците, винаги на кон, но никога никога не тръгва\footnote{never rides on}!” Изглежда, че това наблюдение на вестоносеца беше добре обосновано; защото в Англия научих, че г-н Пит посочил като една от причините за отзоваването на този генерал и изпращането на генералите Амхърст и Уолф\footnote{Wolfe}, че министъра никога не получавал вести от него и нямало как да знае какво прави.

Поради ежедневното очакване да отплават, и това, че и трите лодки\footnote{paquets} бяха отишли в Санди Хук\footnote{Sandy Hook (къде е това?)}, за да се присъединят към флота там, пътниците смятаха, че е най-добре да са на борда, да не би корабите да отплават с внезапна заповед и да ги оставят на сушата. В това положение бяхме шест седмици, ако си спомням правилно, използвайки морските си запаси и принудени да си набавим допълнителни. Най-накрая флотилията\footnote{the fleet} отплава към Луисбърг\footnote{Louisburg}, заедно с генерала и неговата армия, с намерението да обсади и превземе крепостта; на всички лодки\footnote{paquet-boats} в групата им беше наредено да са близо\footnote{to attend} до кораба на генерала и да са в готовност да получат съобщения, ако има такива. Бяхме на море пет дни преди да получим писмо с позволение да отплаваме, при което корабът ни напусна флотилията и се насочи към Англия. Другите две лодки\footnote{paquets} той задържа, отвел ги с него до Халифакс, където прекарал известно време, за да упражнява мъжете си в тренировъчни нападения срещу тренировъчни укрепления, след това размислил относно обсадата на Луисбург\footnote{Louisburg} и се върнал в Ню Йорк с всичките си войски и с двете лодки\footnote{paquets} споменати по-горе, заедно с всичките им пътници! По време на отсъствието му французите и диваците бяха превзели Форт Джордж, който се намира на границата на тази провинция, а диваците бяха изклали мнозина от гарнизона след като се предали. 

След това видях в Лондон капитан Бонел\footnote{Bonnell}, който беше капитан на една от тези лодки\footnote{paquets}. Той ми каза, че след като бил задържан един месец, уведомил негова светлост, че по (дъното на) кораба му се образувал полип\footnote{his ship was grown foul}, дотолкова, че ще възпрепятства бързото му плаване, нещо което е от значение за една\footnote{paquet-boat}, и поискал да му бъде отпуснато време, за да я изкара на сухо\footnote{heave her down} и да изчисти дъното. Той попитал колко време ще отнеме. Той отвърнал, три дни. Генералът отговорил, “Ако можеш да го направиш в рамките на един ден, позволявам; иначе – не; защото вдругиден със сигурност трябва да отплаваш.” Така и не получил позволение, въпреки че бил задържан ден по ден цели три месеца. 

Също в Лондон срещнах един от пътниците на, който беше така ядосан на негова светлост за това, задето толкова дълго го заблуждавал и задъжал в Ню Йорк, и за това че го разкарал до Халифакс и обратно, че се кълнеше, че ще го съди за нанесени вреди. Дали го е направил или не, не научих; но както той го описваше, вредата\footnote{injury} нанесена на неговите работи беше много значителна.

Като цяло много се чудих как на такъв човек е била поверена такава важна работа като ръководството на голяма армия; но понеже оттогава видях повече от широкия свят и от средствата за постигане и мотивите за даване на назначения, учудването ми намаля. Ако беше оставен на този пост, Генерал Шърли, на когото се падна управлението на армията след смъртта на Брадък, според мен щеше да проведе много по-успешна кампания от тази на Лоудаун\footnote{Loudoun} от 1757, която беше толкова придирчива\footnote{frivolous}, скъпа и срамна за нацията ни, че е трудно да се проумее; защото въпреки че Шърли не беше учил за войник\footnote{not a bred soldier}, беше съобразителен\footnote{sensible} и мъдър\footnote{sagacious} по природа, вслушваше се в добрите съвети на другите, умееше да съставя разумни\footnote{judicious} планове и беше бърз и деен в изпълнението им. Вместо да защити колониите с голямата си армия, Лоудаун\footnote{Loudoun} ги остави напълно беззащитни докато безцелно маршируваше\footnote{paraded idly} в Халифакс, поради което Форт Джордж бе загубен, а освен това изкара от релси всичките ни търговски операции и смути търговията като наложи дълго ембарго над износа на провизии, уж за да се предотврати попадането на запаси в ръцете на врага, но в дейстителност, за да се свали цената им в интерес на прекупвачите, в чийто печалби той имал дял, както се говореше може би само въз основа на подозрения. И когато накрая ембаргото беше вдигнато, като забрави да изпрати вести за това в Чарлстън, Каролинската флота беше задържана близо три месеца повече, от което дъната така бяха повредени от червея, че голяма част от нея потъна\footnote{foundered} при преминаването към вкъщи. 

Вярвам, че Шърли беше истински облекчен\footnote{sincerely glad}, че са го освободили от нещо толкова тежко за човек неопитен във военните работи\footnote{unacquainted with military business} като ръководството на една армия. Бях на една забава организирана от град Ню Йорк в чест на лорд Лоудаун\footnote{Loudoun} при встъпването му в длъжност като командващ. Въпреки че с това биваше заменен, Шърли също беше там. Имаше много офицери, граждани и непознати и понеже някои столове бяха взети назаем от съседите, един от тях, доста нисък, се падна на г-н Шърли. Забелязвайки това докато седях до него, му казах, “Дали са ви, сър, твърде ниска позиция\footnote{low seat}.” “Няма значение,” казва той, “г-н Франклин, намирам ниската позиция за най-удобна\footnote{the easiest}.”

Докато бях задържан в Ню Йорк, както вече беше споменато, получих всички сметки за провизиите и т.н., които бях доставил на Брадък, някои от които сметки не можеха да бъдат получени по-рано от различните хора, които бях наел за помощници в тази работа. Представих ги на лорд Лоудаун\footnote{Loudoun}, като пожелах да ми бъдат възстановени разходите. Той ги даде на определения офицер да ги прегледа, който потвърди, че всичко е точно, след като сравни всеки артикул\footnote{item} с ваучера му; потвърди и общата сума, за която негова светлост обеща да ми даде платежно нареждане до ковчежника\footnote{paymaster}. Това обаче беше отложено няколко пъти;  и въпреки че често го посещавах с уговорка да го получа\footnote{called often for it by appointment}, така и не го получих. В крайна сметка, точно преди отпътуването ми, той ми каза, че след като отново размислил, решил да не бърка сметките си с тези на предшествениците си. “А когато си в Англия,” казва той, “само трябва да покажеш сметките си в хазната, и веднага ще ти бъде платено.”

Изтъкнах големите и неочаквани разходи, в които бях вкаран заради забавянето ми в Ню Йорк, като причина да искам да ми бъде платено незабавно, но напразно; и когато отбелязах, че не е справедливо да бъда допълнително забавян или затрудняван в получаването на парите, които бях похарчил, тъй като не вземам комисионна услугите си, той каза “О, сър, не трябва да си мислите, че можете да ни убедите, че не сте спечелил; разбираме по-добре от тези работи и знаем, че всеки, който е зает със снабдяването на армията, намира начин същевременно да напълни и собствените си джобове.” Уверих го, че в моя случай не е така, и че не съм прибрал нито стотинка; но той очевидно не ми повярва; и оттогава установих, че дейсвително често някои си докарват голямо състояние от такъв вид работа. Що се отнася до моите пари, и до ден днешен не са ми платени, за което повече ще разкажа по-долу.

Преди да отплаваме, капитанът на лодката\footnote{the paquet} много се хвали, с бързината на кораба си; за съжаление, когато излязохме на море, за негов голям срам тя се оказа най-бавната\footnote{the dullest} лодка of ninety-six sail\footnote{??}. След много догатки относно причината, когато бяхме близо до друг кораб почти толкова бавен, колкото нашия, който обаче ни изпреварваше, капитанът нареди всички да отидат на кърмата и да стоят колкото се може по-близо до the ensign staff\footnote{??}. Заедно с пътниците бяхме общо около четиридесет човека. Докато стояхме там, корабът оправи\footnote{mended her pace} хода си, и скоро остави спътника ни назад, което ясно доказа подозренията на капиатана, че е натоварена твърде много в предната част. Изглежда всички бурета с вода бяха поставени в предната част; затова нареди да ги преместят по към кърмата, след кеото корабът възстанови качествата си\footnote{recovere'd her character} и се оказа най-бързият в флотилията\footnote{the fleet}. 

Капитанът каза, че веднъж се движил с тринадесет възела, което е същото като\footnote{is accounted} тринадесет мили в час. На борда като пътник беше капитан Кенеди от Флота, който тъврдеше, че е невъзможно, и че никой кораб никога не е плавал толкова бързо, и че трябва да е имало някаква грешка в измервателното въже или грешка в начина по който е хвърлена гредата\footnote{heaving the log}. Това доведе до облог между двамата капитани, който щеше да се разреши, когато има достатъчно вятър. Кенеди прегледа внимателно въжето и като остана доволен, реши сам да хвърли гредата. Съответно няколко дни по-късно, когато вятърът беше много хубав и свеж\footnote{fair and fresh} и капитанът на кораба\footnote{paquet} каза, че вярва, че в този момент се движи с тринадесет възела, Кенеди проведе експеримента и призна, че е загубил облога.

Споменавам това, поради следното наблюдение. Изтъквано е като недостатък на изкуството на корабостроенето, че никога не може да се знае предварително, преди да бъде изпробван, дали един нов кораб ще плава добре или не; защото се е случвало моделът на бързоходен\footnote{good-sailing} кораб да бъде точно пресъздаден в нов, който се оказва забележително бавен. Струва ми се, че това може да е отчасти причинено от различните мнения на моряците\footnote{seamen} относно правилния начин за товарене, такелажа и плаването; всеки си има собствена система; и един и същ съд ще плава по-добре или по-лошо, ако е натоварен според указанията на един или друг капитан. Освен това почти никога не се случва един и същ човек да построи кораба, да го подготви за морето, и да плава с него. Един строи  корпуса, друг снабдява кораба с такелаж, а трети го товари и плава с него. Нито един от тях няма възможността да познава всички идеи и целия опит на останалите и по тази причина на е в състояние да прави правилни заключения базирани на тяхната цялостна комбинация. 

Дори при просто действие като плаване в морето често съм забелязвал различни преценки сред офицерите, които командват последователни смени при един и същ вятър. Един ще постави платната по-косо, а друг по срещу вятъра, така че изглежда нямат някакво определено правило, от което да се ръководят. При все това си мисля, че може да се проведе поредица от опити, първо, за да се определи най-правилната форма на корпуса за бързо плаване; след това, най-удачните размери и разположение на мачтите; след това геометрията и броят на платната, тяхното разположение според вятъра; и най-накрая подредбата на товара. Това е епоха на експерименти и мисля, че една поредица точно проведени и съчетани опити ще бъде много полезна. По тази причина съм уверен, че не след дълго някой хитроумен философ ще се захване с нея и му желая успех.

На няколко пъти по време на преминаването ни ни преследваха, но изпреварихме всичко, и след тридесед дни проведохме първите дълбочиннит измервания. Имахме добро число\footnote{a good obsrvation} и капитанът сметна, че сме толкова близо до пристанището ни, Фолмаут\footnote{Falmouth}, че ако направим добър пробег през нощта, ще успеем да стигнем до входа на пристанището до сутринта, а като плаваме през нощта ще избегнем вражеските наемници, които често патрулираха близо до входа на Ламанша\footnote{the channel}. Съответно всички платна, които бяха налични, бяха опънати\footnote{set}, и понеже вятърът беше много свеж и хубав\footnote{fresh and fair}, пуснахме се точно по него и бързо напредвахме. Капитанът, след замерването, мислеше, че така е нагласил курса, че да мине далеко от островите Сили\footnote{Scilly Isles}; но излгежда, че понякога има силно течение нагоре по Сейнт Джорджс Ченъл\footnote{St. George's Channel}, което заблуждава моряците и причини загубата на ескадрата на Сър Клаудсли Шавъл\footnote{Sir Cloudsley Shovel}. Това течение вероятно стана причина за това, което ни се случи. 

На носа на кораба имахме наблюдател, на който често подвикваха, “гледай внимателно там отпред,” а той винаги отговаряше, “Да да/тъй вярно\footnote{ay ay}”; но може би беше със затворени очи и полузаспал в това време, и отговаряше механично, както се казва; защото не видя фар точно пред нас, който беше скрит от човека на мостика от част от платната\footnote{studdingsails}, както и от осталите наблюдатели, но бе видян поради случайно накланяне на кораба, и причини голям смут, понеже бяхме много близо до него (на мен ми се стори голям колкото колелото на каруца\footnote{cart-wheel}). Беше полунощ, а капитанът ни спеше; но капитан Кенеди като скочи на паулбата и видя опасността, нареди да се обърне корабът с всички платна както си бяха; действие опасно за мачтите, но ни извади от опасността и избегнахме крушение, защото се бяхме засилили точно срещу скалите, на които стоеше фара. Това измъкване остави у мен силно впечатление относно голямата полза от морските фарове и ме накара да взема решение да насърча изграждането на повече такива в Америка, ако доживея да се върна там. 

На сутринта по измерванията на дълбочината и т.н. установихме, че сме близо до пристанището ни, но гъста мъгла криеше сушата от нас. Около девет часа мъглата започна да се вдига от водата както се вдига завеса в театъра и откри отдолу градът Фолмаут\footnote{Falmouth}, съдовете в пристанището и полетата наоколо. За тези, които дълго не бяха виждали друга гледка освен равномерния изглед на празния океан, това беше особено приятен спектакъл, и ни достави още по-голямо удоволствие, тъй като сега бяхме освободени и от тревогите причинени от войната.

Незабавно тръгнах към Лондон със сина ми и само за малко спряхме по пътя, за да видим Стоунхендж в равнината до Салисбъри\footnote{Salisbury Plain} и къщата и градините на лорд Пемброук\footnote{Pembroke} в Уилтън и много любопитните му старинни предмети\footnote{antiquities}. Пристигнахме в Лондон на 27 юли 1757 г.

[16] Тук приключва Автобиографията във вида, в който е публикувана от У. Темпъл Франклин и наследниците му\footnote{his successors}. Последващият текст е написан в последната тодина от живота на Д-р Франклин и беше отпечатано за първи път (на английски) в изданието на г-н Биглоу\footnote{Bigelow} през 1868 г. – бел. ред.

\chapter{Част четвърта}

Щом се установих в квартира, която ми беше предоставена от г-н Чарлз, веднага отидох да посетя д-р Фодъргил\footnote{Fothergill}, на когото бях силно препоръчан, и от когото ме бяха посъветвали да потърся съвет относно начина по който да подходя. Той съветваше да не се подава незабавно оплакване към правителството и смяташе, че първо трябва да се усъществи личен контакт със собствениците, които с помощта и убедителността на някои приятели може би ще бъдат склонени да разрешат работите чрез споразумение\footnote{amicably}. След това се обадих на стария си приятел и кореспондент, г-н Питър Колинсън, който ми каза, че Джон Хенбъри\footnote{Hanbury}, големият търговец от Вирджиния, е поискал да го информират, когато пристигна, за да ме заведе при лорд Гренвил\footnote{Granville}, който по това време беше президент на Съвета\footnote{the Council} и желаел да ме види възможно най-скоро. Съгласих се да отида с него на другата сутрин. Съответно г-н Хенбъри дойде да ме вземе и ме отведе с колата\footnote{carriage} си до дома на този благородник, който ме посрещна много учтиво\footnote{with great civility}; и след някои въпроси относно тогавашното състояние на работите\footnote{state of affairs} в Америка и разисквания по същата тема ми каза: “Вие американците имате погрешна представа относно естеството на политическото ви устройство\footnote{your constitution}; вие твърдите, че инструкциите на краля към неговите губернатори не са закони, и мислите, че имате право да решавате дали да им обръщате внимание или не. Но тези инструкции не са като джобните инструкции, които се дават на един министър\footnote{minister}, който ще пътува в чужбина, за да знае той как да изпълни някоя незначителна част от етикета\footnote{ceremony}. Те се пишат първо от съдии, които познават законите; след това те биват разгледани, обсъдени и може би дори променени в Съвета, след което биват подписвание от краля. Така че що се отнася до вас, те са закон, защото кралят е ЗАКОНОДАТЕЛЯТ НА КОЛОНИИТЕ.” Казах на негова светлост, че тази доктрина е нещо ново за мен. Че винаги съм смятал, че според нашите харти законите ни се изработват от Събранията ни, предлагат се на краля за неговото кралско одобрение\footnote{royal assent}, но щом веднъж ги одобри, кралят не може да ги променя или отменя. И както Събранията не могат да приемат постоянни закони без негово съгласие, така и той не може да направи закон без тяхното. Той ме увери, че съм се объркал напълно\footnote{I was totally mistaken}. Аз обаче не бях на това мнение и понеже разговорът с негова светлост ме накара малко да се разтревожа относно дворцовите настроенията към нас, го записах щом се върнах в квартирата си. Спомних си, че около двадесет години по-рано в един проектозакон внесен в парламента от министерството имаше клауза, според която инструкциите на краля щяха да бъдат закон в колониите, но Камарата на общините я отхвърли, заради което ние я обожавахме като наши приятели и приятели на свободата, докато през 1765, с поведението си към нас, не започна да изглежда, че бяха отказали това суверенно право на краля, само за да го запазят за себе си.

Д-р Фодъргил говори със собствениците и след няколко дни те се съгласиха да се срещнат с мен в къщата на г-н Т. Пен в Спринг гардън. Разговорът започна с взаимни уверения\footnote{declarations} в готовност за постигане на разумно решение\footnote{reasonable accommodations}, но предполагам всяка страна имаше свое собствено разбиране за това какво е разумно. След това разгледахме отделните точки, по които ние имахме оплаквания, които аз изброих. Собствениците оправдаха поведението си доколкото можеха, а аз – това на Събранието. В този момент изглежда бяхме много отдалечени едни от други, а различията в мненията ни толкова големи, че не остана надежда за постигане на съгласие\footnote{to discourage all hope}. Въпреки това заключихме, че аз трябва да им представя основните ни оплаквания\footnote{the heads of our complaints} писмено, а те обещаха да ги разгледат. Направих го скоро след това, но те предадоха документа в ръцете на адвоката си, Фердинанд Джон Парис\footnote{Paris}, който се занимаваше от тяхно име с всичките им юридически дела\footnote{law business} по големият им съдебен спор\footnote{suit} със собственика на съседната им колония Мериленд, лорд Балтимор, който беше продължил седемдесет години, и пишеше всичките им документи\footnote{papers} и съобщения в спора им със Събранието. Той беше горд, гневлив\footnote{сърдит angry} човек, и понеже понякога в отговорите на Събранието се бях се отнасял към писмата му – които бяха наистина слабо аргументирани и с надменен тон – с известна острота some\footnote{severity}, той беше развил смъртна вражда\footnote{mortal enminty} спрямо мен, която се проявяваше винаги, когато се срещнехме, и поради която отказах предложението на собствениците да обсъдим основните оплаквания на четири очи между нас двамата, и отказах да преговарям с всеки друг освен с тях. При това, те предадоха по негов съвет писмото ми на министъра на правосъдието и неговия заместник\footnote{Attorney and Solicitor-General} за тяхното мение и съвет, където то прекара без отговор една година без осем дни, през което време аз често изисквах отговор от собствениците, но не получавах друг, освен този, че все още не са научили мнението на министъра и заместника му. Какво беше то така и не разбрах след като го получиха, защото не го споделиха с мен, а изпратиха дълго съобщение до Събранието, списано и подписано от Парис, в което бяха преписали моето писмо, и се оплакваха от грубостта ми, поради липсата на формалност, предлагаха несъстоятелни оправдания за поведениято си, и добавяха, че биха желали да постигнат споразумение, ако Събранието изпрати някой честен човек, който да преговаря с тях за тази цел, с което даваха да се разбере, че аз не съм такъв. 

Вероятно липсата на формалност или грубостта се състоеше в това, че не се бях обърнал към тях в писмото със званията им\footnote{their assumed titles}, Истински и абсолютни собственици на провинцията Пенсилвания, които аз изпуснах мислейки ги за ненужни в писмо, чиято цел беше само да потвърди писмено това, което бях казал устно.

Но докато чаках техния отговор, Събранието убеди губернатор Дени да одобри закон, който облагаше имотите на собствениците наравно с тези на хората, което беше основната точка в нашия спор, и затова не отговори на съобщението. 

Когато обаче законът пристигна\footnote{came over}, собствениците, по съвета на Парис, решиха да попречат на това той да получи одобрението на краля. Съответно те се обърнаха към личния Съвет на краля\footnote{privy council} и беше насрочено изслушване, за което те наеха двама адвокати, които да говорят срещу закона, а аз двама, които да го подкрепят. Те твърдяха, че целта на закона е да обложи собственическите имоти с по-високи данъци, за да се облекчи бремето на хората, и ча ако бъде оставен в сила, а собствениците оставени на благоволението на хората, които ги мразеха, при определяне на данъците, неизбежно ще бъдат разорени. Отвърнахме, че законът въобще не цели това и няма да има такива последствия. Че оценителите са честни и благоразумни\footnote{discreet} хора и са дали клетва да оценяват справедливо и безпристрастно, и че ползите, които всеки от тях може да очаква от намаляване на собствения си данък за сметка на този на собствениците, са твърде незначителни, за да ги накарат да изменят на клетвата си. Това бяха по същество позициите на двете страни, доколкото си спомням, като изключим това, че силно настояхме, че отменянето на закона ще има вредни последствия, защото парите, които бяха отпечатани и дадени на краля за негова употреба и изхарчени за неговите дела, сто хиляди паунда, сега бяха у хората, и една отмяна ще заличи стойността им и ще разори мнозина и напълно ще обезкуражи отпускането на пари в бъдеще, и с най-силни думи настояхме, че е себично от страна на собствениците да желаят такава всеобща катастрофа само заради неоснователния страх, че имотът им ще бъде обложен твърде високо. След като тези неща бяха казани, лорд Мансфийлд\footnote{Mansfield}, член на съвета, се изправи и след като ме повика ме отведе в стаята на писаря, докато адвокатите пледираха, и ме попита дали наистина вярвам, че при изпълнението на закона няма да бъдат ощетени имотите на собствениците. Казах, че съм сигурен. “Тогава”, казва той, “не бихте имали големи възражения, ако ви помолят да се задължите\footnote{enter into an engagement}, за да гарантирате за това.” Отвърнах, “Никакви.” След това той извика Парис и след кратко обсъждане предложението на негова светлост беше прието и от двете страни; Писарят на Съвета съчини документ за тази цел, който подписах заедно с г-н Чарлз, който също беше Представител\footnote{Agent} на провинцията в обикновените им дела, след което лорд Мансфийлд се върна в Залата на съвета, където най-накрая новият закон беше одобрен\footnote{allowed to pass}. Препоръчани бяха обаче някои промени, за които се задължихме, да бъдат приети от последващ закон, но Събранието не сметна това за нужно; понеже преди да дойде заповедта от Съвета вече бяха събрани данъците за годината, те назначиха комитет, който да прегледа действията на оценителите, като в този комитет назначиха някои близки приятели на собствениците. След пълно разследване\footnote{full inquiry}, челновете на комитета единодушно подписаха доклад, според който са установили, че данъците са били определени съвършено справедливо. 

Събранието сметна това, че се бях задължил по първата точка, за значителна заслуга към провинцията\footnote{essential service}, понеже осигури стойността на хартиените пари разпространени по това време и зцялата провинция. Благодариха ми официално\footnote{in form}, когато се завърнах. Собствениците, обаче, бяха бесни на губернатор Дени, задето беше одобрил закона, и го уволниха със заплахи, че ще го съдят за нарушаване на инструкциите, които се беше задължил\footnote{given bond} да спазва. Понеже го беше направил по настояване на Генерала и в услуга на Негово Величество, и понеже имаше някои влиятелни познати в двора, той пренебрегна заплахите, а те никога не бяха осъществени. . . [Незавършена]
\end{document}
